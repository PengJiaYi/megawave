This module allows to visualize the Fast Fourier Transform
\index{Fourier Transform}
of a real Fimage (or the inverse FFT is the -i option is set).
\begin{itemize}
\item $type$ specifies what will be displayed~:
\begin{itemize}
\item[$\circ$] $type=0$ means the modulus $M$ of the Fourier Transform (default)
\item[$\circ$] $type=1$ means the phase of the Fourier Transform
\item[$\circ$] $type=2$ means the real part of the Fourier Transform
\item[$\circ$] $type=3$ means the imaginary part of the Fourier Transform
\item[$\circ$] $type=4$ means the log power of the Fourier Transform, i.e. $log(1+M)$
\end{itemize}
\item A gain-offset correction is applied to the output to be visualized
in order to obtain a mean of 128 and a standart deviation of $sd$ 
($sd = 50.0$ by default)
\item You can apply a Hamming window to the input image before computing the
FFT by setting the -h option~: the artifacts created by the borders of a
non-periodic image (high energies in vertical and horizontal frequencies)
will be attenuated.
\item Last, you can decide the result to be displayed into a Cimage (instead 
of the screen) using the -o option.
\end{itemize}
