This module applies the so-called grain filter to an image \cite{masnou.morel:image}
\index{filter!grain}.
This is a filter close to the extrema killer 
\index{extrema killer}
of Vincent \cite{vincent:grayscale}, with
one major difference: it is self-dual, meaning that it commutes with the
negative operator,
$$
T(-u) = -T(u).
$$
Its other properties are mainly that it is morphological, affine invariant,
idempotent,
it does not destroy T-junctions and does not move level lines. It is a
semi-local operator, but not local (no associated PDE).
As it is a morphological filter, it can be written as an inf-sup, and the
sup-inf version is the same (it is simultaneously an opening and a closing).

This filter removes the shapes of an image (see module \verb+flst+) that
have a too small area. So this requires the datum of a threshold of area, this
area could be interpreted as the scale to define a scale-space
\cite{monasse.guichard:scale-space}\cite{monasse.guichard:fast}.
It is a good example of a very simple use of the modules \verb+flst+ and
\verb+flst_recontruct+.

