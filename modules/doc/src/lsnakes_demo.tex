This module provides a simple interface to the snakes 
algorithm~\cite{caselles.catte.ea:geometric}, by calling
as many time the module \verb+lsnakes+ as requested.
\index{snake}
It moves the initial snakes (given as polygons) according to the content of an
image \verb+u+, in order to fit the contours of the object enclosed in each polygon.

The output is the object {\em out}, a movie of char images.
Each image of this movie represents the original image to which are superimposed
the snakes stretched at the current master iteration of the algorithm:
the first image displays the initial snakes, the second image the snakes after
the first call to \verb+lsnakes+, the third one the snakes after the second call
to \verb+lsnakes+, and so on up to {\em Nframes}, the number of frames selected by the user.
In addition to this master iteration, you can set several iterations of
\verb+lsnakes+ between two adjacent frames ({\em Niter}), using the \verb+-n+ option.

At the beginning of this module, the module \verb+readpoly+ is called and
displays the image. Please select the initial snakes using the mouse (type 'H'
in the window to get an one-line help). 
When you terminate your selection by typing 'Q', the computation begins.
At the end of the process, you can display the result using the module
\verb+cmview+.
