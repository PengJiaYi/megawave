The module \texttt{flstb\_quantize} is used to extract the shapes at some levels
in a bilinear interpolated image\index{quantization!FLST}. 
Actually, it is based on the tree of shapes
extracted by \texttt{flst\_bilinear} (see documentation for this module) and
provides with the tree of shapes associated to level lines at given levels.

The exception to this is the root which keeps its original gray level.
Notice that if the values of some data points are not in the set of
quantization levels, the original image cannot be reconstructed from the
extracted tree.

For example, for an image with integral gray values at data points, it could
be useful to extract the tree of shapes at half integral levels, since the
associated level lines are likely to be more regular than the ones associated
to integral levels.
