Let $NC$={\em image\_in-$>$ncol} and $NL$={\em image\_in-$>$nrow}, then
the image {\em image\_in} is defined by the set of pixels
$\{ g(i,j)\, /\, i=0, \ldots, NC\, , j=0, \ldots, NL\}$.
The module {\bf fvalues} computes the range of the function 
$g\,: [0,NC-1]\times[0,NL-1]\longrightarrow {\rm I\!\!R}$\,.

\smallskip

The output is an {\tt fsignal} of size $N+1$, with float values 
$\{v_0, v_1, \ldots, v_N\}$, where $N+1$ is the number of different
values in {\em image\_in} ($1\leq N+1\leq NC*NL$).

\smallskip

The module uses a heap sort to put the values in increasing order
and decreasing if the {\bf -i} flag is set.

\smallskip

The {\bf -m} option allows to save the multiplicities of the 
corresponding values in {\em values} into the {\tt fsignal} 
{\em multiplicity} and thus allows to compute the histogram
of {\em image\_in}.

\smallskip

The {\bf -r} option computes an {\tt fimage} {\em rank}, 
this array contains the indexes of the gray values in 
the ordered list {\em values}, {\em i.e.} the rank of 
a pixel's gray value.

\smallskip

Example : if $g(i,j)=v_n$ then {\em values$[n]=v_n$}
and {\em rank$(i,j)=n$}\,, moreover {\em multiplicity$[n]$}
gives the number of occurrences of the value $v_n$ in the image.
