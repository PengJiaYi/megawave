This module extract level lines from a Fimage.
\index{level line}

\begin{itemize}
\item if the \verb+-z+ option is used, the (zero-order) flst is first
computed with area parameter $area$ 
(\verb+flst+ module), and then all level lines are extracted
with the \verb+flst_boundary+ module.
These lines are made of concatenations of segment of length 1. 
As always when using the zero-order flst, be sure that your image is
reasonably quantized before calling this module, since all existing
grey levels are kept in the computation of the flst.
\index{interpolation!nearest neighbor}
\item if not, then the bilinear flst (\verb+flst_bilinear+ module) is used 
and the resulting tree is quantized on the grid $levels$ or
$offset + n \times step$
with the \verb+flstb_quantize+ module. Then, all level lines corresponding
to these levels are computed and sampled with approximately $prec$ points
per pixel with the \verb+flstb_boundary+ module.
\index{interpolation!bilinear}
\end{itemize}

The resulting fcurves are stored in a Flists. In both cases, the flst tree
can be precomputed and passed to the program via the \verb+-t+ option.