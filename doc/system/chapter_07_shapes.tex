\Name{mw\_alloc\_shapes}{Allocate the fields of a \shapes\ structure}
\Summary{
Shapes mw\_alloc\_shapes(shs, nrow, ncol, value)

Shapes shs;

int nrow, ncol;

float value;     /* gray level value of the root */
}

\Description
This function takes as argument a \shapes\ structure and returns it
after having allocated all necessary fields. The input \verb+nrow+ and 
\verb+ncol+ are the dimensions of the image. 
The field \verb+the_shapes+ is allocated to contain
\verb+nrow+$\times$\verb+ncol+$+1$ shapes, which is the maximal number of 
shapes extracted by the FLST (see module \verb+flst+). 
In fact, only one shape is put, the root of the tree, supposed to be 
extracted at gray level \verb+value+. 
The field \verb+smallest_shape+ is also allocated and
initialized, each pixel having as smallest shape the root.

The function returns \verb+shs+, or \Null\ if not enough memory is available
to do the allocation.

\Next
\Example
\begin{verbatim}

Shapes shs;
Fimage image; /* Assume image is allocated */

/*
  Define the structure
*/
shs = mw_new_shapes();
if (!shs) mwerror(FATAL,1,"Not enough memory to define the shapes !\n");

/* 
  At that time, the structure exists but fields are empty : alloc them
  to handle the Fimage image.
*/
if (!mw_alloc_shapes(shs, image->nrow, image->ncol, image->gray[0]))
  mwerror(FATAL,1,"Not enough memory to alloc the shapes !\n");

\end{verbatim}

\newpage %......................................


\Name{mw\_change\_shapes}{(Re)alloc the fields of a \shapes\ structure}
\Summary{
Shapes mw\_change\_shapes(shs, nrow, ncol, value)

Shapes shs;

int nrow, ncol;

float value;     /* gray level value of the root */
}

\Description
If the input pointer \verb+shs+ is \Null, create a new
structure, otherwise delete the currently allocated fields 
(if any) and call \verb+mw_alloc_shapes()+.

The function returns the new structure or \verb+shs+, or \Null\ if not enough 
memory is available to do the allocation.

\Next
\Example
\begin{verbatim}

Shapes shs=NULL;
Fimage image; /* Assume image is allocated */

/*
  Define the structure and alloc the field to handle the Fimage image.
*/
shs = mw_change_shapes(shs, image->nrow, image->ncol, image->gray[0]);
if (!shs) mwerror(FATAL,1,"Not enough memory to alloc the shapes !\n");

\end{verbatim}

\newpage %......................................


\Name{mw\_delete\_shapes}{Delete a \shapes\ structure}
\Summary{
void mw\_delete\_shapes(shs)

Shapes shs;

}

\Description
This function frees the allocated fields and the
structure itself. 
After this call, the memory pointed to by \verb+shs+ must
not be accessed any longer.
Warning : in the contrary to \verb+mw_delete_shape()+, the memory of the 
user-defined field \verb+data+ is not freed.
If this field has been allocated, you should free it before calling
\verb+mw_delete_shapes()+.


\Next
\Example
\begin{verbatim}

Shapes shs=NULL;
Fimage image; /* Assume image is allocated */

/*
  Define the structure and alloc the field to handle the Fimage image.
*/
shs = mw_change_shapes(shs, image->nrow, image->ncol, image->gray[0]);
if (!shs) mwerror(FATAL,1,"Not enough memory to alloc the shapes !\n");

/*
  ... (do the computation) ...
*/

/* 
  Delete the shapes
*/
if (!shs->data) free(shs->data);
mw_delete_shapes(shs);

\end{verbatim}

\newpage %......................................


\Name{mw\_new\_shapes}{Create a \shapes\ structure}
\Summary{
Shapes mw\_new\_shapes()

}
\Description
This function creates a new \shapes\ structure. The fields are initialized
to $0$ or \Null value.
The function returns the address of the new structure, or
\Null\ if not enough memory is available.

\Next
\Example
\begin{verbatim}

Shapes shs;

/*
  Define the structure
*/
shs = mw_new_shapes();
if (!shs) mwerror(FATAL,1,"Not enough memory to define the shapes !\n");

/* 
  At that time, the structure exists but is empty.
*/
\end{verbatim}

\newpage %......................................


