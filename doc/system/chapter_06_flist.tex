\Name{mw\_change\_flist}{Define and allocate a \flist\ structure}
\Summary{
Flist mw\_change\_flist(l,max\_size,size,dim)

Flist l;
int max\_size,size,dim;
}
\Description
This function changes the memory allocation of the \verb+values+ array
of a \flist\ structure, even if no previously memory allocation was done.
The new size (number of elements) of the structure is given by \verb+size+,
the size to allocate (maximal number of elements) by \verb+max_size+,
and the dimension by \verb+dim+.

It can also create the structure if the input \verb+l = NULL+.
Therefore, this function can replace both \verb+mw_new_flist+ and
\verb+mw_realloc_flist+. 
Since the function can set the address of \verb+l+, the variable must be set 
to the return value of the function (See example below).

The function \verb+mw_change_flist+ returns \Null\ if not enough memory is 
available to allocate the structure or the  \verb+values+ array, and an error 
message is issued. 
Your code should check this return value to eventually send a fatal error 
message in the \Null\ case, and do appropriate statement.

\Next
\Example
\begin{verbatim}

Flist l;

/* 
   Allocate l to handle at most 10 samples of couples (2) of 
   floating point values, the default number of samples being 0. 
*/
l = mw_change_flist(NULL,10,0,2);
if (!l) mwerror(FATAL,1,"Not enough memory to continue !\n");


\end{verbatim}

\newpage %......................................


\Name{mw\_clear\_flist}{Clear the array of a \flist\ structure}
\Summary{
void mw\_clear\_flist(l,v)

Flist l;
float v;
}
\Description
This function clears the \verb+values+ array by filling it
with the value \verb+v+ (up to the current number of samples).

\Next
\Example
\begin{verbatim}

Flist l;

/* 
   Allocate l to handle at most 10 samples of couples (2) of 
   floating point values, the default number of samples being 5. 
*/
l = mw_change_flist(NULL,10,5,2);
if (!l) mwerror(FATAL,1,"Not enough memory to continue !\n");

/* 
  Clear the 5 current samples with 0.
*/
mw_clear_flist(l,0.0);


\end{verbatim}

\newpage %......................................


\Name{mw\_copy\_flist}{Copy a the array \flist\ structure}
\Summary{
Flist mw\_copy\_flist(in,out)

Flist in,out;
}
\Description
This function copies the \verb+values+ array and \verb+data+ field 
of the \flist\ structure \verb+in+ into \verb+out+. 
The duplicated \flist\ \verb+out+ is allocated to
at least the current size of \verb+in+.

Since the function can set the address of \verb+out+, the variable must be set 
to the return value of the function (See example below).

The function \verb+mw_copy_flist+ returns \Null\ if not enough memory is 
available to allocate the structure or the \verb+values+ array, and an error 
message is issued. 
Your code should check this return value to eventually send a fatal error 
message in the \Null\ case, and do appropriate statement.

\Next
\Example
\begin{verbatim}

Flist in,out=NULL;

/* 
   Allocate in to handle at most 10 samples of couples (2) of 
   floating point values, the current number of samples being 5. 
*/
in = mw_change_flist(NULL,10,5,2);
if (!in) mwerror(FATAL,1,"Not enough memory to continue !\n");

/* 
  Clear the 5 current samples with 1.
*/
mw_clear_flist(in,1.0);

/*
 Copy in into out. Allocated size for out is 5 samples.
*/
out=mw_copy_flist(in,out);
if (!out) mwerror(FATAL,1,"Not enough memory to copy flist !\n");


\end{verbatim}

\newpage %......................................


\Name{mw\_delete\_flist}{Delete the array and the \flist\ structure}
\Summary{
void mw\_delete\_flist(l)

Flist l;
}
\Description
This function deletes the \verb+values+ array and the structure itself.
Warning : the memory of the user-defined field \verb+data+ is not freed.
If this field has been allocated, you should free it before calling
\verb+mw_delete_flist+.

\Next
\Example
\begin{verbatim}

Flist l;

/* 
   Allocate l to handle at most 10 samples of couples (2) of 
   floating point values, the default number of samples being 5. 
*/
l = mw_change_flist(NULL,10,5,2);
if (!l) mwerror(FATAL,1,"Not enough memory to continue !\n");

/* 
  Allocate the data field for 20 integers.
*/
l->data_size=20*sizeof(int);
l->data= (int *)malloc(l->data_size);
if (!l->data) mwerror(FATAL,1,"Not enough memory to continue !\n");


/*
    ... (statement)...
*/

/* 
  Free the list, including data field. 
*/
free(l->data);
mw_delete_flist(l);


\end{verbatim}

\newpage %......................................
\Name{mw\_enlarge\_flist}{Enlarge the array of a \flist}
\Summary{
Flist mw\_enlarge\_flist(l)

Flist l;
}
\Description
This function performs a memory reallocation on the array
\verb+l->values+ to increase the number of elements that can be recorded.
The enlargement factor is fixed by the constant \verb+MW_LIST_ENLARGE_FACTOR+
defined in the include file \verb+list.h+.
This function is useful when one does not know by advance the size of
the list, and when one wish to avoid multiple reallocations.

If not enough memory is available to perform the reallocation, an error
message is issued and the function returns \verb+NULL+.
Otherwise, the function returns \verb+l+.


\Next
\Example
\begin{verbatim}


/* Fill a flist with diagonal points using mw_enlarge_flist 
   up to a random size, unknown by advance.
*/

Flist l; 

l = mw_change_flist(NULL,2,0,2);
if (l==NULL) mwerror(FATAL,1,"Not enough memory to continue !\n");
i=0;
do
  {
   if ((2*i == l->max_size) && (!mw_enlarge_flist(l)))
       mwerror(FATAL,1,"Not enough memory to continue !\n");
   l->values[i++] = l->values[i++] = i;
  }  while (rand() != 0);
l->size=(i+1)/2;
\end{verbatim}

\newpage %......................................


\Name{mw\_new\_flist}{Create a \flist\ structure}
\Summary{
Flist mw\_new\_flist()

}
\Description
This function creates a new \flist\ structure. The fields are initialized
to $0$ or \Null value.
The function returns the address of the new structure, or
\Null\ if not enough memory is available.

\Next
\Example
\begin{verbatim}

Flist l;

/*
  Define the structure
*/
l = mw_new_flist();
if (!l) mwerror(FATAL,1,"Not enough memory to define the list !\n");

/* 
  At that time, the FList is empty.
*/
\end{verbatim}

\newpage %......................................


\Name{mw\_realloc\_flist}{Realloc the array of a \flist}
\Summary{
Flist mw\_realloc\_flist(l,n)

Flist l;

int n;
}
\Description
This function performs a memory reallocation on the array
\verb+l->values+ so that at most $n$ elements can be recorded.

If not enough memory is available to perform the reallocation, an error
message is issued and the function returns \Null.
Otherwise, the function returns \verb+l+.

\Next
\Example
\begin{verbatim}
Flist l;

/* 
   Allocate l to handle at most 1000 samples of 500-tuple of 
   floating point values, the default number of samples being 1000. 
*/
l = mw_change_flist(NULL,1000,1000,500);
if (!l) mwerror(FATAL,1,"Not enough memory to continue !\n");

/*
    ... (statement)...
*/

/*
   Now we need space for 20 samples only : by doing reallocation,
   we allow to free some memory.
*/
l = mw_realloc_flist(l,20);
if (!l) mwerror(FATAL,1,"Couldn't realloc flist !\n");


\end{verbatim}

\newpage %......................................


