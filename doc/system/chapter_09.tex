%
% Part 9 of the MegaWave2 Guide #2
%   Miscellaneous Features
%

You will find in this section some utilities which may help you to write your modules. 
Contrary to the former sections, some functions described here are not about a memory 
format. 

%-----------------------------------------------
\section{Global System Variables}
%-----------------------------------------------


\label{misc_glob-sys-var}

At any time in a module, you can access to the following external variables\index{external variable}.
Those variables are for reading only, do not change their values !
Notice that you don't have to define those variables in your module, the
definitions are done into the include file \verb+mw.h+.

\begin{itemize}
\item \verb+char *mwname+\index{external variable!mwname} :
This variable contains the name of the current module.

\item \verb+char *mwgroup+\index{external variable!mwgroup} :
This variable contains the group name of the current module, as for example \verb+"common/signal"+ which means that the current module belongs to the subgroup \verb+signal+ which is part of the main group \verb+common+.

\item \verb+int mwerrcnt+\index{external variable!mwerrcnt} :
Give the number of time the function \verb+mwerror+ has be called with the
argument \verb+ERROR+ (see section~\ref{misc_function} page~\pageref{misc_function}).
Since \verb+ERROR+ is not a fatal event, the user has the possibility to terminate the algorithm by checking \verb+mwerrcnt+, if too many errors have been
encountered.

\item \verb+int mwrunmode+\index{external variable!mwrunmode} :
The value of this variable indicates in which context the module is executed.
\begin{itemize}
\item If set to $1$, the module is called in the run-time mode;
\item if set to $2$, the module is called by the window-oriented interpreter (XMegaWave2).
\end{itemize}


\end{itemize}

%+++++++++++++++++++++++++++++++++++++++++++++++
\section{Conversion between memory types}
%+++++++++++++++++++++++++++++++++++++++++++++++

\label{conv_memory_types}

The System Library contains functions to convert memory types\index{convert memory types}. 
However do not expect to find a function to convert structures
which are very dissimilar, as \verb+Curves+ and \verb+Cimage+.
If the meaning one can give of a conversion is not evident or not unique,
a conversion procedure has to be implemented as a module rather than 
as a system function.

Conversion function summaries follow the following rule :
\verb+out = (Y) mw_x_to_y(in,old)+ where \verb+x+ is the internal C type of the
input \verb+in+, \verb+y+ the internal C type of the requested output 
\verb+out+ (letters in lowercase) and \verb+Y+ the cast to the output 
(internal C type of \verb+out+ with first letter in uppercase).
In the last argument \verb+old+ you may put the name of a variable of type \verb+Y+ :
in such a case, the memory allocation will be reused for \verb+out+ (the pointer
\verb+old+ will have the same address than \verb+out+). This is especially useful
when converting lot of images with same size, to avoid memory blowup. 
If you do not want to use this possibility, just put \verb+NULL+ as the last argument :
memory for \verb+out+ will be allocated.

In addition to the various \verb+mw_x_to_y()+ undocumented conversion 
functions, there exists an ``all-purpose'' conversion function called 
\verb+mw_conv_internal_type()+ and documented next page.

\newpage

\Name{mw\_conv\_internal\_type}{Convert any possible internal type to another one}
\Summary{
void *mw\_conv\_internal\_type(mwstruct,typein,typeout)

void *mwstruct; /* Any type of MegaWave2 structure */

char *typein;   /* Type of the input <mwstruct> */

char *typeout;  /* Type of the output structure */
}

\Description
This function may be used instead of the \verb+mw_x_to_y()+ various functions 
to convert any possible internal type \verb+a+ to \verb+b+, even if the 
\verb+mw_a_to_b()+ function does not exist : the system creates 
\verb+mw_conv_internal_type()+ by analyzing existing \verb+mw_x_to_y()+ 
functions, by finding the shortest path between two internal types, say
\verb+a+ and \verb+b+, and by calling appropriate \verb+mw_x_to_y()+ functions
(for example, \verb+mw_a_to_c()+ and  \verb+mw_c_to_b()+ if those functions exist).

The input \verb+mwstruct+ is a variable of internal C type given by the
string \verb+typein+ (use lower letters only). 
The output of the function is a variable of internal C type given by the 
string \verb+typeout+, or \verb+NULL+ if the conversion is impossible.

Do not forget to cast the output to the right type.

\Next
\Example
\begin{verbatim}
Ccimage in;  
Cimage out1;
Fimage out2;

/* The line */
out1 = (Cimage) mw_conv_internal_type(in,"ccimage","cimage");
/* is equivalent to */
out1 = (Cimage) mw_ccimage_to_cimage(in);

if (out1==NULL) mwerror(FATAL,1,"Cannot convert Ccimage to Cimage !\n");

/* But to convert a Ccimage to a Fimage you shall use */
out2 = (Fimage) mw_conv_internal_type(in,"ccimage","fimage");
/* Since the following function does not exist at this time */
out2 = (Fimage) mw_ccimage_to_fimage(in);

if (out2==NULL) mwerror(FATAL,1,"Cannot convert Ccimage to Fimage !\n");

\end{verbatim}

\newpage %......................................


%+++++++++++++++++++++++++++++++++++++++++++++++
\section{Miscellaneous System Functions}
%+++++++++++++++++++++++++++++++++++++++++++++++

\label{misc_function}

The following is a description of some miscellaneous system functions 
which may be of interest for the user.
The list is in alphabetical order.

The most usefull are \verb+mwerror+ and \verb+mwdebug+. 
Please notice that you need to process any error (especially memory allocation failure) by displaying an error message using \verb+mwerror+, and by doing
appropriate statement.

Some other functions are about dynamic memory allocation\index{dynamic memory allocation}. 
They are important, since you are discouraged to use static memory allocation (as \verb+double data[10000]+), but you may skip their description if you are
familiar with the standard C dynamic memory functions.

\newpage %......................................

\Name{mwcalloc}{Dynamic memory allocation}
\Summary{
void *mwcalloc (nelem, elsize)

unsigned  nelem, elsize;
}
\Description
This function allocates space for an array of \verb+nelem+ elements,
each of size \verb+elsize+ bytes, and initializes the space to zeros.

This function returns a pointer to space suitably aligned (after possible pointer coercion) for storage of any type of object. 
Do not forget to cast the return value to the right type of your variable
(see example below).
If not enough memory is available to allocate the array, the function returns \Null.
Your code should check this return value to send an error message in the \Null\ case, and do appropriate statement.

Each space allocated by \verb+mwcalloc+ must be deallocated using \verb+mwcfree+ before exiting the module.

Notice that in the MegaWave2 modules, the standard C function \verb+calloc+ is
redefined to be \verb+mwcalloc+. 
Therefore, if you use \verb+calloc+ in your code you actually call \verb+mwcalloc+.

\Next
\Example
\begin{verbatim}
float *data=NULL; /* Internal use: no Input neither Output of module */

/* Allocates space for 1000 samples of float values */
if  ( (data = (float *) mwcalloc (1000, sizeof(float))) == NULL )
    mwerror(FATAL,1,"Not enough memory.\n");

/* Set the sample #i to the value i */
for (i=0;i<1000;i++) data[i] = i;
\end{verbatim}

\newpage %......................................

\Name{mwcfree}{Dynamic memory deallocation}
\Summary{
void mwcfree (ptr)

char *ptr;
}
\Description
This function deallocates the space pointed to by 
\verb+ptr+ and which has previously been allocated by \verb+mwcalloc+. It does 
nothing if \verb+ptr+ = \Null.

You should set \verb+ptr+ to \Null after this call since the address pointed to by \verb+ptr+ is no longer valid.

Notice that in the MegaWave2 modules, the standard C function \verb+cfree+ is
redefined to be \verb+mwcfree+. 
Therefore, if you use \verb+cfree+ in your code you actually call \verb+mwcfree+.

\Next
\Example
\begin{verbatim}
float *data=NULL; /* Internal use: no Input neither Output of module */

/* Allocates space for 1000 samples of float values */
if  ( (data = (float *) mwcalloc (1000, sizeof(float))) == NULL )
    mwerror(FATAL,1,"Not enough memory.\n");
.
. (statement)
.
/* End of statement: deallocation of the array */
mwcfree((char *) data);
\end{verbatim}

\newpage %......................................

\Name{mwdebug}{print if debug}
\Summary{
void mwdebug(format, \ldots);

char *format;

}
\Description
This function prints its arguments in \ldots under control of the format in \verb+format+, exactly in the same manner that the standard C function \verb+printf+ does. 
The string \verb+<dbg>+ is added to the beginning of the line.

The print is active only when the module has been called with the debugging option on.

\Next
\Example
\begin{verbatim}
Fimage image; 
int x,y;

for (x=0;x<image->ncol;x++) for (y=0;y<image->nrow;y++)
{
  mwdebug("processing pixel (%d,%d)...\n",x,y);
  .
  . (statement)
  .
}
\end{verbatim}

\newpage %......................................

\Name{mwerror}{print an error message}
\Summary{
void mwerror(type, exit\_code, format, \ldots);

int type;

int exit\_code;

char *format;

}
\Description
This function prints its arguments in \ldots under control of the format in \verb+format+ on the standard error output, in the same manner that the standard C function \verb+fprintf(stdout,format,...)+ does. 

A message is added to the print, and an action may be performed, according to the value in \verb+type+ :
\begin{itemize}
\item \verb+WARNING+ :
the additional message is \verb+MegaWave warning (+{\em mwname}\verb+) : + (following is the requested print);

\item \verb+ERROR+ :
the additional message is \verb+MegaWave error (+{\em mwname}\verb+) : +  (following is the requested print), and
the variable \verb+mwerrcnt+ is incremented.

\item \verb+FATAL+ :
the additional message is \verb+MegaWave fatal (+{\em mwname}\verb+) : +  (following is the requested print), and a call to \verb+mwexit(exit_code)+ is performed.

\item \verb+INTERNAL+ :
the additional message is \verb+MegaWave internal (+{\em mwname}\verb+) : + (following is the requested print), and a call to \verb+mwexit(exit_code)+ is performed.
Use it when such error normally never may occur. 
Then, such event points out a fault of the algorithm and the code should be fixed.
One uses to add in the beginning of the print the text \verb+[X]+ where \verb+X+ is the name of the function where the error has been found, in order to make easier the debugging process (see example below).

\item \verb+USAGE+ : 
after the requested print, is printing the usage of the module.
Use it when the input values you get in your module function does not correspond to what the usage requests.
\end{itemize}


\Next
\Example
\begin{verbatim}
/* Compute some norm of any fimage */
static float fnorm(image) 
Fimage image; 

{ float norm;  /* result of the computation */
  .
  .  (statement)
  .
  if (norm < 0.0)
   mwerror(INTERNAL,1,"[fnorm] Negative norm value computed ! (norm=%f)",norm);
  else return(norm);
}
\end{verbatim}

\newpage %......................................

\Name{mwexit}{Module termination}
\Summary{
void mwexit (status)

int status;
}
\Description
This function causes normal program termination of a MegaWave2 module.
The variable \verb+status+ indicates the status of the module when the
termination occurred; value $0$ means successful termination, other values are user-dependent.

Notice that in the MegaWave2 modules, the standard C function \verb+exit+ is
redefined to be \verb+mwexit+. 
Therefore, if you use \verb+exit+ in your code you actually call \verb+mwexit+.

\Next
\Example
\begin{verbatim}
Fimage image; /* Output of module */

/* Try several times an allocation of a fimage of size 256x256 */
while (mw_alloc_fimage(image,256,256) == NULL)
  {
     mwerror(ERROR,1,"Not enough memory !\n");
     if (mwerrcnt > 10) mwexit(-1);
     sleep(2);  /* Wait 2 seconds */
  }
\end{verbatim}

\newpage %......................................

\Name{mwfree}{Dynamic memory deallocation}
\Summary{
void mwfree (ptr)

char *ptr;
}
\Description
This function deallocates the space pointed to by 
\verb+ptr+ and which has previously been allocated by \verb+mwmalloc+. It does 
nothing if \verb+ptr+ = \Null.

You should set \verb+ptr+ to \Null after this call since the address pointed to by \verb+ptr+ is no longer valid.

Notice that in the MegaWave2 modules, the standard C function \verb+free+ is
redefined to be \verb+mwfree+. 
Therefore, if you use \verb+free+ in your code you actually call \verb+mwfree+.

\Next
\Example
\begin{verbatim}
float *data=NULL; /* Internal use: no Input neither Output of module */

/* Allocates space for 1000 samples of float values */
if  ( (data = (float *) mwmalloc (1000*sizeof(float))) == NULL )
    mwerror(FATAL,1,"Not enough memory.\n");
.
. (statement)
.
/* End of statement: deallocation of the array */
mwfree((char *) data);
\end{verbatim}

\newpage %......................................

\Name{mwmalloc}{Dynamic memory allocation}
\Summary{
void  *mwmalloc (size)

size\_t size;
}
\Description
This function allocates space for a block of at least size bytes, but does not initialize the space.

This function returns a pointer to space suitably aligned (after possible pointer coercion) for storage of any type of object. 
Do not forget to cast the return value to the right type of your variable
(see example below).
If not enough memory is available to allocate the array, the function returns \Null. 
Your code should check this return value to send an error message in the \Null\ case, and do appropriate statement.

Each space allocated by \verb+mwmalloc+ must be deallocated using \verb+mwfree+ before exiting the module.

Notice that in the MegaWave2 modules, the standard C function \verb+malloc+ is
redefined to be \verb+mwmalloc+. 
Therefore, if you use \verb+malloc+ in your code you actually call \verb+mwmalloc+.

\Next
\Example
\begin{verbatim}
long *data=NULL; /* Internal use: no Input neither Output of module */

/* Allocates space for 5000 samples of long values */
if  ( (data = (long *) mwmalloc (5000*sizeof(long))) == NULL )
    mwerror(FATAL,1,"Not enough memory.\n");

/* Set the sample #i to the value i */
for (i=0;i<5000;i++) data[i] = i;
\end{verbatim}

\newpage %......................................

\Name{mwrealloc}{Dynamic memory re-allocation}
\Summary{
void  *mwrealloc(ptr, size)

char *ptr;
unsigned size;
}
\Description
This function changes the size of the block pointed to by \verb+ptr+ to
\verb+size+ bytes and returns a pointer to the (possibly moved) block.  
Existing contents are unchanged up to the lesser of the new and old sizes.  
If \verb+ptr+ is a \Null pointer, \verb+mwrealloc+ behaves like 
\verb+mwmalloc+ for the specified size.  
If \verb+size+ is zero and \verb+ptr+ is not a \Null pointer, the object it points to is freed and \Null is returned.

If not enough memory is available to allocate the array, the function returns \Null. 
Your code should check this return value to send an error message in the \Null\ case, and do appropriate statement.

Do not forget to cast the return value to the right type of your variable, and to cast the type of the input pointer \verb+ptr+ to be \verb+char *+ (see example below).

Each space allocated by \verb+mwrealloc+ must be deallocated using \verb+mwfree+ before exiting the module.

Notice that in the MegaWave2 modules, the standard C function \verb+realloc+ is
redefined to be \verb+mwrealloc+. 
Therefore, if you use \verb+realloc+ in your code you actually call \verb+mwrealloc+.

\Next
\Example
\begin{verbatim}
long *ldata=NULL;   /* Internal use: no Input neither Output of module */
double *ddata=NULL;

/* Allocates space for 5000 samples of long values */
if  ( (ldata = (long *) mwmalloc (5000*sizeof(long))) == NULL )
    mwerror(FATAL,1,"Not enough memory.\n");
.
.
.
/* Re-allocates space for 2000 samples of double values, using space allocated
   for ldata 
*/
if  ( (ddata = (double *) mwrealloc ((char *) ldata, 2000*sizeof(double))) == NULL )
    mwerror(FATAL,1,"Not enough memory.\n");

/* Warning : do not use anymore the array ldata ! */
\end{verbatim}

\newpage %......................................



