\Name{mw\_alloc\_biortho\_wtrans2d}{Allocate the arrays of the decomposition}
\Summary{
void *mw\_alloc\_biortho\_wtrans2d(wtrans,level,nrow,ncol)

Wtrans2d wtrans;

int level; 

int nrow,ncol;   
}
\Description
This function allocates the array \verb+images+ of a \wtransdd structure previously created using \verb+mw_new_wtrans2d+, in order to receive
an biorthonormal wavelet representation (spatial decimation, \verb+norient+ $=3$).
Each image \verb+images[l][r]+ for $l=1 \ldots \mbox{{\tt nlevel}}$, $r=0 \ldots \mbox{{\tt norient}}$ is created and allocated to the right size.
Previously allocations are deleted, if any.

The number of levels for the decomposition is given by \verb+level+ and the
size of the original image is given by \verb+nrow+ (number of rows), \verb+ncol+ (number of columns).

The array \verb+images+ can be addressed after this call, if the allocation successed. There is no default values for the images.
The \verb+type+ field of the \wtransdd structure is set to \verb+mw_biorthogonal+.

The function \verb+mw_alloc_biortho_wtrans2d+ returns \Null\ if not enough memory is available to allocate one of the images. 
Your code should check this return value to send an error message in the \Null\ case, and do appropriate statement.

Notice that, if the wavelet transform is an output of a MegaWave2 module, the structure has been already created by the compiler if needed (see \volI): do not perform additional call to \verb+mw_new_wtrans2d+ (see example below).

\Next
\Example
\begin{verbatim}
Wtrans2d Output; /* optional Output of the module */
Fimage Image;    /* needed Input of the module: original image */
int J;           /* internal use */

if (Output) 
{
  /* Output requested : allocate Output for 8 levels of decomposition */
  if(mw_alloc_biortho_wtrans2d(Output, 8, Image->nrow, Image->ncol)==NULL)
    mwerror(FATAL,1,"Not enough memory.\n");
  
  Output->images[0][0] = Image;
  for (J = 1; J <= 8; J++)
  {
     .
     . (Computation of the voice #J)
     .
  }
}
\end{verbatim}

\newpage %......................................

\Name{mw\_alloc\_dyadic\_wtrans2d}{Allocate the arrays of the decomposition}
\Summary{
void *mw\_alloc\_dyadic\_wtrans2d(wtrans,level,nrow,ncol)

Wtrans2d wtrans;

int level; 

int nrow,ncol;   
}
\Description
This function allocates the array \verb+images+ of a \wtransdd structure previously created using \verb+mw_new_wtrans2d+, in order to receive
an dyadic wavelet representation (no spatial decimation, \verb+norient+ $=2$).
Each image \verb+images[l][r]+ for $l=1 \ldots \mbox{{\tt nlevel}}$, $r=0 \ldots \mbox{{\tt norient}}$ is created and allocated to the right size.
Previously allocations are deleted, if any.

The number of levels for the decomposition is given by \verb+level+ and the
size of the original image is given by \verb+nrow+ (number of rows), \verb+ncol+ (number of columns).

The array \verb+images+ can be addressed after this call, if the allocation successed. There is no default values for the images.
The \verb+type+ field of the \wtransdd structure is set to \verb+mw_dyadic+.

The function \verb+mw_alloc_dyadic_wtrans2d+ returns \Null\ if not enough memory is available to allocate one of the images. 
Your code should check this return value to send an error message in the \Null\ case, and do appropriate statement.

Notice that, if the wavelet transform is an output of a MegaWave2 module, the structure has been already created by the compiler if needed (see \volI): do not perform additional call to \verb+mw_new_wtrans2d+ (see example below).

\Next
\Example
\begin{verbatim}
Wtrans2d Output; /* optional Output of the module */
Fimage Image;    /* needed Input of the module: original image */
int J;           /* internal use */

if (Output) 
{
  /* Output requested : allocate Output for 8 levels of decomposition */
  if(mw_alloc_dyadic_wtrans2d(Output, 8, Image->nrow, Image->ncol)==NULL)
    mwerror(FATAL,1,"Not enough memory.\n");
  
  Output->images[0][0] = Image;
  for (J = 1; J <= 8; J++)
  {
     .
     . (Computation of the voice #J)
     .
  }
}
\end{verbatim}

\newpage %......................................

\Name{mw\_alloc\_ortho\_wtrans2d}{Allocate the arrays of the decomposition}
\Summary{
void *mw\_alloc\_ortho\_wtrans2d(wtrans,level,nrow,ncol)

Wtrans2d wtrans;

int level; 

int nrow,ncol;   
}
\Description
This function allocates the array \verb+images+ of a \wtransdd structure previously created using \verb+mw_new_wtrans2d+, in order to receive
an orthonormal wavelet representation (spatial decimation, \verb+norient+ $=3$).
Each image \verb+images[l][r]+ for $l=1 \ldots \mbox{{\tt nlevel}}$, $r=0 \ldots \mbox{{\tt norient}}$ is created and allocated to the right size.
Previously allocations are deleted, if any.

The number of levels for the decomposition is given by \verb+level+ and the
size of the original image is given by \verb+nrow+ (number of rows), \verb+ncol+ (number of columns).

The array \verb+images+ can be addressed after this call, if the allocation successed. There is no default values for the images.
The \verb+type+ field of the \wtransdd structure is set to \verb+mw_orthogonal+.

The function \verb+mw_alloc_ortho_wtrans2d+ returns \Null\ if not enough memory is available to allocate one of the images. 
Your code should check this return value to send an error message in the \Null\ case, and do appropriate statement.

Notice that, if the wavelet transform is an output of a MegaWave2 module, the structure has been already created by the compiler if needed (see \volI): do not perform additional call to \verb+mw_new_wtrans2d+ (see example below).

\Next
\Example
\begin{verbatim}
Wtrans2d Output; /* optional Output of the module */
Fimage Image;    /* needed Input of the module: original image */
int J;           /* internal use */

if (Output) 
{
  /* Output requested : allocate Output for 8 levels of decomposition */
  if(mw_alloc_ortho_wtrans2d(Output, 8, Image->nrow, Image->ncol)==NULL)
    mwerror(FATAL,1,"Not enough memory.\n");
  
  Output->images[0][0] = Image;
  for (J = 1; J <= 8; J++)
  {
     .
     . (Computation of the voice #J)
     .
  }
}
\end{verbatim}

\newpage %......................................

\Name{mw\_delete\_wtrans2d}{Deallocate the wavelet transform space}
\Summary{
void mw\_delete\_wtrans2d(wtrans)

Wtrans2d wtrans;
}
\Description
This function deallocates the memory used by the wavelet transform space \verb+wtrans+ that is, all the memory used by the array of images \verb+images+ (if any), and the structure itself. 

You should set \verb+wtrans = NULL+ after this call since the address pointed
by \verb+wtrans+ is no longer valid.

\Next
\Example
\begin{verbatim}
Wtrans2d wtrans=NULL; /* Internal use: no Input neither Output of module */

if  ( ((wtrans = mw_new_wtrans2d()) == NULL) ||  
       (mw_alloc_ortho_wtrans2d(wtrans, 6, 512, 512)==NULL) )
    mwerror(FATAL,1,"Not enough memory.\n");
.
.
.
mw_delete_wtrans2d(wtrans);
wtrans = NULL;
\end{verbatim}

\newpage %......................................


\Name{mw\_new\_wtrans2d}{Create a new Wtrans2d}
\Summary{
Wtrans2d mw\_new\_wtrans2d();
}
\Description
This function creates a new \wtransdd structure with empty array of images \verb+images+.
No image can be addressed at this time.
The array of images should  be allocated using one of the functions \verb+mw_alloc_X_wtrans2d+ where \verb+X+ depends of the type of the transformation.

You don't need this function for input/output of modules, since the MegaWave2
Compiler already created the structure for you if you need it (see \volI). 
This function is used to create internal variables.
Do not forget to deallocate the internal structures before the end
of the module.

The function \verb+mw_new_wtrans2d+ returns \Null\ if not enough memory is available to create the structure. Your code should check this value to send an
error message in the \Null\ case, and do appropriate statement.

\Next
\Example
\begin{verbatim}
Wtrans2d wtrans=NULL; /* Internal use: no Input neither Output of module */

if  ( ((wtrans = mw_new_wtrans2d()) == NULL) ||  
       (mw_alloc_dyadic_wtrans2d(wtrans, 6, 512, 512)) )
    mwerror(FATAL,1,"Not enough memory.\n");
\end{verbatim}

\newpage %......................................

