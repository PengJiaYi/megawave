%
% Part 10 of the MegaWave2 Guide #2
%   Wdevice and window facilities
%

The Wdevice library\index{Wdevice library} provides an interface to the window manager\index{window manager}: 
it helps the user to write modules which have to access to the window manager resources, as the screen, the mouse, etc.
It not only replaces some painful operations which require a lot of code (such opening a window, mapping the content of an image into a window, etc.) to a simple call to one function, but it provides also an interface which is independant to the type of the window system: the calls to the Wdevice functions remain the same even if the window system changes (and the result should be the same).

This library is independant to the MegaWave2 System Library although some modules cannot be linked
without it : it is added when needed during the link process of a MegaWave2 command.
Of course, one Wdevice library per window system is needed. 
At this time, there exists a Wdevice library for the X Window System\index{X Window System} Version 11
(X11) only.
In the past, one could find a Wdevice library for the Suntools\index{Suntools} System but, because of the 
renunciation of Suntools from Sun MicroSystem, this library is no longer maintained 
(and no longer distributed).

On can found in the system library some packages that use functions defined by Wdevice to perform
more high-levels tasks, such as Wpanel :
The Wpanel\index{Wpanel} (Panel\index{panel|see{Wpanel}} display facilities) is a small package that allows to 
handle buttons and bars. It is not documented yet, and will probably change quite much in the future. For the time 
being, it is used only in the module \verb+llview.c+.

%+++++++++++++++++++++++++++++++++++++++++++++++
\section{Functions Summary}
%+++++++++++++++++++++++++++++++++++++++++++++++
\label{wdevice_function}

The following is a description of the Wdevice library functions 
which may be called by the user. 
The list is in alphabetical order.

You may notice that each function name begins with the letter \verb+W+.

% @@@@@@@@@@ a retirer apres completion de la doc @@@@@@@@@
Warning: the functions summary is not documented yet. 
If you need to access to the screen into a MegaWave2 module (e.g. to draw some figure, etc.) please read the code of the following public MegaWave2 modules, and take inspiration from those:
\begin{itemize}
\item \verb+view_demo.c+;
\item \verb+cview.c+;
\item \verb+ccview.c+;
\item \verb+cmview.c+;
\item \verb+splot.c+;
\item \verb+readpoly.c+.
\end{itemize}
Nevertheless, and because those MegaWave2 modules already exist, you are not likely to really need to learn about the Wdevice library.
% @@@@@@@@@@ a retirer apres completion de la doc @@@@@@@@@


\newpage %......................................

%\input{chapter_10_wdevice.tex}
