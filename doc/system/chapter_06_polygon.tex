\Name{mw\_alloc\_polygon}{Allocate the channels array}
\Summary{
Polygon mw\_alloc\_polygon(polygon,nc)

Polygon polygon;

int nc;
}
\Description
This function allocates the channels array of a \polygon structure previously
created using \verb+mw_new_polygon+. The size of the array is given by 
\verb+nc+, it is the number of different channels.
A channel corresponds to a real parameter associated to the polygon. 
The meaning of such channel has to be defined by the user. 
For example, \verb+polygon->channel[0]+ may be the gray level of the polygon.

Do not use this function if \verb+polygon+ has already an allocated channels array: use the function \verb+mw_change_polygon+ instead.

The function \verb+mw_alloc_polygon+ returns \Null\ if not enough memory is available to allocate the structure or the channels array. 
Your code should check this return value to send an error message in the \Null\ case, and do appropriate statement.

\Next
\Example
See the example of the function \verb+mw_new_polygon+.

\newpage %......................................

\Name{mw\_change\_polygon}{Change the number of channels}
\Summary{
Polygon mw\_change\_polygon(polygon,nc)

Polygon polygon;

int nc;
}

\Description
This function changes the memory allocation for the channels array of
a \polygon structure, even if no previously memory allocation was done. 

The number of channels is given by \verb+nc+; a channel corresponds to
a real parameter associated to the polygon. The meaning of such channel
has to be defined by the user. For example, \verb+polygon->channel[0]+ may be the gray level of the polygon.

This function can also create the structure if the input \verb+polygon = NULL+.
Therefore, this function can replace both \verb+mw_new_polygon+ and
\verb+mw_alloc_polygon+. 
It is the recommended function to set the number of channels for polygons which are input/output of a module.
Since the function can set the address of \verb+polygon+, the variable must be set to the return value of the function (See example below).

The function \verb+mw_change_polygon+ returns \Null\ if not enough memory is available to allocate the structure or the channels array. 
Your code should check this return value to send an error message in the \Null\ case, and do appropriate statement.

\Next
\Example
\begin{verbatim}
Polygon polygon; /* Input of module */

polygon = mw_change_polygon(polygon,1);
if (polygon == NULL) mwerror(FATAL,1,"Not enough memory.\n");
polygon->channel[0] = 255.0;
...
\end{verbatim}
(End of this example as for the \verb+mw_new_polygon+ function).

\newpage %......................................


\Name{mw\_delete\_polygon}{Deallocate a polygon}
\Summary{
void mw\_delete\_polygon(polygon)

Polygon polygon;
}
\Description
This function deallocates all the memory allocated by the polygon variable that is, all the points belonging to this chain, the channels array (if needed) and the \polygon structure itself.
You should set \verb+polygon = NULL+ after this call since the address pointed
by \verb+polygon+ is no longer valid.

\Next
\Example
\begin{verbatim}
/* Remove the first polygon of an existing polygon set (polygons) */

Polygons polygons;/* Existing polygons set (e.g. Input of module) */
Polygon polygon;  /* Internal use */

polygon = polygons->first;
polygons->first=polygons->next;
polygons->next->previous = NULL;
mw_delete_polygon(polygon);
polygon = NULL;
\end{verbatim}

\newpage %......................................


\Name{mw\_length\_polygon}{Return the number of points of a polygon}
\Summary{
unsigned int mw\_length\_polygon(poly);

Polygon poly;
}
\Description
This function return the number of points contained in the given
polygon \verb+poly+. It returns $0$ if the structure is empty.

\Next
\Example
\begin{verbatim}

Polygon polygon; /* Internal use: no Input neither Output of module */
point_curve newp,oldp=NULL;
int i;

polygon = mw_new_polygon();
if (polygon == NULL) mwerror(FATAL,1,"Not enough memory.\n");

/* Define a polygon with 5 points */
for (i=1;i<=5;i++)
{
 newp = mw_new_point_curve();
 if (newp == NULL) mwerror(FATAL,1,"Not enough memory.\n");
 if (i=0) polygon->first = newp;
 newp->x = newp->y = i;
 newp->previous = oldp;
 if (oldp) oldp->next = newp;
 oldp=newp;
} 

/* The length is 5 */
printf("Length=%d\n",mw_length_polygon(polygon));

\end{verbatim}
\newpage %......................................

\Name{mw\_new\_polygon}{Create a new polygon}
\Summary{
Polygon mw\_new\_polygon();
}
\Description
This function creates a new \polygon structure with an empty channels array.
It returns \Null\ if not enough memory is available to create the structure.
Your code should check this value to send an
error message in the \Null\ case, and do appropriate statement.

Since the MegaWave2 compiler allocates structures for input and output 
objects (see \volI), this function is normally used only for internal objects.
Do not forget to deallocate the internal structures before the end
of the module, except if they are part of an input or output polygons set.

\Next
\Example
\begin{verbatim}
/* Define a polygon with 10 points which is the straight line (0,0)-(9,9) */

Polygon polygon; /* Internal use: no Input neither Output of module */
Point_curve newp,oldp=NULL;
int i;

polygon = mw_new_polygon();
if ((polygon == NULL) || (mw_alloc_polygon(polygon,1) == NULL))
    mwerror(FATAL,1,"Not enough memory.\n");
polygon->channel[0] = 255.0;

for (i=0;i<10;i++)
{
 newp = mw_new_point_curve();
 if (newp == NULL) mwerror(FATAL,1,"Not enough memory.\n");
 if (i=0) polygon->first = newp;
 newp->x = newp->y = i;
 newp->previous = oldp;
 if (oldp) oldp->next = newp;
 oldp=newp;
} 
\end{verbatim}
\newpage %......................................

