\Name{mw\_change\_fmovie}{Define the movie structure, if not defined}
\Summary{
Fmovie mw\_change\_fmovie(movie)

Fmovie movie;
}
\Description
This function returns a movie structure if the input \verb+movie = NULL+.
It is provided despite the \verb+mw_new_fmovie+ function for
global coherence with other memory types.

The function \verb+mw_change_fmovie+ returns \Null\ if not enough memory is available to define the structure. Your code should check this return value to 
send an error message in the \Null\ case, and do appropriate statement.

Since the MegaWave2 compiler allocates structures for input and output 
objects (see \volI), this function is normally used only for internal objects.
Do not forget to deallocate the internal structures before the end
of the module.

Images have to be allocated using the appropriate functions (See the example
below).

\Next
\Example
\begin{verbatim}

Fmovie movie=NULL; /* Internal use: no Input neither Output of module */

movie = mw_change_fmovie(movie);
if (movie == NULL) mwerror(FATAL,1,"Not enough memory.\n");
...
\end{verbatim}
(End of this example as for the \verb+mw_new_fmovie+ function).
\newpage %......................................


\Name{mw\_delete\_fmovie}{Deallocate all the movie}
\Summary{
void mw\_delete\_fmovie(movie)

Fmovie movie;
}
\Description
This function deallocates all the memory used by a \fmovie structure:
it deallocates the gray plane of all images, the image structures and
the movie structure itself.

You should set \verb+movie = NULL+ after this call since the address pointed
by \verb+movie+ is no longer valid.

\Next
\Example
See the example of the \verb+mw_new_fmovie+ function: when a memory allocation
fails for \verb+mw_change_fimage+, all the previously memory allocations are
freed by the call to \verb+mw_delete_fmovie(movie)+.

\newpage %......................................


\Name{mw\_new\_fmovie}{Create a new Fmovie}
\Summary{
Fmovie mw\_new\_fmovie();
}
\Description
This function creates a new \fmovie structure.
It returns \Null\ if not enough memory is available to create the structure.
Your code should check this value to send an
error message in the \Null\ case, and do appropriate statement.

Since the MegaWave2 compiler allocates structures for input and output 
objects (see \volI), this function is normally used only for internal objects.
Do not forget to deallocate the internal structures before the end
of the module.

Images have to be allocated using the appropriate functions (See the example
below).

\Next
\Example
\begin{verbatim}
/* Create a movie with 10 images of size (100,100) */

Fmovie movie;  /* Internal use: no Input neither Output of module */
Fimage image,image_prev; /* Internal use */

movie = mw_new_fmovie();
if (movie == NULL) mwerror(FATAL,1,"Not enough memory.\n");
for (l=1;l<=10;l++)
 {
   if ((image = mw_change_fimage(NULL,100,100)) == NULL)
    {
      mw_delete_fmovie(movie);
      mwerror(FATAL,1,"Not enough memory.");
    }
    if (l == 1) movie->first = image;
    else
      {
        image_prev->next = image;
        image->previous = image_prev;
       }
     image_prev = image;
 }
\end{verbatim}
\newpage %......................................

