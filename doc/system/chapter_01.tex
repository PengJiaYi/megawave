%
% Part 1 of the MegaWave2 Guide #2
%   Introduction
%

%-----------------------------------------------
\section{What you will find in this guide}
%-----------------------------------------------

\label{intro_guide}

When you implement an algorithm in MegaWave2, you write a code in C language 
in what we call a module (see \volI). Your algorithm processes some objects
which represent your data. So you need to know how to create an object\index{object} of
the type you want, how to access to it, how to remove it, etc.

This present guide will detail all the available MegaWave2 objects and most
related functions\index{function} which are part of the System Library 
(Sections~\ref{images} to~\ref{rawdata}).
In addition, you will find the description of other functions which may
be called by the user in the module - such as error handling functions -
(Section~\ref{miscellaneous}).
There is also a description of the Wdevice Library, a toolbox for the
window interface (Section~\ref{wdevice}).

This guide is a reference manual : it would be boring to read it from the
beginning to the end. If you are new with MegaWave2, you should entirely read this
introduction were basic principles are explained, and all introductions of 
the next main sections, to get an idea about the various objects you may use.
Afterward, when you will be searching for a particular structure or function, consult 
the contents page~\pageref{contents} or the index page~\pageref{index}.


%-----------------------------------------------
\section{The MegaWave2 memory (internal) types}
%-----------------------------------------------

\label{intro_memory-types}

\index{memory type |see{structure}}
\index{internal type |see{structure}}
\index{C type |see{structure}}

\index{structure}

MegaWave2 objects such as images, movies, signals, curves, \ldots, are 
represented in the module code as {\em pointers to a structure}.
The type of the structure defines the object you want to process, as
\verb+struct fimage+ for an image of Floating points values (the pointer of
this structure is of type \verb+Fimage+).

Each structure has particular fields, as \verb+gray+ for a \verb+Fimage+ which
represents the gray levels plane. They are described in the section presenting
the structure (Section~\ref{images_float-images_structure} page~\pageref{images_float-images_structure} for \verb+Fimage+).

Some fields are common to most structures, they are:
\begin{itemize}
\item \verb+cmt+ : string of maximum size \verb+mw_cmtsize+ where to put the comment associated to the object. 
For input objects and at the beginning of the
module statement, this field contains the comment field of the corresponding file object (if the file type provides a comment field). 
For output objects and at the end of the module statement, this field contains the name of the module plus the comments of the input objects, if any.
This default output value can be overwritten by setting a value to \verb+cmt+.
\item \verb+name+ : string of maximum size \verb+mw_namesize+ where to put the name of this object. For input objects, this field contains the file name of the corresponding file object. 
The default output value is ``?''. 
It can be overwritten.
\end{itemize}

You can of course access to any field in order to read its content. But be carefull
when you want to overwrite the content of a field: some fields have to be
updated by the system library only (e.g. the dimension fields \verb+nrow+ and
\verb+ncol+ of image objects).

Some structures may contain undocumented fields: they are used internally by 
the system library and users should not access to them, especially for writing.

Some conversions between memory types are available as functions of the System Library,
see Section~\ref{conv_memory_types} of this guide for a list of the most current conversion 
functions.


%-------------------------------------------------
\section{File (external) types or file formats}
%-------------------------------------------------

\label{intro_file-types}

\subsection{Generalities}

\index{external type|see{file format}}
\index{file type|see{file format}}
\index{file format}

When a module's command finishes, the output objects (of memory types) have to be saved on disk for future use.
For example, they can be the input of another module's command.
Data may be saved on disk also (or read from disk) when the module is run
into an interpreter such as XMegaWave2, although in this case modules communicate with memory type structures.

This shows that external type objects are needed; they are files written in a predefined format.
MegaWave2 can use some well-known formats available in the public domain, especially to carry the different image memory types.
When no satisfying standard is available to match a given memory type, a specific format is used.
Notice that, whereas there is only one memory type associated to an object, an object of a given memory type may be represented on disk with various file types.

Conversions between some formats are available: you may load an object written
in a file type which is different from the regular one used for the memory type of your object. Depending on the case, you may however lose precision in your data
(in that case, a warning message is send).
For output ojects, MegaWave2 chooses a default file type to write the data.
You can modify this choice using the system option \verb+-ftype+ (see \volI).

A short description of the file types is given in the next sections about the different memory types.

\subsection{Search path convention}

\index{search path}
When a module is called in the command line mode, MegaWave2 searches the file names of the input objects in different directories, following the order:
\begin{enumerate}
\item the current directory of the shell, i.e. ``\verb+.+'';
\item the module's group directory of \verb+$MY_MEGAWAVE2/data+;
\item \verb+$MY_MEGAWAVE2/data+ and its subdirectories;
\item the module's group directory of \verb+$MEGAWAVE2/data+;
\item \verb+$MEGAWAVE2/data+ and its subdirectories.
\end{enumerate}
Notice that this search path convention has changed from MegaWave2 versions 1.x to versions 2.x and
from versions 2.x to versions 3.x.

The output objects are always written in the current directory of the shell.
{\em Beware :
 if you give the same name as the one of an existing file, the content of the previous file will be overwritten 
(there is no confirmation message).}




