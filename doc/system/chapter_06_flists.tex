\Name{mw\_change\_flists}{Define and allocate a \flists\ structure}
\Summary{
Flists mw\_change\_flists(ls,max\_size,size)

Flist ls;

int max\_size,size;
}
\Description
This function changes the memory allocation of the \verb+list+ array
of a \flists\ structure, even if no previously memory allocation was done.
The new size (number of lists) of the structure is given by \verb+size+,
and the size to allocate (maximal number of lists) by \verb+max_size+.

It can also create the structure if the input \verb+ls = NULL+.
Therefore, this function can replace both \verb+mw_new_flists+ and
\verb+mw_realloc_flists+. 
Since the function can set the address of \verb+ls+, the variable must be set 
to the return value of the function (See example below).

The function \verb+mw_change_flists+ returns \Null\ if not enough memory is 
available to allocate the structure or the \verb+list+ array, and an error 
message is issued. 
Your code should check this return value to eventually send a fatal error 
message in the \Null\ case, and do appropriate statement.

\Next
\Example
\begin{verbatim}

Flists ls;

/* 
   Allocate ls to handle at most 10 lists, the current number of
   lists being 0 (no list).
*/
ls = mw_change_flists(NULL,10,0);
if (!ls) mwerror(FATAL,1,"Not enough memory to continue !\n");


\end{verbatim}

\newpage %......................................


\Name{mw\_copy\_flists}{Copy the lists contained in a \flists\ structure}
\Summary{
Flists mw\_copy\_flists(in,out)

Flists in,out;
}
\Description
This function copies the \verb+list+ array and \verb+data+ field 
of the \flists\ structure \verb+in+ into \verb+out+ : 
each list contained in \verb+in+ are duplicated. 
The duplicated \flists\ \verb+out+ is allocated to at least
the current size of \verb+in+. 

Since the function can set the address of \verb+out+, the variable must be set 
to the return value of the function (See example below).

The function \verb+mw_copy_flists+ returns \Null\ if not enough memory is 
available to allocate the structure or the \verb+list+ array, and an error 
message is issued. 
Your code should check this return value to eventually send a fatal error 
message in the \Null\ case, and do appropriate statement.

\Next
\Example
\begin{verbatim}

Flists in,out=NULL;

/* 
   Allocate ls to handle at most 10 lists, the current number of
   lists being 3.
*/
ls = mw_change_flists(NULL,10,3);
if (!ls) mwerror(FATAL,1,"Not enough memory to continue !\n");


/* ... (Here fill the lists) ... */

/*
 Copy in into out. Allocated size for out is 3 lists.
*/
out=mw_copy_flists(in,out);
if (!out) mwerror(FATAL,1,"Not enough memory to copy the lists !\n");


\end{verbatim}

\newpage %......................................


\Name{mw\_delete\_flists}{Delete the lists and the \flists\ structure}
\Summary{
void mw\_delete\_flists(ls)

Flist ls;
}
\Description
This function deletes the lists contained in the \verb+list+ array,
and the structure \flists\ itself.
Warning : the memory of the user-defined field \verb+data+ is not freed.
If this field has been allocated, you should free it before calling
\verb+mw_delete_flists+.

\Next
\Example
\begin{verbatim}

Flist ls;
int i;

/*
    ... (Assume ls has been previoulsy allocated)...
*/

/* 
  Free the lists, including data field. 
*/
for (i=ls->size;i--;) if (ls->list[i]->data) free(ls->list[i]->data);
if (ls->data) free(ls->data);
mw_delete_flists(ls);


\end{verbatim}

\newpage %......................................
\Name{mw\_enlarge\_flists}{Enlarge the number of lists a \flists\ may contain}
\Summary{
Flists mw\_enlarge\_flist(ls)

Flist ls;
}
\Description
This function performs a memory reallocation on the array
\verb+ls->list+ to increase the number of lists that can be recorded.
The enlargement factor is fixed by the constant \verb+MW_LIST_ENLARGE_FACTOR+
defined in the include file \verb+list.h+.
This function is useful when one does not know by advance the number
of lists, and when one wish to avoid multiple reallocations.

If not enough memory is available to perform the reallocation, an error
message is issued and the function returns \verb+NULL+.
Otherwise, the function returns \verb+ls+.


\Next
\Example
\begin{verbatim}


/* Fill a flists with lists until the user enters 'Q'.
*/

Flist ls; 
Flist l;
char c;

ls = mw_change_flists(NULL,10,0);
if (ls==NULL) mwerror(FATAL,1,"Not enough memory to continue !\n");
do {
  if (ls->size == ls->max_size) 
    if (mw_enlarge_flists(ls)==NULL)
      mwerror(FATAL,1,"Not enough memory to continue !\n");
  l = mw_change_flist(NULL,10,10,2);
  if (l==NULL) mwerror(FATAL,1,"Not enough memory to continue !\n");
  mw_clear_flist(l,1.0)
  ls->list[ls->size++] = l;
  scanf("%c",&c);
    } while (c!='Q');
\end{verbatim}

\newpage %......................................


\Name{mw\_new\_flists}{Create a \flists\ structure}
\Summary{
Flists mw\_new\_flists()

}
\Description
This function creates a new \flists\ structure. The fields are initialized
to $0$ or \Null value.
The function returns the address of the new structure, or
\Null\ if not enough memory is available.

\Next
\Example
\begin{verbatim}

Flists ls;

/*
  Define the structure
*/
ls = mw_new_flists();
if (!ls) mwerror(FATAL,1,"Not enough memory to define the lists !\n");

/* 
  At that time, the FLists is empty (no lists).
*/
\end{verbatim}

\newpage %......................................


\Name{mw\_realloc\_flists}{Realloc the list array of the \flists}
\Summary{
Flists mw\_realloc\_flists(ls,n)

Flists ls;

int n;
}
\Description
This function performs a memory reallocation on the array
\verb+ls->list+ so that at most $n$ lists can be recorded.

If not enough memory is available to perform the reallocation, an error
message is issued and the function returns \Null.
Otherwise, the function returns \verb+ls+.

\Next
\Example
\begin{verbatim}
Flists ls;

/*
   Allocate ls to handle 10 lists.
*/

ls = mw_new_flists();
if (!ls) mwerror(FATAL,1,"Not enough memory to continue !\n");
ls = mw_realloc_flists(ls,10);
if (!ls) mwerror(FATAL,1,"Not enough memory to continue !\n");

\end{verbatim}

\newpage %......................................


