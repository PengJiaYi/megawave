%
% Part 4 of the MegaWave2 Guide #2
%   Signals
%

We call signal\index{signal} a one-dimensional sequence of scalars. 
Signals may be used to represent various kind of physical data (such
as sound) , as well as mathematical data (e.g. impulse response of filters, 
vectors, \ldots).

Notice that at this time, only signals of floating points values are
implemented.

%-----------------------------------------------
\section{Float signals}
%-----------------------------------------------


\label{signals_float-signals}
\def\fsignal{{\tt Fsignal}}
\index{structure!\fsignal}

The {\em Float Signals} memory type is used to represent one-dimensional 
sequences of floating points values.

%+++++++++++++++++++++++++++++++++++++++++++++++
\subsection{The structure Fsignal}
%+++++++++++++++++++++++++++++++++++++++++++++++
\label{signals_float-signals_structure}


 Beginners should only focus on the first two fields of this structure:
{\small
\begin{verbatim}
typedef struct fsignal {
  int size;        /* Number of samples */
  float *values;   /* The samples */

  float scale;     /* Scale of the signal */
  float shift;     /* shifting of the signal with respect to zero */
  float gain;      /* Gain of the signal given by the digitalization process */
  float sgrate;    /* Sampling rate given by the digitalization process */
  int bpsample;    /* Number of bits per sample for audio drivers */

  char cmt[mw_cmtsize]; /* Comments */
  char name[mw_namesize]; /* Name of the image */
  
  /* Defines the signifiant part of the signal : */
  int firstp;     /* index of the first point not aff. by left side effect */
  int lastp;      /* index of the last point not aff. by right side effect */
  float param;    /* distance between two succesive uncorrelated points */

} *Fsignal;
\end{verbatim}
}

The field \verb+size+ gives the number of samples loaded in the signal.
Do not change by yourself the content of this field: the size of the signal
has to be modified using functions of the library only 
(see section~\ref{signals_float-signals_function} page~\pageref{signals_float-signals_function}).

The field \verb+values+ is an array which gives the value of each sample:
if \verb+signal+ is a variable of \fsignal\ type, \verb+signal->values[0]+
is the first sample of the signal, \verb+signal->values[1]+ the second,
and so one up to the last sample  \verb+signal->values[signal->size-1]+.

%+++++++++++++++++++++++++++++++++++++++++++++++
\subsection{Related file (external) types}
%+++++++++++++++++++++++++++++++++++++++++++++++
\label{signals_float-file_type}

The list of the available formats is the following:
\begin{enumerate}
\item \verb+"A_FSIGNAL"+\index{file format!A\_FSIGNAL} 
MegaWave2 Data Ascii format with a \verb+def fsignal+ area.
This area includes the value of the different fields of the object, as
\verb+comments, scale, shift, ...+ and at the end the samples of the signal.
Since this format uses Ascii encoding, you may read or modify the file just by
editing it using a text editor.
It can also be plotted using the standard tool {\tt gnuplot}.

\item \verb+"WAVE_PCM"+\index{file format!WAVE\_PCM}  
Microsoft's RIFF WAVE sound file format with PCM encoding.
Use this format to perform sound\index{sound processing} and 
speech\index{speech processing} processing with MegaWave2.
Stereo inputs are converted to mono when loaded into a \fsignal.
Since this format performs bit-encoding, on any output \fsignal\ variables
you should set the field \verb+bpsample+ to the number of bits you want the
data to be saved. Default value is $8 \times $\verb+sizeof(float)+ (on most
architectures $32$), because this matches the size of the samples in the
\fsignal\ structure. 
However, this value leads to strange results on some audio drivers. 
If you plan to send the
signal on a audio driver\index{audio}, 
recommended numbers of bits are $16$ (signed word) 
or $8$ (signed char). Take care to format your data to fit the corresponding 
range before playing the file ($[-32768,+32767]$ for signed word and 
$[-128,+127]$ for signed char) or you will not get the expected sound.
Another important field to get the right result is \verb+sgrate+, where you
have to set the sample rate in Hz that is, the number of samples per second.

\end{enumerate}

%+++++++++++++++++++++++++++++++++++++++++++++++
\subsection{Functions Summary}
%+++++++++++++++++++++++++++++++++++++++++++++++
\label{signals_float-signals_function}

The following is a description of all the functions related to 
the \fsignal\ type. The list is in alphabetical order.

\newpage %......................................

\Name{mw\_alloc\_fsignal}{Allocate the array of values}
\Summary{
Fsignal mw\_alloc\_fsignal(signal,n)

Fsignal signal;

int n;
}
\Description
This function allocates the array \verb+values+ of a \fsignal\ structure previously created using \verb+mw_new_fsignal+. The size of the signal is given by 
\verb+n+, it corresponds to the number of samples.

Values can be addressed after this call, if the allocation successed. There is
no default values.

Do not use this function if \verb+signal+ has already an allocated array: use
the function \verb+mw_change_fsignal+ instead.

The function \verb+mw_alloc_fsignal+ returns \Null\ if not enough memory is available to allocate the array. Your code should check this return value to 
send an error message in the \Null\ case, and do appropriate statement.

\Next
\Example
\begin{verbatim}
Fsignal signal=NULL; /* Internal use: no Input neither Output of module */
int i;

/* Create a signal with 1000 samples */
if  ( ((signal = mw_new_fsignal()) == NULL) ||
      (mw_alloc_fsignal(signal,1000) == NULL) )
    mwerror(FATAL,1,"Not enough memory.\n");

/* Set the sample #i to the value i */
for (i=0;i<signal->size;i++) signal->values[i] = i;
\end{verbatim}

\newpage %......................................

\Name{mw\_change\_fsignal}{Change the size of the array of values}
\Summary{
Fsignal mw\_change\_fsignal(signal, n)

Fsignal signal;

int n;
}
\Description
This function changes the memory allocation of the array \verb+values+ of a \fsignal\ structure, even if no previously memory allocation was done.
The new size of the signal is given by \verb+n+, it corresponds to the number of samples.

The function \verb+mw_change_fsignal+ can also create the structure if the input \verb+signal = NULL+.
Therefore, this function can replace both \verb+mw_new_fsignal+ and
\verb+mw_alloc_fsignal+. It is the recommended function to set signal 
size of input/output modules. Since the function can set the 
address of \verb+signal+, the variable must be set to the return value of 
the function (See example below).

The function \verb+mw_change_fsignal+ returns \Null\ if not enough memory is available to allocate the array. 
Your code should check this return value to 
send an error message in the \Null\ case, and do appropriate statement.

\Next
\Example
\begin{verbatim}
Fsignal u; 

/* Usage when <u> IS NOT an output of the module (and has not been
   previously allocated): the function returns a new structure's address
*/
u = mw_change_fsignal(NULL, 1000);

/* Usage when <u> IS an output of the module (or has been previously
   allocated): DO NOT change the structure's address
*/
u = mw_change_fsignal(u, 1000);

if (u == NULL) mwerror(FATAL,1,"Not enough memory.\n");

\end{verbatim}

\newpage %......................................

\Name{mw\_clear\_fsignal}{Clear all values}
\Summary{
void mw\_clear\_fsignal(signal, v)

Fsignal signal;

float v;
}
\Description
This function fills the fsignal \verb+signal+ with the value given by $v$:
all samples will have the value $v$. 

\Next
\Example
\begin{verbatim}
Fsignal signal; /* Output of module */

signal = mw_change_fsignal(signal, 1000);
if (signal == NULL) mwerror(FATAL,1,"Not enough memory.\n");
/* Set all samples to 0.0 */
mw_clear_fsignal(signal,0.0);
\end{verbatim}

\newpage %......................................

\Name{mw\_copy\_fsignal\_header}{Copy the header of a signal into another one}
\Summary{
void mw\_copy\_fsignal\_header(in, out)

Fsignal in,out;
}
\Description
This function copies the header of the signal \verb+in+ into the header of the signal \verb+out+, but the size-related fields. The size of the two signals may be different
(this includes the case where size of \verb+out+ is null).

\Next
\Example
\begin{verbatim}

Fsignal G; /* Needed Input */
Fsignal F; /* Optional Output */

  if (F) 
    {
    printf("F option is active: copy header of G into F\n");
    mw_copy_fsignal_header(G, F);
     }
  else  printf("F option is not active\n");
\end{verbatim}

\newpage %......................................

\Name{mw\_copy\_fsignal}{Copy a signal into another one}
\Summary{
void mw\_copy\_fsignal(in, out)

Fsignal in,out;
}
\Description
This function copies the header and the content of the array \verb+values+ of the signal \verb+in+ into the corresponding fields of the signal \verb+out+.
The size of the two signals must be the same (this implies the \verb+out+ signal to
be allocated).

The speed of this function depends to the C library implementation, but it is
usually very fast (trying to do faster is a waste of time).

\Next
\Example
\begin{verbatim}

Fsignal G; /* Needed Input */
Fsignal F; /* Optional Output */

  if (F) {
    printf("F option is active: copy G in F\n");
    if ((F = mw_change_fsignal(F, G->size)) == NULL)
        mwerror(FATAL,1,"Not enough memory.\n");
    else mw_copy_fsignal(G, F);
   }
  else  printf("F option is not active\n");
\end{verbatim}

\newpage %......................................

\Name{mw\_copy\_fsignal\_values}{Copy the values of a signal into another one}
\Summary{
void mw\_copy\_fsignal\_values(in, out)

Fsignal in,out;
}
\Description
This function copies the content of the array \verb+values+ of the signal \verb+in+ into the array \verb+values+ of the signal \verb+out+.
The size of the two signals must be the same (this implies the \verb+out+ signal to
be allocated).

The speed of this function depends to the C library implementation, but it is
usually very fast (trying to do faster is a waste of time).

\Next
\Example
\begin{verbatim}

Fsignal G; /* Needed Input */
Fsignal F; /* Optional Output */

  if (F) {
    printf("F option is active: copy G in F\n");
    if ((F = mw_change_fsignal(F, G->size)) == NULL)
        mwerror(FATAL,1,"Not enough memory.\n");
    else mw_copy_fsignal_values(G, F);
   }
  else  printf("F option is not active\n");
\end{verbatim}

\newpage %......................................

\Name{mw\_delete\_fsignal}{Deallocate the signal}
\Summary{
void mw\_delete\_fsignal(signal)

Fsignal signal;
}
\Description
This function deallocates the array \verb+values+ of a \fsignal\ structure previously allocated using \verb+mw_alloc_fsignal+ or \verb+mw_change_fsignal+, and the structure itself. 

You should set \verb+signal = NULL+ after this call since the address pointed
by \verb+signal+ is no longer valid.

\Next
\Example
\begin{verbatim}
Fsignal signal=NULL; /* Internal use: no Input neither Output of module */

if  ( ((signal = mw_new_fsignal()) == NULL) ||
      (mw_alloc_fsignal(signal,1000) == NULL) )
    mwerror(FATAL,1,"Not enough memory.\n");
.
.
.
mw_delete_fsignal(signal);
signal = NULL;

\end{verbatim}

\newpage %......................................


\Name{mw\_new\_fsignal}{Create a new Fsignal}
\Summary{
Fsignal mw\_new\_fsignal();
}
\Description
This function creates a new \fsignal\ structure with an empty array \verb+values+.
No samples can be addressed at this time.
The array \verb+values+ should  be allocated using the function \verb+mw_alloc_fsignal+ or \verb+mw_change_fsignal+.

Do not use this function for input/output of modules, since the MegaWave2
Compiler already created the structure for you if you need it (see \volI). Use instead the function \verb+mw_change_fsignal+.
Do not forget to deallocate the internal structures before the end
of the module.

The function \verb+mw_new_fsignal+ returns \Null\ if not enough memory is available to create the structure. Your code should check this value to send an
error message in the \Null\ case, and do appropriate statement.

\Next
\Example
\begin{verbatim}
Fsignal signal=NULL; /* Internal use: no Input neither Output of module */

if  ( ((signal = mw_new_fsignal()) == NULL) ||
      (mw_alloc_fsignal(signal,1000) == NULL) )
    mwerror(FATAL,1,"Not enough memory.\n");
\end{verbatim}

\newpage %......................................


