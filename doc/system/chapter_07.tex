%
% Part 7 of the MegaWave2 Guide #2
%   Morphological structures
%

This section describes the various morphological structures used to represent
images. We call morphological representation\index{morphological representation}
 any complete decomposition which
is invariant by (local or global) contrast changes. More precisely, 
if $\cal R$ is the representation operator and $c$ a contrast change function
\index{contrast change} (that is, any non-decreasing real function), the contrast change
 invariance 
corresponds to the property ${\cal R}(c(u)) = c({\cal R}(u))$ for every image $u$.
Exemples of such representations are based by level sets\index{level set}, level lines and connected
components of level sets.

\def\shape{{\tt Shape}}
\def\shapes{{\tt Shapes}}

We begin our description with the \shape\index{shape} and \shapes\ structures. These are not
the first developed in MegaWave2, but they are going to play an increasing role :
they allow to handle level sets and connected components of level sets in a tree
structure very useful to develop morphological shape-based applications.
In addition, computation of these structures can be performed in a way faster
than the traditional level set decomposition, using the Fast Level Set
Transform\index{Fast Level Set Transform} ({\em FLST}\index{FLST|see{Fast Level Set Transform}} in short).
The FLST has been created by Pascal Monasse during its PhD thesis.
The following description of the \shape\ and \shapes\ structures has been
written with his help.
 
%-----------------------------------------------
\section{Shape}
%-----------------------------------------------


\label{shape}

A \shape\index{structure!\shape} is a set of pixels based on a level set of an
image. It can be a level set itself, one of its connected component,
or a shape as defined by the FLST (see module \verb+flst+) that is,
in short, a connected component of a level set with filled holes.
Notice that a \shape\ has no reference to the image in which it is extracted,
so a \shape\ can be constructed from scratch, without an initial image.

The basic fields are:
\begin{itemize}
\item \verb+inferior_type+: a nonzero value indicates that the \shape\
corresponds to a lower level set\index{level set!lower} of level $\lambda$ ($\{x : u(x) \leq \lambda\}$\footnote{we denote by $x$ a point in the image $u$.} or 
$\{x : u(x) < \lambda\}$, those sets being noted in short by
$[u \leq \lambda]$ and $[u<\lambda]$),
while a zero value indicates an upper level set\index{level set!upper} ($[u\geq \lambda]$ or $[u>\lambda]$).
\item \verb+value+: the gray level $\lambda$ of the level set.
\item \verb+area+: the area, i.e., the number of pixels of the shape.
\item \verb+pixels+: an array of pixel coordinates containing \verb+area+
elements.
\item \verb+boundary+: a \flist\ of dimension two containing the
vertices of a polygonal representation of the boundary.
\item \verb+open+: a nonzero value indicates that the \shape\ meets
the border of the image. The name of this field comes from the fact that if
the boundary is a curve, it is an open curve.
\end{itemize}
Moreover, there is an additional field \verb+removed+ indicating if the
shape is to be taken into account. This field is interesting only in the case
where the shape is part of a structure.

A shape is supposed to be included in a tree structure driven by
inclusion. This is the case for example when the shapes are all lower (or all
upper) level sets: in this case the tree has no ramification, since the level
sets are monotone for inclusion. There is a true tree structure when they are
{\em connected components} of lower (or upper) level sets. This is also true
for the shapes in the sense of the FLST.

In the vocabulary of graphs, the edges of the tree adjacent to the shape are
stored in the fields \verb+parent+, \verb+child+ and
\verb+next_sibling+. The \verb+child+ field corresponds actually to the
first child of the shape. The other ones can be recovered by following the
pointers \verb+next_sibling+. For example, to call the function \verb+foo+
successively with the children of shape \verb+s+ as argument, we would write
the following code snippet:\par
\verb+for(c = s->child; c != NULL; c=c->next_sibling) foo(c);+

The parent contains the shape while the shape contains its children. Functions
for accessing these three fields are given: they take into account that
some shapes may be ignored, as indicated by the field \verb+removed+.

It is dangerous to remove the root of the tree by setting its \verb+removed+
field: many functions rely on the fact that we have a root.

%+++++++++++++++++++++++++++++++++++++++++++++++
\subsection{The structure Shape}
%+++++++++++++++++++++++++++++++++++++++++++++++
\label{shape_structure}

The meaning of the different fields is explained above. There are two
additional fields, \verb+data+ and \verb+data_size+, whose content is left
to the choice of the user. \verb+data+ is supposed to point to a memory
extent of (at least) \verb+data_size+ bytes, if this value is
positive. Failure in this assumption may lead to a memory corruption in
I/O operations. 

{\small
\begin{verbatim}
typedef struct shape
{
  char inferior_type; /* Indicates if it is extracted from a superior 
                         or inferior level set */
  float value; /* Limiting gray-level of the level set */
  char open; /* Indicates if the shape meets the border of the image */
  int area; /* Area of the shape = area of the cc of level set 
                                 + areas of the holes */
  char removed; /* Indicates whether the shape exists or not */

  Point_plane pixels; /* The array of pixels contained in the shape */

  Flist boundary; /* The boundary curve defining the shape */

  /* Data to include it in a tree. It has a parent (the smallest containing 
     shape), children (the largest contained shapes, whose first is pChild 
     and the others are its siblings), and siblings (the other children of 
     its parent) */
  struct shape *parent, *next_sibling, *child;

  int data_size;     /* size of data[] in bytes */
  void* data;        /* User defined field (saved). A pointer to something */

} *Shape;
\end{verbatim}
}

%+++++++++++++++++++++++++++++++++++++++++++++++
\subsection{Related file (external) types}
%+++++++++++++++++++++++++++++++++++++++++++++++
\label{shape-file_type}

\begin{enumerate}
\item \verb+"MW2_SHAPE"+\index{file format!MW2\_SHAPE} MegaWave2 binary format.
\end{enumerate}

%+++++++++++++++++++++++++++++++++++++++++++++++
\subsection{Functions Summary}
%+++++++++++++++++++++++++++++++++++++++++++++++
\label{shape_function}

The following is a description of all the functions related to 
the \shape\ type. The list is in alphabetical order.

\newpage %......................................

\Name{mw\_change\_shape}{Create a \shape\ structure if necessary}
\Summary{
Shape mw\_change\_shape(sh)

Shape sh;
}
\Description
This function creates a \shape\ structure if \verb+sh+ is not
already defined. 
The fields are initialized to $0$ or \Null value.
The function returns the address of the structure, or
\Null\ if not enough memory is available.

\Next
\Example
\begin{verbatim}

Shape sh=NULL;

/*
  Define the structure
*/
sh = mw_change_shape(sh);
if (!sh) mwerror(FATAL,1,"Not enough memory to define the shape !\n");

/* 
  At that time, the shape is empty.
*/
\end{verbatim}

\newpage %......................................


\Name{mw\_delete\_shape}{Free the memory allocated for a \shape\ structure}
\Summary{
void mw\_delete\_shape(sh)

Shape sh;
}
\Description
This function deletes the \verb+pixels+ array, the \verb+boundary+ \flist,
the \verb+data+ array (if needed), and the structure itself.

\Next
\Example
\begin{verbatim}


Shape sh;

/*
  Define the structure
*/
sh = mw_new_shape();
if (!sh) mwerror(FATAL,1,"Not enough memory to define the shape !\n");

/*
    ...(computation of the shape)...
*/

/* 
  Free the shape, including data field. 
*/
mw_delete_shape(sh);

\end{verbatim}

\newpage %......................................
\Name{mw\_get\_first\_child\_shape}{Return the first child of a shape in the tree}
\Summary{
Shape mw\_get\_first\_child\_shape(sh)

Shape sh;
}
\Description
This function returns the first child of the shape \verb+sh+, skipping
removed shapes (field \verb+removed+). This is equivalent to
\verb+sh->child+ if this shape is not removed.

\newpage %......................................


\Name{mw\_get\_next\_sibling\_shape}{Return the next sibling of a shape in the
tree}
\Summary{
Shape mw\_get\_next\_sibling\_shape(sh)

Shape sh;
}
\Description
This function returns the next sibling (shape sharing the same parent) of
the shape \verb+sh+, skipping removed shapes (field \verb+removed+). This
is equivalent to \verb+sh->next_sibling+ if this shape is not removed.

\newpage %......................................


\Name{mw\_get\_not\_removed\_shape}{Return the first shape not removed in subtree}
\Summary{
Shape mw\_get\_not\_removed\_shape(sh)

Shape sh;
}
\Description
This function returns \verb+sh+ if this shape is not removed (field
\verb+removed+), else it is equivalent to \verb+mw_get_first_child(sh)+ that
is, it returns the first child, skipping removed shapes.

\newpage %......................................


\Name{mw\_get\_parent\_shape}{Return the parent of the shape in the tree}
\Summary{
Shape mw\_get\_parent\_shape(sh)

Shape sh;
}
\Description
This function returns the parent of
the shape \verb+sh+, skipping removed shapes (field \verb+removed+). This
is equivalent to \verb+sh->parent+ if this shape is not removed.


\newpage %......................................


\Name{mw\_get\_smallest\_shape}{Return the smallest shape containing a given pixel}
\Summary{
Shape mw\_get\_smallest\_shape(shs,x,y)

Shapes shs;
int x,y;

}

\Description
This function returns the smallest shape containing the pixel at position
$(x,y)$, ignoring removed shapes (field \verb+removed+). This is equivalent
to \verb!shs->smallest_shape[y*shs->ncol+x]! provided this shape is not
removed.

\newpage %......................................


\Name{mw\_new\_shape}{Create a \shape\ structure}
\Summary{
Shape mw\_new\_shape()

}
\Description
This function creates a new \shape\ structure. The fields are initialized
to $0$ or \Null value.
The function returns the address of the new structure, or
\Null\ if not enough memory is available.

\Next
\Example
\begin{verbatim}

Shape sh;

/*
  Define the structure
*/
sh = mw_new_shape();
if (!sh) mwerror(FATAL,1,"Not enough memory to define the shape !\n");

/* 
  At that time, the shape is empty.
*/
\end{verbatim}

\newpage %......................................




%-----------------------------------------------
\section{Shapes}
%-----------------------------------------------


\label{shapes}

A \shapes\index{structure!\shapes} structure is a collection of shapes extracted from an image. 
The fields \verb+nrow+ and \verb+ncol+ are the dimensions of the image. 
The field \verb+interpolation+ is the convention used to extract level
lines. Currently, the valid values are 0 (module \verb+flst+) and 1 (module
\verb+flst_bilinear+).

The elements are stored consecutively in the array \verb+the_shapes+ of size
\verb+nb_shapes+. By convention, the shape at index 0 is the root of the
tree.

The field \verb+smallest_shape+ is an array of size
\verb+nrow+$\times$\verb+ncol+ giving for each pixel the smallest shape in
the tree that contains it. By going upward in the tree, it is possible to know
all the shapes containing a given pixel.

%+++++++++++++++++++++++++++++++++++++++++++++++
\subsection{The structure Shapes}
%+++++++++++++++++++++++++++++++++++++++++++++++
\label{shapes_structure}

The meaning of the fields is explained above. The fields \verb+data_size+
and \verb+data+ are left to the user.
{\small
\begin{verbatim}
typedef struct shapes
{
  char cmt[mw_cmtsize];   /* Comments */
  char name[mw_namesize]; /* Name of the set */
  int nrow;               /* Number of rows (dy) of the image */
  int ncol;               /* Number of columns (dx) of the image */
  int interpolation;      /* Interpolation used for the level lines:
                             0=nearest neighbor, 1=bilinear */
  Shape the_shapes; /* Array of the shapes.
                       The root of the tree is at index 0 */
 
  int nb_shapes; /* The number of shapes (the size of the array the_shapes) */
 
  /* Link between pixels and shapes */
  Shape *smallest_shape; /* An image giving for each pixel
                            the smallest shape containing it */
 
  int data_size;     /* size of data[] in bytes */
  void* data;        /* User defined field (saved). A pointer to something */
 
} *Shapes;
\end{verbatim}
}

%+++++++++++++++++++++++++++++++++++++++++++++++
\subsection{Related file (external) types}
%+++++++++++++++++++++++++++++++++++++++++++++++
\label{shapes-file_type}

\begin{enumerate}
\item \verb+"MW2_SHAPES"+\index{file format!MW2\_SHAPE} MegaWave2 binary format.
\end{enumerate}

%+++++++++++++++++++++++++++++++++++++++++++++++
\subsection{Functions Summary}
%+++++++++++++++++++++++++++++++++++++++++++++++
\label{shapes_function}

The following is a description of all the functions related to 
the \shapes\ type. The list is in alphabetical order.

\newpage %......................................

\Name{mw\_alloc\_shapes}{Allocate the fields of a \shapes\ structure}
\Summary{
Shapes mw\_alloc\_shapes(shs, nrow, ncol, value)

Shapes shs;

int nrow, ncol;

float value;     /* gray level value of the root */
}

\Description
This function takes as argument a \shapes\ structure and returns it
after having allocated all necessary fields. The input \verb+nrow+ and 
\verb+ncol+ are the dimensions of the image. 
The field \verb+the_shapes+ is allocated to contain
\verb+nrow+$\times$\verb+ncol+$+1$ shapes, which is the maximal number of 
shapes extracted by the FLST (see module \verb+flst+). 
In fact, only one shape is put, the root of the tree, supposed to be 
extracted at gray level \verb+value+. 
The field \verb+smallest_shape+ is also allocated and
initialized, each pixel having as smallest shape the root.

The function returns \verb+shs+, or \Null\ if not enough memory is available
to do the allocation.

\Next
\Example
\begin{verbatim}

Shapes shs;
Fimage image; /* Assume image is allocated */

/*
  Define the structure
*/
shs = mw_new_shapes();
if (!shs) mwerror(FATAL,1,"Not enough memory to define the shapes !\n");

/* 
  At that time, the structure exists but fields are empty : alloc them
  to handle the Fimage image.
*/
if (!mw_alloc_shapes(shs, image->nrow, image->ncol, image->gray[0]))
  mwerror(FATAL,1,"Not enough memory to alloc the shapes !\n");

\end{verbatim}

\newpage %......................................


\Name{mw\_change\_shapes}{(Re)alloc the fields of a \shapes\ structure}
\Summary{
Shapes mw\_change\_shapes(shs, nrow, ncol, value)

Shapes shs;

int nrow, ncol;

float value;     /* gray level value of the root */
}

\Description
If the input pointer \verb+shs+ is \Null, create a new
structure, otherwise delete the currently allocated fields 
(if any) and call \verb+mw_alloc_shapes()+.

The function returns the new structure or \verb+shs+, or \Null\ if not enough 
memory is available to do the allocation.

\Next
\Example
\begin{verbatim}

Shapes shs=NULL;
Fimage image; /* Assume image is allocated */

/*
  Define the structure and alloc the field to handle the Fimage image.
*/
shs = mw_change_shapes(shs, image->nrow, image->ncol, image->gray[0]);
if (!shs) mwerror(FATAL,1,"Not enough memory to alloc the shapes !\n");

\end{verbatim}

\newpage %......................................


\Name{mw\_delete\_shapes}{Delete a \shapes\ structure}
\Summary{
void mw\_delete\_shapes(shs)

Shapes shs;

}

\Description
This function frees the allocated fields and the
structure itself. 
After this call, the memory pointed to by \verb+shs+ must
not be accessed any longer.
Warning : in the contrary to \verb+mw_delete_shape()+, the memory of the 
user-defined field \verb+data+ is not freed.
If this field has been allocated, you should free it before calling
\verb+mw_delete_shapes()+.


\Next
\Example
\begin{verbatim}

Shapes shs=NULL;
Fimage image; /* Assume image is allocated */

/*
  Define the structure and alloc the field to handle the Fimage image.
*/
shs = mw_change_shapes(shs, image->nrow, image->ncol, image->gray[0]);
if (!shs) mwerror(FATAL,1,"Not enough memory to alloc the shapes !\n");

/*
  ... (do the computation) ...
*/

/* 
  Delete the shapes
*/
if (!shs->data) free(shs->data);
mw_delete_shapes(shs);

\end{verbatim}

\newpage %......................................


\Name{mw\_new\_shapes}{Create a \shapes\ structure}
\Summary{
Shapes mw\_new\_shapes()

}
\Description
This function creates a new \shapes\ structure. The fields are initialized
to $0$ or \Null value.
The function returns the address of the new structure, or
\Null\ if not enough memory is available.

\Next
\Example
\begin{verbatim}

Shapes shs;

/*
  Define the structure
*/
shs = mw_new_shapes();
if (!shs) mwerror(FATAL,1,"Not enough memory to define the shapes !\n");

/* 
  At that time, the structure exists but is empty.
*/
\end{verbatim}

\newpage %......................................




%-----------------------------------------------
\section{Point with a type field}
%-----------------------------------------------


\label{point_type}
\def\pt{{\tt Point\_type}}
\def\mline{{\tt Morpho\_line}}
\def\fmline{{\tt Fmorpho\_line}}

The \pt\index{structure!\point_type@\pt} structure is complementary to the \point\ structure (See Section~\ref{curves-polygons_point}): it
is used to record the type of the point, a valuable information in morphological shape-based algorithms.
While the \point\ structure was mainly defined to be used as part of a \curve\ structure, the \pt\ structure
is related to the \mline\ structure (See Section~\ref{morpho_line}).

%+++++++++++++++++++++++++++++++++++++++++++++++
\subsection{The structure Point\_type}
%+++++++++++++++++++++++++++++++++++++++++++++++
\label{point_type-structure}

This is the C definition of the structure:
{\small
\begin{verbatim}
typedef struct point_type {
  unsigned char type; /* Type of the point, e.g. (exact meaning can vary; See modules)
                         0 : regular point;
                         1 : point in the image's border;
                         2 : T-junction;
                         3 : Tau-junction;
                         4 : X-junction;
                         5 : Y-junction.
                         */
  struct point_type *previous; /* Pointer to the previous point (may be NULL) */
  struct point_type *next; /* Pointer to the next point (may be NULL) */
} *Point_type;
\end{verbatim}
}

%+++++++++++++++++++++++++++++++++++++++++++++++
\subsection{Related file (external) types}
%+++++++++++++++++++++++++++++++++++++++++++++++
\label{point_type-file_type}

Not available: at this time, the \pt\ object cannot be used as input/output variables of modules.
It can be saved as part of a \mline\ or \fmline\ structure.

%+++++++++++++++++++++++++++++++++++++++++++++++
\subsection{Functions Summary}
%+++++++++++++++++++++++++++++++++++++++++++++++
\label{point_type-function}

The following is a description of all the functions related to 
the \pt\ type. The list is in alphabetical order.

\newpage %......................................

\Name{mw\_change\_point\_type}{Define the point\_type structure, if not defined}
\Summary{
Point\_type mw\_change\_point\_type(pt)

Point\_type pt;
}
\Description
This function returns a \pt\ structure if the input \verb+pt = NULL+.
It is provided despite the \\
\verb+mw_new_point_type()+ function for
global coherence with other memory types.

The function \verb+mw_change_point_type+ returns \Null\ if not enough memory is available to allocate the structure. 
Your code should check this return value to send an error message in the \Null\ case, and do appropriate statement.

Since the MegaWave2 compiler allocates structures for input and output 
objects (see \volI), this function is normally used only for internal objects.
Do not forget to deallocate the internal structures before the end
of the module, except if they are part of an input or output chain.

\Next
\Example
\begin{verbatim}

Point_type pt=NULL; /* Internal use: no Input neither Output of module */

/* Define a point type as image border */

pt = mw_change_point_type(pt);
if (pt == NULL) mwerror(FATAL,1,"Not enough memory.\n");
pt->type = 1; /* image border */
\end{verbatim}

\newpage %......................................


\Name{mw\_copy\_point\_type}{Copy all point types starting from the given one}
\Summary{
Point\_type mw\_copy\_point\_type(in,out)

Point\_type in, out;
}
\Description
This function copies the current point type and the next point types contained in the chain defined at the starting point type \verb+in+. 
The result is put in \verb+out+, which
may not be a predefined structure : in case of \verb+out=NULL+, the \verb+out+
structure is allocated.

The function \verb+mw_copy_point_type+ returns \Null\ if not enough memory is available to perform
the copy, or \verb+out+ elsewhere.
Your code should check this return value to send an error message in the \Null\ case, and do appropriate statement.

\Next
\Example
\begin{verbatim}

Point_type in; /* Predefined point */
Point_type out=NULL; 

out=mw_copy_point_type(in,out);
if (!out) mwerror(FATAL,1,"Not enough memory.\n");
\end{verbatim}

\newpage %......................................


\Name{mw\_delete\_point\_type}{Deallocate the point\_type structure}
\Summary{
void mw\_delete\_point\_type(pt)

Point\_type pt;
}
\Description
This function deallocates the \pt\ structures starting from
the given \verb+pt+, including this point itself.
You should set \verb+pt = NULL+ after this call since the address pointed
by \verb+pt+ is no longer valid.
To deallocate a point only and not all the next points of the
chain, just use \verb+free(pt)+.

\Next
\Example
\begin{verbatim}
/* Remove the first point_type of an existing morpho_line */

Morpho_line ll; /* Existing morpho_line (e.g. Input of module) */
Point_type pt;  /* Internal use */

pt = ll->first_type;
ll->first_type=pt->next;
pt->next->previous = NULL;
free(pt);
pt = NULL;

/* Remove all point_type of an existing morpho_line */

mw_delete_point_type(ll->first_type);
\end{verbatim}

\newpage %......................................


\Name{mw\_new\_point\_type}{Create a new point\_type structure}
\Summary{
Point\_type mw\_new\_point\_type();
}
\Description
This function creates a new \pt\ structure.
It returns \Null\ if not enough memory is available to create the structure.
Your code should check this value to send an
error message in the \Null\ case, and do appropriate statement.

Since the MegaWave2 compiler allocates structures for input and output 
objects (see \volI), this function is normally used only for internal objects.
Do not forget to deallocate the internal point structures before the end
of the module, except if they are part of an input or output curve.

\Next
\Example
\begin{verbatim}
/* Insert the point (0,0) with type 1 at the end of an existing morpho_line */

Morpho_line ll;  /* Existing morpho_line (e.g. Input of module) */
Point_curve point,p;  /* Internal use: no Input neither Output of module */
Point_type pt,t;

/* Define the point (0,0) with type 1 */
point = mw_new_point_curve();
if (point == NULL) mwerror(FATAL,1,"Not enough memory.\n");
pt = mw_new_point_type();
if (pt == NULL) mwerror(FATAL,1,"Not enough memory.\n");
point->x = point->y = 0;
pt->type=1;

/* Find the last point of the morpho_line */
p = ll->first_point; t = ll->first_type; 
while (p->next) {p=p->next; t=t->next;}

/* Insert the point */
p->next = point;
t->next = pt;
point->previous = p;
pt->previous = t;

/* Do not deallocate point_curve and point_type or morpho_line will become inconsistent */ 
\end{verbatim}
\newpage %......................................



%-----------------------------------------------
\section{Horizontal segment}
%-----------------------------------------------


\label{hsegment}

\def\hsegment{{\tt Hsegment}}
\def\mset{{\tt Morpho\_set}}

\index{set of points|see{segment}}

The \hsegment\ structure is useful for describing all pixels belonging to a (connected or non-connected)
set, without taking the border into consideration.
An horizontal segment is given by a left and a right point. 
If the shape of the set is more height than width, you should rather use vertical segments
(not yet defined).
The morpho set defined in Section~\ref{morpho_set} makes the use
of the \hsegment\ structure\index{segment!horizontal}, which defines an horizontal segment.

%+++++++++++++++++++++++++++++++++++++++++++++++
\subsection{The structure Hsegment}
%+++++++++++++++++++++++++++++++++++++++++++++++
\label{hsegment-structure}

This is the C definition of the structure \hsegment:
\index{structure!\hsegment@\hsegment} 
{\small
\begin{verbatim}
typedef struct hsegment {
  int xstart; /* Left x-coordinate of the segment */
  int xend;   /* Right x-coordinate of the segment */
  int y;      /* y-coordinate of the segment */
  struct hsegment *previous; /* Pointer to the previous segment (may be NULL) */
  struct hsegment *next;     /* Pointer to the next segment (may be NULL) */
} *Hsegment;

\end{verbatim}
}

%+++++++++++++++++++++++++++++++++++++++++++++++
\subsection{Related file (external) types}
%+++++++++++++++++++++++++++++++++++++++++++++++
\label{hsegment-file_type}

Not available: at this time, the \hsegment\ object cannot be used as input/output variables of modules.
It can be saved as part of a \mset\ structure.

%+++++++++++++++++++++++++++++++++++++++++++++++
\subsection{Functions Summary}
%+++++++++++++++++++++++++++++++++++++++++++++++
\label{hsegment-function}

The following is a description of all the functions related to 
the \hsegment\ type. The list is in alphabetical order.

\newpage %......................................

\Name{mw\_change\_hsegment}{Define the hsegment structure, if not defined}
\Summary{
Hsegment mw\_change\_hsegment(seg)

Hsegment seg;
}
\Description
This function returns a \hsegment\ structure if the input \verb+seg = NULL+.
It is provided despite the \\
\verb+mw_new_hsegment()+ function for
global coherence with other memory types.

The function \verb+mw_change_hsegment+ returns \Null\ if not enough memory is available to allocate the structure. 
Your code should check this return value to send an error message in the \Null\ case, and do appropriate statement.

Since the MegaWave2 compiler allocates structures for input and output 
objects (see \volI), this function is normally used only for internal objects.
Do not forget to deallocate the internal structures before the end
of the module, except if they are part of an input or output chain.

\Next
\Example
\begin{verbatim}

Hsegment seg=NULL; /* Internal use: no Input neither Output of module */

/* Define the horizontal segment (0,10)-(200,10) */

seg = mw_change_hsegment(seg);
if (seg == NULL) mwerror(FATAL,1,"Not enough memory.\n");
seg->xstart=0; 
seg->xend=200;
seg->y=10;
\end{verbatim}

\newpage %......................................


\Name{mw\_delete\_hsegment}{Deallocate a chain of horizontal segments}
\Summary{
void mw\_delete\_hsegment(seg)

Hsegment seg;
}
\Description
This function deallocates the chain of horizontal segments starting
from \verb+seg+. Previous segments are not deallocated.
You should set \verb+seg = NULL+ after this call since the address pointed
by \verb+seg+ is no longer valid.

\Next
\Example
\begin{verbatim}
Hsegment seg0,newseg,oldseg; 
int i;

/* Create a chain of 10 horizontal segments, starting from seg0 */

if (!(seg0=mw_new_hsegment())) mwerror(FATAL,1,"Not enough memory.\n");
seg0->xstart=0; seg0->xend=200; seg0->y=1;
oldseg=seg0;
for (i=2; i<=10; i++)
{
  if (!(newseg=mw_new_hsegment())) mwerror(FATAL,1,"Not enough memory.\n");
  newseg->xstart=0; newseg->xend=200; newseg->y=i;
  newseg->previous=oldseg;
  oldseg->next=newseg;
  oldseg=newseg;
}

/* .
   .
   (statement)
   .
   .
*/

/* Deallocate the chain of segments */
mw_delete_hsegment(seg0);
\end{verbatim}

\newpage %......................................


\Name{mw\_new\_hsegment}{Create a new hsegment structure}
\Summary{
Hsegment mw\_new\_hsegment()

}
\Description
This function returns a new \hsegment\ structure, or \Null\ if not enough 
memory is available to allocate the structure. 
Your code should check this return value to send an error message in the 
\Null\ case, and do appropriate statement.

The new structure is created with fields set to $0$ or \Null.

\Next
\Example
\begin{verbatim}

Hsegment seg; /* Internal use: no Input neither Output of module */

/* Define the horizontal segment (0,10)-(200,10) */

if (!(seg=mw_new_hsegment())) mwerror(FATAL,1,"Not enough memory.\n");
seg->xstart=0; 
seg->xend=200;
seg->y=10;
\end{verbatim}

\newpage %......................................





%-----------------------------------------------
\section{Morpho set}
%-----------------------------------------------


\label{morpho_set}

We call morpho set \index{morpho set} any connected component of set of the form
$\{x : \lambda_1 \leq u(x) \leq \lambda_2\}$, this set being noted in short by
$[\lambda_1 \leq u \leq \lambda_2]$. Notice that for $\lambda_1=-\infty$ we get
a lower level set\index{level set!lower} and for $\lambda_2=+\infty$ an upper level 
set\index{level set!upper}. In the case $\lambda_1 = \lambda_2$ the morpho set
will be called iso set\index{iso set}.
The structure \mset\index{structure!\morpho_set@\mset} can be used to handle such
morpho set. A \mset\ is given by a list of horizontal segments (See Section~\ref{hsegment}),
where levels $\lambda_1$ and $\lambda_2$ are recorded. Some additional information can
be recorded, such as the neighbor morpho sets. 
Please notice that some fields are likely to change in the future.

%+++++++++++++++++++++++++++++++++++++++++++++++
\subsection{The structure Morpho\_set}
%+++++++++++++++++++++++++++++++++++++++++++++++
\label{mset-structure}

This is the C definition of the structure \mset:
\index{structure!\morpho_set@\mset} 
{\small
\begin{verbatim}
typedef struct morpho_set {
  unsigned int num;      /* Morpho set number (range in the Morpho_sets struct.) */
  Hsegment first_segment; /* Pointer to the first segment of the morpho set */
  Hsegment last_segment;  /* Pointer to the last segment of the morpho set */  
  float minvalue;        /* Minimum gray level value of this set */
  float maxvalue;        /* Maximum gray level value of this set */
  unsigned char stated;  /* 1 if this m.s. has already been stated, 0 otherwise */
  int area;              /* Area of the set (number of pixels belonging to this set) */
  struct morpho_sets *neighbor; /* Pointer to a chain of neighbor morpho sets (may be NULL)*/
} *Morpho_set;
\end{verbatim}
}

%+++++++++++++++++++++++++++++++++++++++++++++++
\subsection{Related file (external) types}
%+++++++++++++++++++++++++++++++++++++++++++++++
\label{mset-file_type}

\begin{enumerate}
\item \verb+"MW2_MORPHO_SET"+\index{file format!MW2\_MORPHO\_SET} MegaWave2 binary format.
\end{enumerate}

%+++++++++++++++++++++++++++++++++++++++++++++++
\subsection{Functions Summary}
%+++++++++++++++++++++++++++++++++++++++++++++++
\label{mset-function}

The following is a description of all the functions related to 
the \mset\ type. The list is in alphabetical order.

\newpage %......................................

\Name{mw\_change\_morpho\_set}{Define a morpho set, if not already defined}
\Summary{
Morpho\_set mw\_change\_morpho\_set(ms)

Morpho\_set ms;
}
\Description
This function returns a \mset\ structure if the input \verb+ms = NULL+.
It is provided despite the \\
\verb+mw_new_morpho_set()+ function for
global coherence with other memory types.

The function \verb+mw_change_morpho_set+ returns \Null\ if not enough memory is available to allocate the structure. 
Your code should check this return value to send an error message in the \Null\ case, and do appropriate statement.

Since the MegaWave2 compiler allocates structures for input and output 
objects (see \volI), this function is normally used only for internal objects.
Do not forget to deallocate the internal structures before the end
of the module, except if they are part of an input or output chain.

\Next
\Example
\begin{verbatim}

Morpho_set ms=NULL; /* Internal use: no Input neither Output of module */
Hsegment seg=NULL;

/* Define a morpho set containing one segment only */

if (!(seg=mw_change_hsegment(seg)) ||
    !(ms=mw_change_morpho_set(ms))) mwerror(FATAL,1,"Not enough memory.\n");
seg->xstart=0; 
seg->xend=200;
seg->y=10;

ms->first_segment=seg;
ms->minvalue=0.0;
ms->maxvalue = 1.0;
ms->area=201;

\end{verbatim}

\newpage %......................................


\Name{mw\_copy\_morpho\_set}{Copy a morpho set into another one}
\Summary{
Morpho\_set mw\_copy\_morpho\_set(in,out)

Morpho\_set in, out;
}
\Description
This function copies the \mset\ \verb+in+ into \verb+out+.
The chain of segments are also duplicated.
The result is put in \verb+out+, which may not be a predefined structure : in case 
of \verb+out=NULL+, the \verb+out+ structure is allocated.

The function \verb+mw_copy_morpho_set+ returns \Null\ if not enough memory is available to perform
the copy, or \verb+out+ elsewhere.
Your code should check this return value to send an error message in the \Null\ case, and do appropriate statement.

\Next
\Example
\begin{verbatim}

Morpho_set in; /* Predefined morpho_set */
Morpho_set out=NULL; 

out=mw_copy_morpho_set(in,out);
if (!out) mwerror(FATAL,1,"Not enough memory.\n");
\end{verbatim}

\newpage %......................................


\Name{mw\_delete\_morpho\_set}{Deallocate a morpho set}
\Summary{
void mw\_delete\_morpho\_set(ms)

Morpho\_set ms;
}
\Description
This function deallocates the \mset\ \verb+ms+, including the
chain of horizontal segments.
You should set \verb+ms = NULL+ after this call since the address pointed
by \verb+ms+ is no longer valid.

\Next
\Example
\begin{verbatim}

Morpho_set ms; /* Internal use: no Input neither Output of module */
Hsegment seg;

/* Define a morpho set containing one segment only */

if (!(seg=mw_new_hsegment()) ||
    !(ms=mw_new_morpho_set())) mwerror(FATAL,1,"Not enough memory.\n");
seg->xstart=0; 
seg->xend=200;
seg->y=10;

ms->first_segment=seg;
ms->minvalue=0.0;
ms->maxvalue = 1.0;
ms->area=201;

/* .
   .
   (statement)
   .
   .
*/

/* Deallocate the morpho_set */
mw_delete_morpho_set(ms);

\end{verbatim}

\newpage %......................................


\Name{mw\_length\_morpho\_set}{Return the number of segments a morpho set contains}
\Summary{
unsigned int mw\_length\_morpho\_set(ms)

Morpho\_set ms;
}
\Description
This function returns the number of segments contained in the input
\verb+ms+.
It returns $0$ if the structure is empty or undefined.

\Next
\Example
\begin{verbatim}

Morpho_set ms=NULL; /* Internal use: no Input neither Output of module */
Hsegment seg=NULL;

/* Define a morpho set containing one segment only */

if (!(seg=mw_change_hsegment(seg)) ||
    !(ms=mw_change_morpho_set(ms))) mwerror(FATAL,1,"Not enough memory.\n");
seg->xstart=0; 
seg->xend=200;
seg->y=10;

ms->first_segment=seg;
ms->minvalue=0.0;
ms->maxvalue = 1.0;
ms->area=201;

/* This will print 1 */
printf("%d",mw_length_morpho_set(ms));
\end{verbatim}

\newpage %......................................


\Name{mw\_new\_morpho\_set}{Create a new morpho set}
\Summary{
Morpho\_set mw\_new\_morpho\_set()

}
\Description
This function returns a new \mset\ structure, or \Null\ if not enough 
memory is available to allocate the structure. 
Your code should check this return value to send an error message in the 
\Null\ case, and do appropriate statement.

The new structure is created with fields set to $0$ or \Null.

\Next
\Example
\begin{verbatim}

Morpho_set ms; /* Internal use: no Input neither Output of module */
Hsegment seg;

/* Define a morpho set containing one segment only */

if (!(seg=mw_new_hsegment()) ||
    !(ms=mw_new_morpho_set())) mwerror(FATAL,1,"Not enough memory.\n");
seg->xstart=0; 
seg->xend=200;
seg->y=10;

ms->first_segment=seg;
ms->minvalue=0.0;
ms->maxvalue = 1.0;
ms->area=201;
\end{verbatim}

\newpage %......................................




%-----------------------------------------------
\section{Chain of morpho sets}
%-----------------------------------------------


\label{morpho_sets}
\def\msets{{\tt Morpho\_sets}}
\def\mimage{{\tt Mimage}}

The \msets\index{structure!\morpho_sets@\msets} structure is useful to record a set (or chain) of 
morpho sets. This structure is used by the \mimage\ structure (See Section~\ref{mimage}) to 
handle all the morpho sets an image contains.

%+++++++++++++++++++++++++++++++++++++++++++++++
\subsection{The structure Morpho\_sets}
%+++++++++++++++++++++++++++++++++++++++++++++++
\label{msets-structure}

This is the C definition of the structure \msets:
\index{structure!\morpho_sets@\msets} 
{\small
\begin{verbatim}
typedef struct morpho_sets {
  Morpho_set morphoset;         /* Pointer to the current morpho set */
  struct morpho_sets *previous; /* Pointer to the previous morpho sets of the chain */
  struct morpho_sets *next;     /* Pointer to the next morpho sets of the chain */
  /* For use in Mimage only */
  struct morpho_line *morpholine;  /* Pointer to the associated morpho line */
} *Morpho_sets;
\end{verbatim}
}

%+++++++++++++++++++++++++++++++++++++++++++++++
\subsection{Related file (external) types}
%+++++++++++++++++++++++++++++++++++++++++++++++
\label{msets-file_type}

\begin{enumerate}
\item \verb+"MW2_MORPHO_SETS"+\index{file format!MW2\_MORPHO\_SETS} MegaWave2 binary format.
\end{enumerate}

%+++++++++++++++++++++++++++++++++++++++++++++++
\subsection{Functions Summary}
%+++++++++++++++++++++++++++++++++++++++++++++++
\label{msets-function}

The following is a description of all the functions related to 
the \msets\ type. The list is in alphabetical order.

\newpage %......................................

% === A POURSUIVRE ===
\Name{mw\_change\_morpho\_sets}{Define a morpho sets, if not already defined}
\Summary{
Morpho\_sets mw\_change\_morpho\_sets(mss)

Morpho\_sets mss;
}
\Description
This function returns a \msets\ structure if the input \verb+mss = NULL+.
It is provided despite the \\
\verb+mw_new_morpho_sets()+ function for
global coherence with other memory types.

The function \verb+mw_change_morpho_sets+ returns \Null\ if not enough memory is available to allocate the structure. 
Your code should check this return value to send an error message in the \Null\ case, and do appropriate statement.

Since the MegaWave2 compiler allocates structures for input and output 
objects (see \volI), this function is normally used only for internal objects.
Do not forget to deallocate the internal structures before the end
of the module, except if they are part of an input or output chain.

\Next
\Example
\begin{verbatim}
Morpho_sets mss=NULL; /* Internal use: no Input neither Output of module */
Hsegment seg;
Morpho_set ms;

/* Define a morpho sets containing one morpho set */

if (!(seg=mw_new_hsegment()) ||
    !(ms=mw_new_morpho_set()) ||
    !(mss=mw_change_morpho_sets(mss))) mwerror(FATAL,1,"Not enough memory.\n");
seg->xstart=0; 
seg->xend=200;
seg->y=10;

ms->first_segment=seg;
ms->minvalue=0.0;
ms->maxvalue = 1.0;
ms->area=201;

mss->morphoset=ms;


\end{verbatim}

\newpage %......................................


\Name{mw\_copy\_morpho\_sets}{Copy a morpho sets into another one}
\Summary{
Morpho\_sets mw\_copy\_morpho\_sets(in,out)

Morpho\_sets in, out;
}
\Description
This function copies the \msets\ \verb+in+ into \verb+out+.
The \mset\ pointed by the \verb+in->morphoset+ field is not only copied, but
also all the chain starting from \verb+in+.
The neighbor \msets\ pointed by each \mset\ are also copied.
The result is put in \verb+out+, which may not be a predefined structure : in case 
of \verb+out=NULL+, the \verb+out+ structure is allocated.

The function \verb+mw_copy_morpho_sets+ returns \Null\ if not enough memory is available to perform
the copy, or \verb+out+ elsewhere.
Your code should check this return value to send an error message in the \Null\ case, and do appropriate statement.

\Next
\Example
\begin{verbatim}

Morpho_sets in; /* Predefined morpho_sets */
Morpho_sets out=NULL; 

out=mw_copy_morpho_sets(in,out);
if (!out) mwerror(FATAL,1,"Not enough memory.\n");
\end{verbatim}

\newpage %......................................


\Name{mw\_delete\_morpho\_sets}{Deallocate a morpho sets}
\Summary{
void mw\_delete\_morpho\_sets(mss)

Morpho\_sets mss;
}
\Description
This function frees the \mset\ \verb+mss->morphoset+, all the
chain starting from \verb+mss+ and it deallocates the \msets\ \verb+mss+ structure.
You should sets \verb+mss = NULL+ after this call since the address pointed
by \verb+mss+ is no longer valid.

\Next
\Example
\begin{verbatim}

Morpho_sets mss; /* Internal use: no Input neither Output of module */
Hsegment seg;
Morpho_set ms;

/* Define a morpho sets containing one morpho set */

if (!(seg=mw_new_hsegment()) ||
    !(ms=mw_new_morpho_set()) ||
    !(mss=mw_new_morpho_sets())) mwerror(FATAL,1,"Not enough memory.\n");
seg->xstart=0; 
seg->xend=200;
seg->y=10;

ms->first_segment=seg; ms->minvalue=0.0; ms->maxvalue = 1.0; ms->area=201;
mss->morphoset=ms;

/* .
   .
   (statement)
   .
   .
*/

/* Deallocate the morpho_sets mss ; ms and seg will be also deallocated. */
mw_delete_morpho_set(mss);

\end{verbatim}

\newpage %......................................


\Name{mw\_length\_morpho\_sets}{Return the number of morpho sets a Morpho\_sets structure contains}
\Summary{
unsigned int mw\_length\_morpho\_sets(mss)

Morpho\_sets mss;
}
\Description
This function returns the number of morpho sets the \msets\ structure \verb+mss+ contains,
starting the chain from the current position given by \verb+mss+.
It returns $0$ if the structure is empty or undefined.

\Next
\Example
\begin{verbatim}

Morpho_sets mss; /* Internal use: no Input neither Output of module */
Hsegment seg;
Morpho_set ms;

/* Define a morpho sets containing one morpho set */

if (!(seg=mw_new_hsegment()) ||
    !(ms=mw_new_morpho_set()) ||
    !(mss=mw_new_morpho_sets())) mwerror(FATAL,1,"Not enough memory.\n");
seg->xstart=0; 
seg->xend=200;
seg->y=10;

ms->first_segment=seg; ms->minvalue=0.0; ms->maxvalue = 1.0; ms->area=201;
mss->morphoset=ms;

/* This will print 1 */
printf("%d",mw_length_morpho_sets(mss));
\end{verbatim}

\newpage %......................................


\Name{mw\_new\_morpho\_sets}{Create a new morpho sets}
\Summary{
Morpho\_sets mw\_new\_morpho\_sets()

}
\Description
This function returns a new \msets\ structure, or \Null\ if not enough 
memory is available to allocate the structure. 
Your code should check this return value to send an error message in the 
\Null\ case, and do appropriate statement.

The new structure is created with fields set to $0$ or \Null.

\Next
\Example
\begin{verbatim}

Morpho_sets mss; /* Internal use: no Input neither Output of module */
Hsegment seg;
Morpho_set ms;

/* Define a morpho sets containing one morpho set */

if (!(seg=mw_new_hsegment()) ||
    !(ms=mw_new_morpho_set()) ||
    !(mss=mw_new_morpho_sets())) mwerror(FATAL,1,"Not enough memory.\n");
seg->xstart=0; 
seg->xend=200;
seg->y=10;

ms->first_segment=seg;
ms->minvalue=0.0;
ms->maxvalue = 1.0;
ms->area=201;

mss->morphoset=ms;

\end{verbatim}

\newpage %......................................




%-----------------------------------------------
\section{Morpho line}
%-----------------------------------------------


\label{morpho_line}
\def\mline{{\tt Morpho\_line}}

A morpho line \index{morpho line} is the border of a morpho set.
Assuming a right choice of grid and point connectivity so that a Jordan's theorem 
follows, a morpho line is a closed curve dividing the grid in two connected components : 
the interior of the morpho set and the exterior one. Actually, because an image has a finite
support, a morpho line may also intersects the image border : in such case, the
curve remains open. There is another restriction to the Jordan's theorem : most
of modules using morpho lines (such as \verb+ml_extract+) consider the 4-connectivity only in
the square grid, so the border may cut the connected component to several pieces and the 
corresponding morpho lines may be self-intersecting.
Notice that if the morpho set is a level set, the corresponding border is a
level line\index{level line}. And if the morpho set is an iso set, its border is
an iso line\index{iso line}.

The structure \mline\index{structure!\morpho_line@\mline} can be used to handle such
morpho line. First a \mline\ is a curve, so the \point\ structure is used to record it
(field \verb+first_point+). There are additional fields, to give information on the
line (type of the points, closed or open curve) and to allow the reconstruction of the
morpho set (\verb+minvalue+, \verb+maxvalue+).

%+++++++++++++++++++++++++++++++++++++++++++++++
\subsection{The structure Morpho\_line}
%+++++++++++++++++++++++++++++++++++++++++++++++
\label{mline-structure}

This is the C definition of the structure \mline:
\index{structure!\morpho_line@\mline} 
{\small
\begin{verbatim}
typedef struct morpho_line {
  Point_curve first_point;/* Pointer to the first point of the morpho_line curve */
  Point_type first_type;  /* Pointer to the first Point_type */
  float minvalue;         /* Minimum gray level value of this morpho line */
  float maxvalue;         /* Maximum gray level value of this morpho line */
  unsigned char open;     /* 0 if the morpho line is closed, opened otherwise */
  float data;             /* User-defined data field (saved) */
  void *pdata;            /* User-defined data field : pointer to something (not saved) */
   
  /* For use in Mimage only */
  struct morpho_sets *morphosets;/* Pointer to the associated morpho sets */
  unsigned int num;              /* Morpho line number (range in the chain) */
  struct morpho_line *previous;  /* Pointer to the previous m.l. (may be NULL) */
  struct morpho_line *next;      /* Pointer to the next m.l. (may be NULL) */
} *Morpho_line;
\end{verbatim}
}

%+++++++++++++++++++++++++++++++++++++++++++++++
\subsection{Related file (external) types}
%+++++++++++++++++++++++++++++++++++++++++++++++
\label{mline-file_type}

\begin{enumerate}
\item \verb+"MW2_MORPHO_LINE"+\index{file format!MW2\_MORPHO\_LINE} MegaWave2 binary format.
\end{enumerate}

%+++++++++++++++++++++++++++++++++++++++++++++++
\subsection{Functions Summary}
%+++++++++++++++++++++++++++++++++++++++++++++++
\label{mline-function}

The following is a description of all the functions related to 
the \mline\ type. The list is in alphabetical order.

\newpage %......................................

\Name{mw\_change\_morpho\_line}{Define a morpho line, if not already defined}
\Summary{
Morpho\_line mw\_change\_morpho\_line(ml)

Morpho\_line ml;
}
\Description
This function returns a \mline\ structure if the input \verb+ml = NULL+.
It is provided despite the \\
\verb+mw_new_morpho_line()+ function for
global coherence with other memory types.

The function \verb+mw_change_morpho_line+ returns \Null\ if not enough memory is available to allocate the structure. 
Your code should check this return value to send an error message in the \Null\ case, and do appropriate statement.

Since the MegaWave2 compiler allocates structures for input and output 
objects (see \volI), this function is normally used only for internal objects.
Do not forget to deallocate the internal structures before the end
of the module, except if they are part of an input or output chain.

\Next
\Example
\begin{verbatim}

/* Copy the curve of a morpho line into another morpho line */
Morpho_line in,out=NULL;

out=mw_change_morpho_line(out);
if (!out) mwerror(FATAL,1,"Not enough memory !\n");
out->open = in->open;
if ( ((out->first_point = mw_new_point_curve()) == NULL) ||
     ((out->first_type = mw_new_point_type()) == NULL) )
   mwerror(FATAL, 1,"Not enough memory !\n");
mw_copy_point_curve(in->first_point,out->first_point);
mw_copy_point_type(in->first_type,out->first_type);

\end{verbatim}

\newpage %......................................


\Name{mw\_copy\_morpho\_line}{Copy a morpho line into another one}
\Summary{
Morpho\_line mw\_copy\_morpho\_line(in,out)

Morpho\_line in, out;
}
\Description
This function copies the \mline\ \verb+in+ into \verb+out+.
All fields are copied but the following : \verb+pdata+, \verb+morphosets+,
\verb+num+, \verb+previous+ and \verb+next+.
The result is put in \verb+out+, which may not be a predefined structure : in case 
of \verb+out=NULL+, the \verb+out+ structure is allocated.

The function \verb+mw_copy_morpho_line+ returns \Null\ if not enough memory is available to perform
the copy, or \verb+out+ elsewhere.
Your code should check this return value to send an error message in the \Null\ case, and do appropriate statement.

\Next
\Example
\begin{verbatim}

Morpho_line in; /* Predefined morpho_line */
Morpho_line out=NULL; 

out=mw_copy_morpho_line(in,out);
if (!out) mwerror(FATAL,1,"Not enough memory.\n");
\end{verbatim}

\newpage %......................................


\Name{mw\_delete\_morpho\_line}{Deallocate a morpho line}
\Summary{
void mw\_delete\_morpho\_line(ml)

Morpho\_line ml;
}
\Description
This function deallocates the \mline\ \verb+ml+ structure, including the
curve (\point) and the chain of types (\pt). Other pointers are
not deallicated.
You should line \verb+ml = NULL+ after this call since the address pointed
by \verb+ml+ is no longer valid.

\Next
\Example
\begin{verbatim}

Morpho_line ml; /* Internal use: no Input neither Output of module */
Point_curve pt;

/* Define a morpho line containing the point (0,0) only */

if (!(pt=mw_new_point_curve()) ||
    !(ml=mw_new_morpho_line())) mwerror(FATAL,1,"Not enough memory.\n");
pt->x=pt->y=0;
ml->first_point=pt;

/* .
   .
   (statement)
   .
   .
*/

/* Deallocate the morpho_line */
mw_delete_morpho_line(ml);

\end{verbatim}

\newpage %......................................


\Name{mw\_length\_morpho\_line}{Return the number of points a morpho line contains}
\Summary{
unsigned int mw\_length\_morpho\_line(ml)

Morpho\_line ml;
}
\Description
This function returns the number of points contained in the input
\verb+ml+.
It returns $0$ if the structure is empty or undefined.
If the field \verb+first_type+ is not \Null, the number of points
defined by this field must equal the number of points in the curve.

\Next
\Example
\begin{verbatim}

Morpho_line ml; /* Internal use: no Input neither Output of module */
Point_curve pt;

/* Define a morpho line containing the point (0,0) only */

if (!(pt=mw_new_point_curve()) ||
    !(ml=mw_new_morpho_line())) mwerror(FATAL,1,"Not enough memory.\n");
pt->x=pt->y=0;
ml->first_point=pt;

/* This will print 1 */
printf("%d",mw_length_morpho_line(ml));
\end{verbatim}

\newpage %......................................


\Name{mw\_new\_morpho\_line}{Create a new morpho line}
\Summary{
Morpho\_line mw\_new\_morpho\_line()

}
\Description
This function returns a new \mline\ structure, or \Null\ if not enough 
memory is available to allocate the structure. 
Your code should check this return value to send an error message in the 
\Null\ case, and do appropriate statement.

The new structure is created with fields set to $0$ or \Null.

\Next
\Example
\begin{verbatim}

/* Copy the curve of a morpho line into another morpho line */
Morpho_line in,out;

out=mw_new_morpho_line();
if (!out) mwerror(FATAL,1,"Not enough memory !\n");
out->open = in->open;
if ( ((out->first_point = mw_new_point_curve()) == NULL) ||
     ((out->first_type = mw_new_point_type()) == NULL) )
   mwerror(FATAL, 1,"Not enough memory !\n");
mw_copy_point_curve(in->first_point,out->first_point);
mw_copy_point_type(in->first_type,out->first_type);

\end{verbatim}

\newpage %......................................




%-----------------------------------------------
\section{Morpho line in the continuous plane} 
%-----------------------------------------------


\label{fmorpho_line}
\def\fmline{{\tt Fmorpho\_line}}

The structure \fmline\ \index{structure!\fmorpho_line@\fmline} is used to
handle morpho lines in the continuous plane.
Indeed, if the morpho lines obtained from digital images contain discrete
(integer) coordinates, one may want to process the morpho lines using
continuous operators, such as geometric smoothing. The resulting morpho
lines are no more made by discrete coordinates.
In a \fmline, the points are recorded using the \fpoint\ structure (See
Section~\ref{curves-polygons_real}).

%+++++++++++++++++++++++++++++++++++++++++++++++
\subsection{The structure Fmorpho\_line}
%+++++++++++++++++++++++++++++++++++++++++++++++
\label{fmline-structure}

This is the C definition of the structure \fmline:
\index{structure!\fmorpho_line@\fmline} 
{\small
\begin{verbatim}
typedef struct fmorpho_line {
  Point_fcurve first_point;/* Pointer to the first point of the fmorpho_line curve */
  Point_type first_type;   /* Pointer to the first Point_type */
  float minvalue;          /* Minimum gray level value of this morpho line */
  float maxvalue;          /* Maximum gray level value of this morpho line */
  unsigned char open;      /* 0 if the morpho line is closed, opened otherwise */
  float data;             /* User-defined data field (saved) */
  void *pdata;            /* User-defined data field : pointer to something (not saved) */
 
  /* For use in Mimage only */
  struct fmorpho_line *previous; /* Pointer to the previous m.l. (may be NULL) */
  struct fmorpho_line *next;     /* Pointer to the next m.l. (may be NULL) */
} *Fmorpho_line;
\end{verbatim}
}

%+++++++++++++++++++++++++++++++++++++++++++++++
\subsection{Related file (external) types}
%+++++++++++++++++++++++++++++++++++++++++++++++
\label{fmline-file_type}

\begin{enumerate}
\item \verb+"MW2_FMORPHO_LINE"+\index{file format!MW2\_FMORPHO\_LINE} MegaWave2 binary format.
\end{enumerate}

%+++++++++++++++++++++++++++++++++++++++++++++++
\subsection{Functions Summary}
%+++++++++++++++++++++++++++++++++++++++++++++++
\label{fmline-function}

We won't waste space to describe functions related to the \fmline\ structure : they
are the same than those related to \mline, except that the name ``morpho\_line'' has
to be changed to ``fmorpho\_line''.

%-----------------------------------------------
\section{Morphological image}
%-----------------------------------------------


\label{mimage}

A morphological image\index{morphological image} may record in a structure called \mimage\ all morpho sets
and morpho lines the image contains. It is therefore potentially a very redundant
(and very huge) structure, but this plenty of information may be useful to perform 
morphological operations. Of course, not all fields need to be set as the same time,
for example a \mimage\ may contain the level lines only. But from this (complete) 
information, all other fields may be computed.

The \mimage\ structure has been created before the \shapes\ structure was developed
(See Section~\ref{shapes}).
It does not use the tree structure associated to FLST-based algorithms.
For this reason, the \shapes\ object should be preferred to the \mimage\ one for
future developments.

%+++++++++++++++++++++++++++++++++++++++++++++++
\subsection{The structure Mimage}
%+++++++++++++++++++++++++++++++++++++++++++++++
\label{mimage-structure}

This is the C definition of the structure \mimage:
\index{structure!\mimage@\mimage} 
{\small
\begin{verbatim}
typedef struct mimage {
  char cmt[mw_cmtsize];   /* Comments */
  char name[mw_namesize]; /* Name of the set */
  int nrow;               /* Number of rows (dy) */
  int ncol;               /* Number of columns (dx) */
  float minvalue;         /* Minimal Gray level value in the image */
  float maxvalue;         /* Maximal Gray level value in the image */
  Morpho_line first_ml;   /* Pointer to the first morpho line in the discrete grid */
  Fmorpho_line first_fml; /* Pointer to the first morpho line in the continuous plane */
  Morpho_sets first_ms;   /* Pointer to the first morpho sets in the discrete grid */
} *Mimage;
\end{verbatim}
}

%+++++++++++++++++++++++++++++++++++++++++++++++
\subsection{Related file (external) types}
%+++++++++++++++++++++++++++++++++++++++++++++++
\label{mimage-file_type}

\begin{enumerate}
\item \verb+"MW2_MIMAGE"+\index{file format!MW2\_MIMAGE} MegaWave2 binary format.
\end{enumerate}

%+++++++++++++++++++++++++++++++++++++++++++++++
\subsection{Functions Summary}
%+++++++++++++++++++++++++++++++++++++++++++++++
\label{mimage-function}

The following is a description of all the functions related to 
the \mimage\ type. The list is in alphabetical order.

\newpage %......................................

\Name{mw\_change\_mimage}{Define a morphological image, if not already defined}
\Summary{
Mimage mw\_change\_mimage(mi)

Mimage mi;
}
\Description
This function returns a \mimage\ structure if the input \verb+mi = NULL+.
It is provided despite the \\
\verb+mw_new_mimage()+ function for
global coherence with other memory types.

The function \verb+mw_change_mimage+ returns \Null\ if not enough memory is available to allocate the structure. 
Your code should check this return value to send an error message in the \Null\ case, and do appropriate statement.

Since the MegaWave2 compiler allocates structures for input and output 
objects (see \volI), this function is normally used only for internal objects.
Do not forget to deallocate the internal structures before the end
of the module, except if they are part of an input or output chain.

\Next
\Example
\begin{verbatim}
/* Copy the morpho lines only of a morphological image into another morphological image */

Mimage in,out=NULL;

out=mw_change_mimage(out);
if (!out) mwerror(FATAL,1,"Not enough memory !\n");
out->nrow = in->nrow;
out->ncol = in->ncol;
out->minvalue=in->minvalue;
out->maxvalue=in->maxvalue;
if (in->firstml) 
  {
    out->firstml=mw_copy_morpho_line(in->firstml, out->firstml);
    if (!out->firstml) mwerror(FATAL, 1,"Not enough memory !\n");
  }
\end{verbatim}

\newpage %......................................


\Name{mw\_copy\_mimage}{Copy a morphological image into another one}
\Summary{
Mimage mw\_copy\_mimage(in,out)

Mimage in, out;
}
\Description
This function copies the \mimage\ \verb+in+ into \verb+out+.
All fields are copied, including the chains of \msets, \mline\ and \fmline.
The result is put in \verb+out+, which may not be a predefined structure : in case 
of \verb+out=NULL+, the \verb+out+ structure is allocated.

The function \verb+mw_copy_mimage+ returns \Null\ if not enough memory is available to perform
the copy, or \verb+out+ elsewhere.
Your code should check this return value to send an error message in the \Null\ case, and do appropriate statement.

\Next
\Example
\begin{verbatim}

mimage in; /* Predefined mimage */
mimage out=NULL; 

out=mw_copy_mimage(in,out);
if (!out) mwerror(FATAL,1,"Not enough memory.\n");
\end{verbatim}

\newpage %......................................


\Name{mw\_delete\_mimage}{Deallocate a morphological image}
\Summary{
void mw\_delete\_mimage(mi)

mimage mi;
}
\Description
This function deallocates the \mimage\ \verb+mi+ structure, including the
chains of \msets, \mline\ and \fmline.
You should line \verb+mi = NULL+ after this call since the address pointed
by \verb+mi+ is no longer valid.

\Next
\Example
\begin{verbatim}

Mimage mi; /* Internal use: no Input neither Output of module */

Morpho_line ml; 
Point_curve pt;
Fmorpho_line fml; 
Point_fcurve fpt;
Morpho_sets mss;
Hsegment seg;
Morpho_set ms;


/* Define a morpho line containing the point (0,0) only */

if (!(pt=mw_new_point_curve()) ||
    !(ml=mw_new_morpho_line())) mwerror(FATAL,1,"Not enough memory.\n");
pt->x=pt->y=0;
ml->first_point=pt;

/* Define a fmorpho line containing the point (0.5,0.5) only */

if (!(fpt=mw_new_point_fcurve()) ||
    !(fml=mw_new_fmorpho_line())) mwerror(FATAL,1,"Not enough memory.\n");
fpt->x=fpt->y=0.5;
fml->first_point=fpt;

/* Define a morpho sets containing one morpho set */

if (!(seg=mw_new_hsegment()) ||
    !(ms=mw_new_morpho_set()) ||
    !(mss=mw_new_morpho_sets())) mwerror(FATAL,1,"Not enough memory.\n");
seg->xstart=0; 
seg->xend=200;
seg->y=10;
ms->first_segment=seg; ms->minvalue=0.0; ms->maxvalue = 1.0; ms->area=201;
mss->morphoset=ms;

/* Define a morphological image made by one morpho line, one fmorpho line and
   one morpho sets.
*/

if (!(mi=mw_new_morpho_line())) mwerror(FATAL,1,"Not enough memory.\n");
mi->first_ml=ml;
mi->first_fml=fml;
mi->first_ms=ms;

/* .
   .
   (statement)
   .
   .
*/


/* Deallocate the mimage, including ml, fml and ms */
mw_delete_mimage(mi);

\end{verbatim}

\newpage %......................................


\Name{mw\_length\_fml\_mimage}{Return the number of morpho lines a morphological image contains}
\Summary{
unsigned int mw\_length\_fml\_mimage(mi)

Mimage mi;
}
\Description
This function returns the number of fmorpho lines contained in the input
\verb+mi+.
It returns $0$ if the structure is empty or undefined.

\Next
\Example
See example page~\pageref{length_ml-example}.

\newpage %......................................


\Name{mw\_length\_ml\_mimage}{Return the number of morpho lines a morphological image contains}
\Summary{
unsigned int mw\_length\_ml\_mimage(mi)

Mimage mi;
}
\Description
This function returns the number of morpho lines contained in the input
\verb+mi+.
It returns $0$ if the structure is empty or undefined.

\Next
\Example
\label{length_ml-example}
\begin{verbatim}


Mimage mi; /* Internal use: no Input neither Output of module */

Morpho_line ml; 
Point_curve pt;
Fmorpho_line fml; 
Point_fcurve fpt;
Morpho_sets mss;
Hsegment seg;
Morpho_set ms;


/* Define a morpho line containing the point (0,0) only */

if (!(pt=mw_new_point_curve()) ||
    !(ml=mw_new_morpho_line())) mwerror(FATAL,1,"Not enough memory.\n");
pt->x=pt->y=0;
ml->first_point=pt;

/* Define a fmorpho line containing the point (0.5,0.5) only */

if (!(fpt=mw_new_point_fcurve()) ||
    !(fml=mw_new_fmorpho_line())) mwerror(FATAL,1,"Not enough memory.\n");
fpt->x=fpt->y=0.5;
fml->first_point=fpt;

/* Define a morpho sets containing one morpho set */

if (!(seg=mw_new_hsegment()) ||
    !(ms=mw_new_morpho_set()) ||
    !(mss=mw_new_morpho_sets())) mwerror(FATAL,1,"Not enough memory.\n");
seg->xstart=0; 
seg->xend=200;
seg->y=10;
ms->first_segment=seg; ms->minvalue=0.0; ms->maxvalue = 1.0; ms->area=201;
mss->morphoset=ms;

/* Define a morphological image made by one morpho line, one fmorpho line and
   one morpho sets.
*/

if (!(mi=mw_new_morpho_line())) mwerror(FATAL,1,"Not enough memory.\n");
mi->first_ml=ml;
mi->first_fml=fml;
mi->first_ms=ms;


/* This will print 1 */
printf("%d",mw_length_ml_mimage(mi));
/* This will print 1 */
printf("%d",mw_length_fml_mimage(mi));
/* This will print 1 */
printf("%d",mw_length_ms_mimage(mi));

\end{verbatim}

\newpage %......................................


\Name{mw\_length\_ms\_mimage}{Return the number of morpho sets a morphological image contains}
\Summary{
unsigned int mw\_length\_ms\_mimage(mi)

Mimage mi;
}
\Description
This function returns the number of morpho sets contained in the input
\verb+mi+.
It returns $0$ if the structure is empty or undefined.

\Next
\Example
See example page~\pageref{length_ml-example}.

\newpage %......................................


\Name{mw\_new\_mimage}{Create a new morphological image}
\Summary{
Mimage mw\_new\_mimage()

}
\Description
This function returns a new \mimage\ structure, or \Null\ if not enough 
memory is available to allocate the structure. 
Your code should check this return value to send an error message in the 
\Null\ case, and do appropriate statement.

The new structure is created with fields set to $0$ or \Null.

\Next
\Example
\begin{verbatim}
/* Copy the morpho lines only of a morphological image into another morphological image */

Mimage in,out;

out=mw_new_mimage();
if (!out) mwerror(FATAL,1,"Not enough memory !\n");
out->nrow = in->nrow;
out->ncol = in->ncol;
out->minvalue=in->minvalue;
out->maxvalue=in->maxvalue;
if (in->firstml) 
  {
    out->firstml=mw_copy_morpho_line(in->firstml, out->firstml);
    if (!out->firstml) mwerror(FATAL, 1,"Not enough memory !\n");
  }
\end{verbatim}

\newpage %......................................



