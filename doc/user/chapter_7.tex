%
% Part 7 of the MegaWave2 Guide #1
%   Documentation
%

\index{documentation}
{\bf Warning : } in MegaWave2 versions 2.x, the \verb+doc+ directory structure has 
changed\index{directory tree!doc}.
Now put the documentation you write (the \verb+.tex+ files) into the 
\verb+$MEGAWAVE2/doc/src/+ directory (administrator) or \verb+$MY_MEGAWAVE2/doc/src/+ 
(plain user) directory, without creating subdirectories for the groups : the \verb+.tex+ 
files of all modules of all groups belong to the same directory.
The \verb+$MEGAWAVE2/doc/obj/+ and  \verb+$MY_MEGAWAVE2/doc/obj/+ directories contain files
that are automatically generated (such as \verb+.doc+ and \verb+.dvi+ files). Do not put
your \verb+.tex+ files here, they may be destroyed by the system !

%+++++++++++++++++++++++++++++++++++++++++++++++
\section{Document a module}
%+++++++++++++++++++++++++++++++++++++++++++++++
\label{document_module}

\index{documentation!module}
Each module you write must be documented.
MegaWave2 helps you as much as possible in this unattractive task:
when you compile a module, a document skeleton is generated in the directory
\verb+$MEGAWAVE2/doc/obj/+ (or \verb+$MY_MEGAWAVE2/doc/obj/+).
It takes the name of the module and the extension \verb+.doc+.
This file documents in the \LaTeX\index{LaTeX} language about everything that could be
automatically done, as the synopsis of the run-time command, the summary of
the module function, the release number and the copyright for the authors and
laboratories.

(Un)Fortunately, there will always be a part of the documentation which cannot
be automatically generated: the mathematical description of the algorithm.
This part is the {\em Description} field of the documentation.
Therefore, the author of a new module must write this content in a file
into the \verb+$MEGAWAVE2/doc/src/+ (administrator only) or 
\verb+$MY_MEGAWAVE2/doc/src/+ (plain user) directory.
Give this file the name of the module with the extension \verb+.tex+.
As the file will be inserted by MegaWave2 into the whole documentation, you must 
write the text using a subset of the \LaTeX\ language.
In particular, don't use any commands about the style or the presentation
(as \verb+\begin{document}+, \verb+\documentstyle+, \verb+\newpage+, \verb+\section+, \ldots).

You can get a lot of examples of such documents by seeking the files into
subdirectories of \verb+$MEGAWAVE2/doc/src+.

Since MegaWave2 V 2.21, the system uses bibliographic\index{bibliography} databases where all citations\index{citation}
are recorded. As this allows cross-references, it is now possible to get all modules associated to
a given article. In order to fulfill the new requirements, citations have to be set
in the \verb+.tex+ file using the standard \verb+\cite+ \LaTeX\ command,
and the corresponding references have to be given in the bibliographic database (.bib file), NO MORE in the \verb+.tex+ file.
Beware, do not collapse citations : use e.g. \verb+\cite{key1}\cite{key2}+ instead of \verb+\cite{key1,key2}+.
References for private modules have to be put in \verb+$MY_MEGAWAVE2/doc/private.bib+\index{private.bib}
while references for public modules are in \verb+$MEGAWAVE2/doc/public.bib+\index{public.bib}.
Of course, only references for private modules not in \verb+public.bib+ have to be in \verb+private.bib+.
See the system macro {\tt mwmodbibtex} for more information about the bibliographic database.


%+++++++++++++++++++++++++++++++++++++++++++++++
\section{Document a macro}
%+++++++++++++++++++++++++++++++++++++++++++++++
\label{document_macro}

\index{documentation!macro}
The documentation of a macro follows the same rules as the ones for modules.
Therefore, for each new user's macro, you must write a description file 
\verb+$MEGAWAVE2/doc/src/MACRO.tex+ (for a public macro, administrator only), or 
\verb+$MY_MEGAWAVE2/doc/src/MACRO.tex+ (for a private macro), \verb+MACRO+ being 
the name of the macro located in a subdirectory of \verb+$MEGAWAVE2/src/+ (or of
\verb+$MY_MEGAWAVE2/src/+).
In this file, you put in \LaTeX\ the part which corresponds to the {\em Description} field of 
the documentation.

The difference between macros and modules is that macros do not need really need a compilation,
and therefore you may call a specific command to generate the document skeleton 
\newline
\verb+$MEGAWAVE2/doc/obj/MACRO.doc+ (or \verb+$MY_MEGAWAVE2/doc/obj/MACRO.doc+),
which is the macro \verb+cmw2macro+ (see section~\ref{sysmacros_summary}).
Since MegaWave2 Versions 2.x, \verb+cmw2+ will also perform this task if its argument
is a user's macro.

%+++++++++++++++++++++++++++++++++++++++++++++++
\section{Print the documentation}
%+++++++++++++++++++++++++++++++++++++++++++++++
\label{document_print}

\index{documentation!print}
Once the document skeleton of a module or macro is generated, and the description file is
written, you can compile the document skeleton by calling the macro \verb+mwdoclatex+
\index{system's macro!mwdoclatex} (see section~\ref{sysmacros_summary}).
Of course this needs to have the \LaTeX\index{LaTeX} environment installed on your system.
You get a \verb+DVI+ file which can be printed on various devices using commands of the standard TeXware distribution.

If you want to print the documentation of all modules and user's macros instead of only one,
use the macro \verb+mwmakedoc+\index{system's macro!mwmakedoc}. 
This will actually create an updated sample of the Volume Three of the MegaWave2 guides (``MegaWave2 User's Modules Library'').

In the document skeleton of each module (not available for macros), a field named 
``See Also'' lists all the modules calling the module or called by the module: 
this is a dependencies list. In order to update this
list, you need to compile each module which the option \verb+-dep+ and you need to
run the macro \verb+mwdep+\index{system's macro!mwdep} before the macros \verb+mwdoclatex+ and \verb+mwmakedoc+.
