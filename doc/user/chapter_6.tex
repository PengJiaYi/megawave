%
% Part 6 of the MegaWave2 Guide #1
%   User's Macros
%

You have been already introduced to the macros by the 
section~\ref{intro_macros}.
An user's macro\index{user's macro} is a Bourne shell\index{Bourne shell} script which uses a normalized header,
and which calls a sequence of commands, such as MegaWave2 modules.

Since system's macros use the header of the user's macro to perform
some operations, as to make the documentation of the macro
(see section~\ref{document_macro}), you have to correctly fill the header 
when you write a new macro.

In order to easily recognize a user's macro from a user's module
(they are located in the same source directory), we use capital letters to 
name macros (at least for the first letter) and lower letters to name modules.
We recommand you to follow this convention.

%+++++++++++++++++++++++++++++++++++++++++++++++
\section{Header of a Macro}
%+++++++++++++++++++++++++++++++++++++++++++++++
\label{usermacros_header}

Each macro must have an header\index{header!macro} which consists of variables to be filled.
Falling to define most of these variables leads malfunctions when calling 
some systems's macros.

In the following is the list of the variables to define
\begin{itemize}
\item \verb+_Prog+: Name of the macro.
\item \verb+_Group+: Group where belongs the macro.
\item \verb+_Func+: Short description of the function performed by the macro.
\item \verb+_Vers+: Version number of the macro.
\item \verb+_Date+: Year of the creation or last change.
\item \verb+_Auth+: Name of the author.
\item \verb+_Usage+: Usage line, following Unix command conventions (do not
mention the program name).
\item \verb+_Labo+: Name and address of the institution, when it is not
the CMLA.
\end{itemize}

You can get examples of such headers by reading the source of some systems's 
macros (which share the same header conventions) or of some public user's
macros given in the distribution.

%+++++++++++++++++++++++++++++++++++++++++++++++
\section{How to use user's macros}
%+++++++++++++++++++++++++++++++++++++++++++++++
\label{usermacros_howto}

An user's macro being a Bourne shell\index{Bourne shell} script, you don't really need to compile 
it before to call it. But in order to be able to call it from any location,
a link has to be created in a directory recorded in the path variable
(which is \verb+$MEGAWAVE2/shell+ if you are the administrator,
\verb+$MY_MEGAWAVE2/shell+ otherwise). In addition to that, the documentation 
must also be generated. Therefore, you should compile a user's macro as
if it was a module, using \verb+cmw2+\index{system's macro!cmw2}.




