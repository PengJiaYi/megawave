%
% Part 2 of the MegaWave2 Guide #1
%   Installation
%

%+++++++++++++++++++++++++++++++++++++++++++++++
\section{Installation of the system}
%+++++++++++++++++++++++++++++++++++++++++++++++
\label{install_system}

\index{installation}
This section explains how to install the MegaWave2 software on your machine, it is written for the MegaWave2 administrator.
If you are a plain user, you can skip this and go to Section~\ref{install_user}.
To be the MegaWave2 administrator\index{administrator}, you don't need to be root\index{root} (super-user)\footnote{It is not a good idea to install MegaWave2 as root. If you have root privileges, you should better create a {\em megawave} account and install the
software using this account.}.
The administrator has to compile the kernel and the modules, and to manage the local public version of the software; in 
addition to the installation, he may update the modules library with some private modules
from the users (see Section~\ref{intro_philo}).


%-----------------------------------------------
\subsection{Upgrading to a new MegaWave2 version}
%-----------------------------------------------
\label{upgrading_megawave2}

If a former MegaWave2 software is installed on your machine, you may upgrade to a new version 
by following the same instructions as for a first installation. Old public modules will be
replaced by new ones. It is likely that you or other users would have developed private 
modules (i.e. modules set in \verb+$MY_MEGAWAVE2/src+) that make use of older system library
and older public modules. In such a case when the new installation is finished,
check if public modules used by private modules have changed in some way (especially in the 
input/output parameters) and modify accordingly private modules. And then, recompile all
private modules (run a \verb+cmw2_all -clear -2p -dep .+ under \verb+$MY_MEGAWAVE2/src+).
All MegaWave2 user's should do such a recompilation on their own private modules, even if the
sources (.c files) didn't require any modification : from one MegaWave2 version to a new one, 
the kernel may change a lot and therefore objects (such as .o and .doc files) generated by the 
old kernel may not be compatible with the new one.

%-----------------------------------------------
\subsection{Restoration of the package}
%-----------------------------------------------
\label{install_system_restoration}

\index{distribution package}
The MegaWave2 administrator has to install the distribution package in a subdirectory of its home directory.
Let us call this subdirectory \verb+$PRIVATE_MEGAWAVE2+.
If needed, the administrator may also create a directory (usually outside its home directory) where
the public version of the software will be copied. 
Let us call this subdirectory \verb+$PUBLIC_MEGAWAVE2+.

You should begin the installation by restoring the full MegaWave2 package in \verb+$PRIVATE_MEGAWAVE2+. 
As you are reading this document, you probably have already completed the restoration (in that
case, you can go to the next Section~\ref{install_system_set-up}).

If not, the best way to download\index{download} the last version of the software and to get
full instructions for the restoration is to 
\htmladdnormallink{visit our {\em World Wide WEB} page}{\httpmw}\footnote{\httpmw} on the Internet.
The package should be in a file named \verb+MegaWave2_V<n>.tar.Z+,  
\verb+MegaWave2_V<n>.tar.gz+ or \verb+MegaWave2_V<n>.tgz+, where \verb+<n>+ is the version number.
After downloading it and if your WWW browser didn't do it for you, decompress the tar file using 
\verb+uncompress+ (\verb+.Z+ file) or using \verb+gunzip+ (\verb+.gz+ and \verb+.tgz+ files), and extract the
files from the tar file using \verb+tar xf MegaWave2_V<n>.tar+.
If you are using the GNU tar command, you may also handle restoration of \verb+MegaWave2_V<n>.tgz+
in one pass by means of the ``z'' flag : \verb+tar xfz MegaWave2_V<n>.tgz+.

In any case, first load the file named README and follow the instructions in it. 
When the restoration will be done, you will get a main directory containing the software,
with a name referred to as \verb+$PRIVATE_MEGAWAVE2+.

Please read first the additional information put in the file \verb+$PRIVATE_MEGAWAVE2/README+.

If you are the administrator, you should entirely read the following sections. However, you may try to
install the software quickly by calling the shell \verb+Install+\index{Install} located at the root directory
of the MegaWave2 package \verb+$PRIVATE_MEGAWAVE2+. If your system and machine architecture is the same
as one we have, this may install the whole software without pain.

%-----------------------------------------------
\subsection{Directories structure}
%-----------------------------------------------
\label{install_directories_struc}

\index{directories structure}
We know from Section~\ref{intro_philo} that modules are split between {\em public modules}\index{public modules} put in a 
directory named \verb+$MEGAWAVE2+ and {\em private modules}\index{private modules} put in \verb+$MY_MEGAWAVE2+. 

Actually, for the MegaWave2 administrator things are a little bit more complicated : the administrator may
want to be able to check new public modules and new kernel binaries before making them available for plain users. 
Therefore, the administrator has his own \verb+$MEGAWAVE2+ directory wich is not necessary the same as the
\verb+$MEGAWAVE2+ directory of plain users.
His own \verb+$MEGAWAVE2+ directory is called {\em private MegaWave2}\index{private MegaWave2} 
(with associated environment variable \verb+$PRIVATE_MEGAWAVE2+\index{environment variable!PRIVATE\_MEGAWAVE2}) whereas the 
\verb+$MEGAWAVE2+ directory of plain users is called {\em public MegaWave2}\index{public MegaWave2} (\verb+$PUBLIC_MEGAWAVE2+
\index{environment variable!PUBLIC\_MEGAWAVE2}).
Of course, if you don't plan to update the system or if the only user is the administrator you should set 
\verb+$PUBLIC_MEGAWAVE2+ $=$ \verb+$PRIVATE_MEGAWAVE2+.
Notice that the administrator always has \verb+$PRIVATE_MEGAWAVE2+ $=$ \verb+$MEGAWAVE2+.

% ----------
\setlength{\unitlength}{1mm}
\begin{figure}
\caption{The various MegaWave2 directories a user may see.}
\label{install_directories_fig}

\begin{picture}(150,90)(0,0)

\put(0,70){A plain user}
\put(50,80){{\tt \$MY\_MEGAWAVE2} (read/write) : user's private modules}
\put(50,60){{\tt \$MEGAWAVE2} (read only)}
\put(0,40){The administrator}
\put(60,55){{\Large $=$}}
\put(50,50){{\tt \$PUBLIC\_MEGAWAVE2} (read/write)}
\put(50,30){{\tt \$PRIVATE\_MEGAWAVE2 =  \$MEGAWAVE2}  (r/w) : distribution,}
\put(75,26){future public modules, kernel binaries and sources}
\put(50,10){{\tt \$MY\_MEGAWAVE2} (r/w) : modules in development}

\put(80,55){current public modules and kernel binaries}

\put(30,71){\vector(2,1){15}}
\put(30,71){\vector(2,-1){15}}

\put(35,41){\vector(2,1){13}}
\put(35,41){\vector(2,-1){13}}
\put(35,41){\vector(1,-2){14}}
\end{picture}
\end{figure}

% ----------



%-----------------------------------------------
\subsection{Set up the environment}
%-----------------------------------------------
\label{install_system_set-up}

You have to set up some environment variables\index{environment variable} used by MegaWave2.
Put these definitions in a file which is executed at login time (such as \verb+.profile+,\verb+.login+,\ldots) or when opening a new shell (e.g. \verb+.cshrc+).
They have to be set both for the administrator account and for all user accounts.
Therefore, it may be better for the administrator to put the definitions in a file which will be sourced at login time by all users.

The system's macro \verb+mwsetenv+\index{system's macro!mwsetenv} helps you to generate such a file, 
see its description in Section~\ref{sysmacros_summary}. 
This macro is called by the shell \verb+Install+\index{Install} (see 
Sections~\ref{install_system_make-ready} and~\ref{sysmacros_summary}), so if you choose to install the software
in this way you should not have to set up the environment manually.

The following variables are needed {\em for the administrator only}:
\begin{itemize}
\item \verb+PUBLIC_MEGAWAVE2+\index{environment variable!PUBLIC\_MEGAWAVE2} : 
directory where the public installation of the software has to be made, 
usually outside the administrator home directory.
You need write permission on it. Plain users need read and execute permissions, but they should not have write 
permission.\\
\nopagebreak
Example: \verb+setenv PUBLIC_MEGAWAVE2 /usr/local/share/megawave2+.
\item \verb+PRIVATE_MEGAWAVE2+\index{environment variable!PRIVATE\_MEGAWAVE2} : 
directory where the temporary installation of the software has to be made
(future public version), usually inside the administrator home directory. This is also the directory where
the original distribution is put. Plain users do not need any permission on it. \\
\nopagebreak
Example: \verb+setenv PRIVATE_MEGAWAVE2 ${home}/megawave2+.
\end{itemize}

The following variables are needed for all users (including the administrator):
\begin{itemize}
\item \verb+MEGAWAVE2+\index{environment variable!MEGAWAVE2} : 
for a plain user, directory where the public installation of the software is made (ask your
administrator for it). For the MegaWave2 administrator, set it to \verb+PRIVATE_MEGAWAVE2+.\\
Example: \verb+setenv MEGAWAVE2 /usr/local/share/megawave2+.
\item \verb+MY_MEGAWAVE2+\index{environment variable!MY\_MEGAWAVE2} : 
directory where the private user's version of the modules is. \\
Example: \verb+setenv MY_MEGAWAVE2 ${home}/my_megawave2+.
\item \verb+MW_MACHINETYPE+\index{environment variable!MW\_MACHINETYPE} : machine architecture as returned by the macro \verb+mwarch+\index{system's macro!mwarch} (if the
version of this macro accepts the option \verb+-s+, use it). \\
Example: \verb+setenv MW_MACHINETYPE `$MEGAWAVE2/sys/shell/mwarch -s`+.
On Sun computers running Solaris (SunOS 5.x or higher), you need to call \verb+mwarch+ with
the option \verb+-s+. Otherwise, MegaWave2 will confuse the objects with those for SunOS 4.x.
\item \verb+MW_SYSTEMTYPE+ : name of the operating system as HPUX, SunOS, \ldots. \\
Example: \verb+setenv MW_SYSTEMTYPE `uname | tr -d -`+.
\end{itemize}
Special attention is requested if the directory path put in \verb+MEGAWAVE2+
or \verb+MY_MEGAWAVE2+ corresponds to a link or to an automounted file: you
should always put the true pathname and not the link name. In short, if
you type under your shell the command \verb+cd $MEGAWAVE2+ followed by
the command \verb+/bin/pwd+ (and not the shell built-in command \verb+pwd+),
 you must get the same pathname as the one put in \verb+$MEGAWAVE2+.

The following variables may be needed:
\begin{itemize}
\item \verb+MW_INCLUDEX11+\index{environment variable!MW\_INCLUDEX11} : directory where the X Window include files are, if not in \verb+/usr/include/X11+. If you are the administrator, this variable is always required. \\
Example: \verb+setenv MW_INCLUDEX11 /usr/include/X11R5+
\item \verb+MW_LIBX11+\index{environment variable!MW\_LIBX11}  : directory where the X Window libraries are, if not in \verb+/usr/lib/X11+. 
If you are the administrator, this variable is always required.\\
Example: \verb+setenv MW_LIBX11 /usr/lib/X11R5+.
\item \verb+MW_INCLUDEXm+\index{environment variable!MW\_INCLUDEXm}  : XMegaWave2 only. 
Directory where the Motif include files are, if not in \verb+/usr/include/Xm+. \\
Example: \verb+setenv MW_INCLUDEXm /usr/include/Motif1.2+.
\item \verb+MW_LIBXm+\index{environment variable!MW\_LIBXm}  : XMegaWave2 only. 
Directory where the Motif libraries are, if not in \verb+/usr/lib+.  \\
Example: \verb+setenv MW_LIBXm /usr/lib/Motif1.2+.
\item \verb+MW_LIBTIFF+\index{environment variable!MW\_LIBTIFF}  : directory path where the TIFF\index{TIFF library} library 
(libtiff\index{libtiff|see{TIFF library}}) is located, if you want to use TIFF image format.
\item \verb+MW_LIBJPEG+\index{environment variable!MW\_LIBJPEG}  : directory path where the JPEG\index{JPEG library} library (libjpeg\index{libjpeg|see{JPEG library}}) is located, if you want to use JPEG image format.
\item \verb+LD_LIBRARY_PATH+\index{environment variable!LD\_LIBRARY\_PATH}  : add directory path where the MegaWave2 system shared libraries are located, on UNIX 
systems (e.g. IRIX, SUN SOLARIS, LINUX) which require this definition for their runtime linker.  Example :\\
\verb+setenv LD_LIBRARY_PATH ${LD_LIBRARY_PATH}:${MEGAWAVE2}/sys/lib/${MW_MACHINETYPE}+.
\item \verb+LD_RUN_PATH+\index{environment variable!LD\_RUN\_PATH} : 
add directory path where the MegaWave2 system shared libraries are located, on UNIX 
systems (e.g. LINUX) which require this definition for their runtime linker.  Example :\\
\verb+setenv LD_RUN_PATH ${LD_RUN_PATH}:${MEGAWAVE2}/sys/lib/${MW_MACHINETYPE}+.
\end{itemize}

Note:
\begin{itemize}
\item The X Window system Version 11 is a priori needed only by modules which use the Wdevice library to display signals, images, movies, \ldots However, it cannot be discarded since it is required to compile the kernel.
\item The Motif Window system is needed only for XMegaWave2\index{XMegaWave2}; 
in that case you need also the X Window system Version 11.
\item \LaTeX\index{LaTeX} is needed only to make a new documentation for the modules you are going to write.
But to read the current documentation, you need a DVI viewer. MegaWave2 assumes \verb+xdvi+
\index{xdvi} (DVI Previewer for the X Window System) is installed on your system.
\item The TIFF library is needed to load and save images in the TIFF format only.
You may already have this library, e.g. if you use the \verb+XV+ software\index{xv} of John Bradley.
If not, you can load it by anonymous ftp at the following Internet address:
\verb+sgi.com (directory graphics/tiff)+.
\item The JPEG library is needed to load and save images in the JPEG/JFIF format, as defined by the 
Independent JPEG Group. Be aware that this is a loosely image format.
\end{itemize}

Last, you need to update your path variable in order to allow execution of the MegaWave2 commands from any location. 
Add the following paths (the first one should have greatest priority): 
\verb+${MY_MEGAWAVE2}/shell+, \verb+${MY_MEGAWAVE2}/bin/${MW_MACHINETYPE}+, \\
\verb+${MEGAWAVE2}/sys/bin/${MW_MACHINETYPE}+, \verb+${MEGAWAVE2}/sys/shell+,\\ \verb+${MEGAWAVE2}/bin/${MW_MACHINETYPE}+.

Example: 
\begin{tabbing}
\verb+set path=(${MEGAWAVE2}/sys/bin/${MW_MACHINETYPE} ${MEGAWAVE2}/sys/shell+ \ldots \\
\verb+          ${MEGAWAVE2}/bin/${MW_MACHINETYPE} $path)+ \\
\verb+set path=(${MY_MEGAWAVE2}/shell ${MY_MEGAWAVE2}/bin/${MW_MACHINETYPE} $path)+
\end{tabbing}


%-----------------------------------------------
\subsection{Make the software ready}
%-----------------------------------------------
\label{install_system_make-ready}

The simplest way to install the software is to call the shell \verb+Install+\index{Install} located at the root directory
of the MegaWave2 package \verb+$PRIVATE_MEGAWAVE2+, and to answer some questions. 
Actually, this shell simply calls the macro \verb+mwinstall+\index{system's macro!mwinstall} 
with parameters depending on your answers (see the description of this system macro in 
Section~\ref{sysmacros_summary}).
If \verb+mwinstall+ successfully exits, you should not have to do anything else to install the software
but to include the environment file generated by \verb+mwsetenv+ in your \verb+.profile+ or \verb+.cshrc+.

If your system is the same as one we have, the installation procedure should be straightforward.
If not or if something goes wrong during the installation procedure, you will have to compile the whole thing
manually. The following explains the main steps you should manually complete. You may also have a look on
the installation shells, such as \verb+Install+ and \verb+mwinstall+, for a better understanding of what has
to be done. Once the environment variables are set (see Section~\ref{install_system_set-up}), you should try to compile
the kernel first, and the modules and user's macros afterwards. 

The kernel\index{kernel}, that is to say the system functions and the MegaWave2 preprocessor, is no longer pre-compiled on
MegaWave2 versions 2.x. You may compile it for your system architecture using the shell 
\verb+$PRIVATE_MEGAWAVE2/kernel/Install+.

The software contains the modules as source files (located in the directory \verb+$MEGAWAVE2/src+), so they have to be 
compiled for your machine architecture.
Just type \verb+cmw2_all $MEGAWAVE2/src+ to compile all the modules located in the subdirectories
of \verb+$MEGAWAVE2/src+.
If your system can run the compiler, you probably won't encounter a lot of errors at this time.
Most likely errors are those about standard libraries or include files not found: check the environment setup or the installation of your operating system.
If you get an error about an unsatisfied symbol while compiling a module \verb+A+, it is probably because \verb+A+ calls a module \verb+B+ which has not been already compiled. 
To fix that, you may type \verb+cmw2_all $MEGAWAVE2/src+ one more time or, preferably, 
use the option \verb+-2p+ (two pass - does not work for all linkers, see 
Section~\ref{sysmacros}).

%-----------------------------------------------------
\subsection{No longer registration, but still license}
%-----------------------------------------------------
\label{install_system_registration}

\index{registration}
When the modules are compiled, you may want to check some algorithms. 
Just type from your Shell the name of the module you want to run: MegaWave2
recalls you the parameters needed to execute the corresponding algorithm.
Please refer to the Volume three: ``MegaWave2 User's Modules Library'' to learn more about the different modules.
You may also run the system's macro \verb+mwdoc M+\index{system's macro!mwdoc} to get the list of available modules
and user's macros, together with a short description.

On MegaWave2 versions 1.x you had noticed that some input and output data were disturbed. 
This is because you had to register the copy of MegaWave2 you got in order to use it freely.
This is no longer the case on MegaWave2 version 2.x.

However, the use of MegaWave2 is still under a license\index{license}. By using it, you accept the terms of this
license. 
You will find the license text at Section~\ref{annex_license} page~\pageref{annex_license}.
If you do not agree with these conditions, you have to remove all the MegaWave2 files in your possession. 
Otherwise, your copy may be declared illegal regarding the european laws.

%+++++++++++++++++++++++++++++++++++++++++++++++
\section{Installation for the user}
%+++++++++++++++++++++++++++++++++++++++++++++++
\label{install_user}

\index{installation}
This section explains what each user must do in order to run the software.
It includes the MegaWave2 administrator which may also want to use the software as
a plain user.

If you were using an older MegaWave2 software, you will have to recompile all
private modules after the new software will be installed : check if some of your
private modules need to be upgraded and run a \verb+cmw2_all -clear -2p -dep .+ 
under \verb+$MY_MEGAWAVE2/src+. You have to redo all compilations even if your
private modules didn't require any modification : from one MegaWave2 version to a 
new one, the kernel may change a lot and therefore objects (such as .o and .doc files) 
generated by the old kernel may not be compatible with the new one.

%-----------------------------------------------
\subsection{Set up the environment}
%-----------------------------------------------
\label{install_user_set-up}

The user may have to set up some variables of the environment.
Two cases may be encountered: if the MegaWave2 administrator has put the definitions in a specific file
(this is automatically done if he has used the standard installation procedure), you just have to load this file at 
login time or when opening a new shell (by using the shell command \verb+source+ or the \verb+.+ - dot -).
Ask the local administrator about this possibility.
If there is no such a file, you have to set the variables in your own configuration file as explained at Section~\ref{install_system_set-up} page~\pageref{install_system_set-up}.

In addition to the standard environment setup, you may want to define the
following variables 
\begin{itemize}
\item \verb+MW_STDOUT+\index{environment variable!MW\_STDOUT}, \verb+MW_STDERR+\index{environment variable!MW\_STDERR} \\
If you type under your C-compatible shell \verb+setenv MW_STDOUT /dev/null+,
you ask the system to redirect the standard output of all modules to 
the null device that is, to the trash.
Make the experience by calling a run-time module which usually prints a 
lot of messages... No more messages will disturb your terminal !
But you have lost the messages. If you want to get them in a file 
(let us call \verb+output+ its name), type \verb+setenv MW_STDOUT output+.
You can reset the default prints by removing your definition: \verb+unsetenv MW_STDOUT+.
You can also redirect the standard error by setting the variable 
\verb+MW_STDERR+. We do not recommend you to redirect this output to the
trash, since you won't be able to know if your modules correctly exit.
\item \verb+MW_CHECK_HIDDEN+\index{environment variable!MW\_CHECK\_HIDDEN} \\
If this variable is set, 
each time a module is called a check is performed using the system's macro \verb+mwwhere+ to see if the 
called module does not hide another one of same name. Indeed, path for private modules is set
before path for public modules in the \verb+PATH+ variable (and the same rule applies for module's library at
link time). So, if you write a user's module called \verb+foo+ while a \verb+foo+ public module already exists, 
the public module will be hidden by the first one. In such a case and providing \verb+MW_CHECK_HIDDEN+ is
set, a warning message will be issued by the non-hidden module at running-time.
The same procedure applies to macro, to check if a private user's macro hides a public one.
However, for macros the check is performed when the header/usage text is output only (e.g. when calling
the macro with -help or with an invalid syntax).
\end{itemize}


%-----------------------------------------------
\subsection{Make the directory tree ready}
%-----------------------------------------------
\label{install_user_make-dir}

\index{directory tree}
An user may want to develop new modules. 
In this case, he writes its private modules into a local MegaWave2 directory called \verb+$MY_MEGAWAVE2+.
This directory should contain several sub-directories although some of them are optional:
\begin{itemize}
\item \verb+src+\index{directory tree!src} : sources of the user's modules and macros. 
May be divided into subdirectories representing groups.
\item \verb+bin+\index{directory tree!bin}  : executable modules. You have one subdirectory per machine architecture on which you have compiled the modules.
\item \verb+lib+\index{directory tree!lib}  : user's library of modules. You have one subdirectory per machine architecture on which you have compiled the modules.
\item \verb+obj+\index{directory tree!obj}  : object files.  You have one subdirectory per machine architecture on which you have compiled the modules.
\item \verb+doc+\index{directory tree!doc}  : documentation. May be divided into \verb+src+ (source) and \verb+obj+ (object) subdirectories.
\item \verb+data+\index{directory tree!data}  : samples of MegaWave2 data files such as images, movies, filters, 
shapes. May be divided into subdirectories representing groups or whatever else. 
\item \verb+shell+\index{directory tree!shell}  : links to the user's macros (Bourne shell scripts). 
\item \verb+mwi+\index{directory tree!mwi}  : usage interface for the interpreter (XMegaWave2).
\item \verb+tmp+\index{directory tree!tmp} : directory for temporary files.
\end{itemize}

You should only focus on the directories \verb+src+, \verb+doc/src+ and maybe \verb+data+ (the last one being optional), where you will have to write something.
Other directories will be automatically updated by the system.

Subdirectories of these directories will be automatically created when necessary (except groups into \verb+src+, see Section~\ref{intro_philo} page~\pageref{intro_philo}).
But the directories themselves have to be created before the first call to the MegaWave2 compiler.
You can make them manually, or you can call the macro \verb+mwnewuser+\index{system's macro!mwnewuser}.









