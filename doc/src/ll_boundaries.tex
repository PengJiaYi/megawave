This modules computes all the level lines (that is, the connected
components of the boundaries of the level sets) of an image 
(with the same options \verb+-s+, \verb+-p+, \verb+-z+ and \verb+-t+ 
as the \verb+ll_extract+ module) and keeps the most contrasted ones.
The detection is made using a thresholding function balancing between 
the minimum contrast (gradient norm) along the curve and the
length of the curve. Unless \verb+-a+ option is specified, only
on representant (the most meaningful one) is kept for each 
monotone sequence of meaningful level lines (that is, a sequence
of level lines included in each others, with no inclusion
branching and no contrast reversal).
The thresholding function ensures
that at most $10^{-eps}$ false detections could occur by chance
with independent random gradient values (see [1]).

\medskip

Notice that contrary to the \verb+ll_edges+ detection, this 
module is constrained to keep full (ie closed) level lines and 
cannot break them into parts.

\medskip

The result is a collection of curves stored in a Flists structure.

\bigskip

{\large \bf References}

\medskip

[1] A. Desolneux, L. Moisan, J.-M. Morel, ``Edge Detection by Helmholtz 
Principle'', {\it Journal of Mathematical Imaging and Vision}, 
vol 14:3, pp 271-284, 2001
