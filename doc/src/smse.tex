{\em smse} computes the maximal relative difference, mean square error, signal to noise ratio, and peak signal to noise ratio between two univariate digitized signals whose sample values are read in files {\em Signal1} and {\em Signal2}. 

The maximal relative difference is given by 
\[
MRD = 100 \times \frac{\max_{x}|f_{1}(x) - f_{2}(x)|}{\max_{x,i=1,2}f_{i}(x) - \min_{x,i=1,2}f_{i}(x)}
\]
where $f_{i}(x)$ is the $x^{st}$ sample value in signal $i$, $i=1,2$. 

The mean square error is defined by 
\[
MSE = \frac{\sum_{x}|f_{1}(x) - f_{2}(x)|^{2}}{n}
\]
where $n$ is the number of samples in {\em Signal1} and {\em Signal2}.

In maximal relative difference and mean square difference the two signals have symetric roles. This is no longer the case in (peak) signal to noise ratio. 
These quantities can be calculated for any two signals, but are more meaningful if $f_{1}$ is considered as an original signal and $f_{2}$ as an approximation to it. The signal to noise ratio and peak signal to noise ratio are defined as follows
\[
SNR = 10 \times \log_{10} \frac{\sigma_{1}^{2}}{MSE} 
\]
\[
PSNR = 10 \times \log_{10} \frac{(\max_{x} f_{1}(x) - \min_{x} f_{1}(x))^{2}}{MSE}
\]
where $\sigma_{1}^{2}$ is the empirical variance of the original signal.

The -n option specifies that the two signals are normalized to $0.0$ mean and $1.0$ variance before the computation of the quantities listed above.





