This module computes the bit rate needed to encode
all points of the curves contained in the input set.
The bit rate is given in bit per point, the number of points being
the total number of points contained in all curves.
The curves are supposed to be composed by connected points
for the 4-connectivity. To encode curves with holes, rather use
the module \verb+cvscompress+.

This module first calls \verb+cvsorgcode+ to get the 
bit rate to code the origin points, and after that
\verb+cvsfreecode+ to get the bit rate to code the other points
by the Freeman algorithm.

The option \verb+-s+ avoids to code large parts of multiple points,
by splitting curves into pieces of non-overlapping curves.
The length of the permitted overlapping is given by the parameter
\verb+L+. 
Be aware that by splitting curves, the amount of origin points,
hard to code, grows.
Do not use this option if it is non-sense to modify your curves 
structure.

The option \verb+-o+ is meaningful together with \verb+-s+ : it
allows to output the curves which were really coded.

The function outputs also the total number of points \verb+N+
and the number of bits \verb+B+ needed to code the curves.
