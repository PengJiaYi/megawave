This module generates one or several training sets of rectangular blocks 
of pixels which are extracted from one or several floating point images. 
The input images are read in {\em Image}, {\em Image2}, ..., {\em Image8} 
files. The resulting training set(s) is (are) stored in {\em TrainingSet}, 
({\em TrainingSet2}, ..., {\em TrainingSet2}) file(s). 

The -w and -h options can be used to set the width and height of rectangular 
blocks extracted from the image(s). {\em VectorWidth} and {\em VectorHeight}
are strictly positive integers. 

\begin{figure}[hbt]
\vspace{0.4cm}
\parbox[t]{6.5cm}{
%\epsfxsize=6.5cm \epsfbox{figures/vectshape1.eps}
%\insertps{figures/vectshape1.eps}{4.5cm}{1cm}
%\centerline{\psfig{figure=figures/vectshape1.eps,height=3in,width=4.5in}}
}
\hspace{0.55cm}
\parbox[t]{6.5cm}{
%\insertps{figures/vectshape2.ps}{4.5cm}{1cm}
%\centerline{\psfig{figure=figures/boelinft16384.ps,height=3in,width=4.5in}}
}
\caption[overlap]{On the left : blocks extracted from image are not overlapping. On the right : blocks extracted from image are overlapping.
\label{fig:overlap}
}
\end{figure}

When the -l option  is not set, extracted blocks are not overlapping 
(as in the left part of figure~\ref{fig:overlap}). On the contrary, 
when this option is selected, extracted blocks are overlapping 
(as in the right part of figure~\ref{fig:overlap}). This option 
thus enables to extract a larger number of vectors from a single image 
(this number is multiplied roughly by 
{\em VectorWidth $\times$ VectorHeight}). 

\begin{figure}[hbt]
\vspace{0.4cm}
\parbox[t]{6.5cm}{
%\epsfxsize=6.5cm \epsfbox{figures/vectshape3.eps}
%\insertps{figures/vectshape3.eps}{4.5cm}{1cm}
%\centerline{\psfig{figure=figures/vectshape3.eps,height=3in,width=4.5in}}
}
\caption[decim]{Blocks are not constructed with contiguous pixels. 
Here $Decim = 2$.
\label{fig:decim}
}
\end{figure}

The -d option enables to construct vectors with non contiguous pixels. 
Namely, the vertical and horizontal distance between the pixels is 
{\em Decim} (See figure~\ref{fig:decim}). {\em Decim} is a strictly positive 
number. When equal to 1 (default value), vectors are constructed 
with contiguous pixels. This option is especially useful when 
the input images are sub-images of wavelet transforms generated with 
{\em dywave2} and {\em dyowave2}.  

The -e option is effective only when the -l option has also been selected. 
It indicates that extracted blocks should not be overlapping if they 
are located at a distance less than {\em Edge} from the edges of the 
image. {\em Edge} is a positive integer. This option is especially useful when 
the input images are sub-images of wavelet transforms which has been 
computed with special edge processing methods (like the one in 
{\em owave2} module).

The -t, -u and -v options enable to sort the extracted blocks according 
to their energy and put them in different training sets according 
to their energy value. The energy of a $K$-dimensional vector 
$X=\{X_k\}_{1\leq k\leq K}$ is simply defined by
\[
E(X) = \frac{1}{K} \sum_{k=1}^K |X_k|^2.
\]
Given a set of thresholds $T_1>\ldots >T_S>0$ (with $S\leq 3$), 
the blocks extracted from the image(s) are sorted by finding the index $s$ 
such that 
\[
T_{s-1}> E(X) \geq T_s
\]
(with $T_0 = +\infty$ and $T_{S+1} = 0$). One thus obtains $S+1$ 
(sub-) training sets which are stored respectively in the files 
{\em TrainingSet}, {\em TrainingSet2}, ...
The thresholds $T_1, \ldots$ depend on the mean energy $E_{m}$ of all 
the extracted vectors and on the input values {\em ThresVal1}, 
{\em ThresVal2}, and {\em ThresVal3} :
\begin{eqnarray*} 
T_1 & = ThresVal1 * E_m \;\;\;\;\;\;\; \mbox{if the -t option is selected} \\
T_2 & = ThresVal2 * E_m \;\;\;\;\;\;\; \mbox{if the -u option is selected} \\
T_3 & = ThresVal3 * E_m \;\;\;\;\;\;\; \mbox{if the -v option is selected} 
\end{eqnarray*}
These options are especially useful when one wants to make training sets 
for the creation of codebooks for classified vector quantization (see 
{\em flbg\_adap} and {\em fvq} modules. {\em ThresVal1}, {\em ThresVal2}, 
and {\em ThresVal3} are floating point values. They should range 
between 0.5 and 5.0 in order to get reasonable sizes of training sets. 
If it happens that $T_1$ is too large so that the first training set is 
empty, then the module automatically decreases the value of {\em ThresVal1}
so that the first training set becomes non empty. The same is done 
with $T_2$ and $T_3$. Notice that if for example the -t option is selected 
while neither the -m, -n and -o options are selected, then 
the {\em mk\_trainset} module will only put in {\em TrainingSet} file 
all the extracted blocks whose energy is larger then $ThresVal1 * E_m$. 

The -f option enables to ensure that the size of the resulting training sets 
are big enough when compared to the size of the codebooks which will 
be created based on these training sets (see {\em flbg} and {\em flbg\_adap} 
modules). {\em SizeCB} is a positive integer. If the -f option 
is selected, then the module will try ensure that the size of 
all the generated training sets is larger than $10\times SizeCB$ 
(of course this is not possible if for example the total number of extracted 
blocks is less that $10\times SizeCB$). 
