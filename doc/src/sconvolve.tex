
{\em sconvolve} computes the convolution of a fsignal with a discrete filter which is also of fsignal type. The signal can eventually be interpolated or over-sampled (adding $M-1$ zero-valued samples betwen each successive samples) before the convolution, and the result can eventually be decimated or sub-sampled (discarding $N-1$ samples over $N$).

Let $\{x_{k}\}$ and $\{y_{k}\}$ be the sample values of the input and output signal, and $\{h_{k}\}$ the impulse response of the filter, then :
\[
y_{k} = \sum_{n} h_{Nk-Ml} x_{l}
\]

Because the signal is of finite size, one has to use special tricks to compute the edge (i.e. the first and last) coefficients. These are the following :
\begin{itemize}
\item
One can extend the signal with \( 0 \)-valued samples.
\item
One can periodize the signal.
\item
One can also reflect the signal around each edge.
\item
Finally one can use special filters to compute the coefficients near edges. For example one can use filters that correspond to wavelet basis adapted to the interval (see {\em owave1} module's documentation). 
\end{itemize}

Notice that if $n$ is the size of the input signal, then the size of the output signal is $nM/N$, unless the -p option is selected.

The coefficients \( h_{k} \) of the filter's impulse response are stored in the fsignal type file {\em ImpulseResponse}. The coefficients of the filter's impulse response for computing the edge coefficients are stored in the fimage type file {\em EdgeIR}.

The sample values of the input signal are read in the fsignal type file {\em Signal}, the sample values of the output signal are stored in the file {\em FicConvol}.

\begin{itemize}
\item
The -d option specifies the rate $N$ of sub-sampling after the convolution.
\item
The -i option specifies the rate $M$ of over-sampling before convolution.
\item
The -e option specifies the edge processing mode.
\begin{itemize}
\item
0 : 0 extension.
\item
1 : periodization.
\item
2 : reflexion.
\item
3 : adapted filtering (default).
\end{itemize}
\item
The -p option specifies the extension or shrinkage of the output signal near edges. It works only if the edge processing mode is 0 or 1, and enables then to get exact reconstruciton in wavelet decompositions (see {\em owave1}).
\begin{itemize}
\item
0 : no extension nor shrinkage (default).
\item
1 : prolongation. All the coefficients $y_{k}$ which are not always nul when the 0 extension is selected are kept.
\item
2 : shrinkage, the size of output is $(input.lastp \, - \, input.firstp + 1)uprate/domwrate$.
\end{itemize}
\item
The -b option specifies the high-pass filtering, i.e. the convolution is performed with $\{ g_{k} = (-1)^{k+1} h_{1-k} \}$ rather than $\{ h_{k} \}$.
\item
The -r option specifies that the convolution is performed with reflected filter $\{ h_{-k} \}$ (resp. $\{ g_{-k} \}$) rather than $\{ h_{k} \}$ (resp. $\{ g_{k} \}$).
\end{itemize}
