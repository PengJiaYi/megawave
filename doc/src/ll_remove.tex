This module copies the level lines recorded in the 
input mimage to the output mimage, but the ones with length (number of pixels)
less than a given threshold. 
The length threshold ($10$ by default) can be changed using the option
\verb+-l+.

The morpho line \verb+first_ml+ field of the input mimage must contain a
chain of level lines only that is, the boundaries of the connex components 
of the level sets 
$K_i=\{ (x,y)\,/\, g(x,y)\geq v_i\}$, for $i=1,\ldots,N$.
or of the level sets
$L_i=\{ (x,y)\,/\, g(x,y)\leq v_i\}$,
$g$ being the gray levels image, $N$ the number of gray levels in $g$
and $(v_i)_i$ the sequence of gray levels sorted in increasing order.
Such mimage can be computed from a bit-mapped image using the module
\verb+ll_decompose+.

The operation of removing the smallest level lines can be interpreted, because
of the level lines inclusion principle, as a thresholding of extrema gray
levels values in small areas.
When the input mimage contains level lines associated to $K_i$, the extrema
gray levels are maxima; when the level lines are associated to $L_i$, they
are minima.

You may get some applications of this filter in~\cite{froment:perceptible}
and \cite{froment:functional}\index{filter!grain}.
It is a good example of a very simple use of the modules \verb+ml_decompose+ and
\verb+ml_reconstruct+. However, since the Fast Level Sets 
Transform~\cite{monasse.guichard:fast} (\verb+flst+) has been implemented, 
you would get faster computation by using flst-based modules 
(such as \verb+fgrain+, which considers the area instead of the length).