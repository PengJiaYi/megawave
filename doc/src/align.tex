This module extracts ``meaningful alignments'' from an image.
An alignment is detected when a discrete segment of the image
contains a relatively high number of points for which the
direction orthogonal to the gradient (that is, the direction 
of the level line passing through the point) is equal to the direction
of the segment modulo a given precision $p$. The detection is made
using a thresholding function balancing between the density of
aligned points and the length of the segment. This function ensures
that at most $10^{-eps}$ false detections could occur by chance
with random gradient directions (see~\cite{desolneux.moisan.ea:meaningful}).

\medskip

The result is a 5-Flist, that has 0 element if no alignment was detected.
The \verb+-c+ option store the detected segments as a collection of curves
(Flists), allowing direct visualization with the \verb+fkview+ module.
