%\newcommand{\Disp}[1]{{\displaystyle #1}}
\def\Disp#1{{\displaystyle #1}}
% Displaystyle en mode ligne
%\newcommand{\Dfrac}[2]{{\displaystyle \frac{#1}{#2}}}
\def\Dfrac#1#2{{\displaystyle \frac{#1}{#2}}}
% Donne des fractions en 'display' en mode texte


The command {\bf segct} generates, starting from {\em fimage}, 
a 2-normal segmentation of this picture as will be described below.\\
The initial picture is broken in a grid of elementary cells, composed of
$D\times D$-pixel squares ($D$={\em size\_of\_grid}). 
We use a classical region-growing algorithm to achieve the partition of the 
image in homogeneous regions.

The criterion which measures the resemblance of regions is given by the 
Mumford and Shah model. Call $g$ the picture defined on an open rectangle $R$,
$u$ a piecewise constant function, which aim is to approximate $g$, and
let $B$ be the boundaries between the regions, i.e. the set of jump points of
$u$. With Mumford and Shah we introduce the following functional which has
to be minimized
$$ {\rm E}(B)=\int_{R} (u-g)^{2} + \lambda\ell(B),$$
where $\ell$ is the 1-dimensional Hausdorff measure and $\lambda$ a real
scaling coefficient. The parameter $\lambda$ gives a weight to the length
of the boundaries: a small value allows a lot of boundaries whereas a big
value tends to reduce the boundary length $\ell(B)$.\\
The property of being 2-normal for a segmentation of $R$ in regions
$\Disp{\bigcup_{i} O_{i}}$ has been used by Pavlidis in the seventies. 
We say that $B$ is 2-normal if, given any two regions $O_{i},O_{j}$ having
common boundary $\partial(O_{i},O_{j})$, the following inequality for
the energy holds 
$$ {\rm E}(B) \leq {\rm E}(B\setminus\partial(O_{i},O_{j})). $$
Which yields
$$ \lambda\ell(\partial(O_{i},O_{j}))\leq\frac{|O_{i}|\cdot |O_{j}|}{|O_{i}|+|O_{j}|}\| u_{i}-u_{j}\| ^{2},$$
where $|.|$ denotes the surface measure and $u_{i},u_{j}$ are the mean values
of $g$ on $O_{i},O_{j}$, for example 
$u_{i}=\Dfrac{1}{|O_{i}|}\Disp{\int_{O_{i}}}g$.

The algorithm proceeds as follows. 
The initial picture is partitioned in squares of side {\em size\_of\_grid}
pixels. 
Construct a list of the corresponding 
symbolic regions containing all the information needed (surface, sum 
of gray-levels in the square, length of boundaries).
Construct a table in which to each boundary is associated the 
value of $\lambda$ for which this boundary will disappear (i.e.
 $E$ decreases by merging the two adjacent regions).
Using and updating this ``merit'' table the algorithm proceeds.
The information on the new region, constructed when a merging decision occurred,
 is taken
from the two old one's, so there is no need to return on pixel level.

There are two (incompatible) options : either one wants the segmentation
to stop at {\em nb\_of\_regions} regions or the final segmentation is
at scale {\em lambda}.

If the first option is chosen the program stops when the desired 
number of regions ({\em nb\_of\_regions})
is reached. To be more precise it stops at the value of $\lambda$ for which this
particular number of regions is reached, there might be less regions
remaining if {\em nb\_of\_regions} doesn't correspond to a 2-normal segmentation
for any $\lambda$.

In case the final $\lambda$ is fixed the program stops at {\em lambda} or
for the value of $\lambda$ just lower than {\em lambda}. For example if
{\em lambda}=13.45 and the current $\lambda_c$=13.09, then if the next
boundary in the segmentation will vanish for $\lambda_n$=14.67 the program
stops. The obtained segmentation is also valid for each $\lambda$ strictly
lower than the next value $\lambda_n$.

If the program is used as module in another program the final values
of $\lambda$ and the number of regions are passed to the corresponding
variables. On command line execution this information is anyway displayed.

The program estimates the memory used by the process, this mainly depends
on the precision of the initial grid, for example $D=1$ is the best possible
precision, but it needs a lot of memory for the data structure 
(1 pixel = 1 region). So the program turns out to be most efficient 
(i.e. fast) on machines with a big amount of central memory. Indeed as there
is no previous information on the picture available and as regions can grow
in any direction, the whole structure should be available in central memory.

The output shows the initial quadratic error, the total boundary length and
the number of regions the algorithm starts with, after reaching 
{\em nb\_of\_regions} the quadratic error, boundary length, number of region
and the current value of $\lambda$.

The output is file {\em boundary} which represents the boundary set
on a white background and is stored under {\tt cimage}-format.

An optional output is file {\em curves}
 which contains the boundary set under 
{\tt curve}-format (use {\bf -c}).
Using {\bf curves\_cimage} the result can be visualized
on the original {\em fimage} for example.

To obtain the gray-level reconstruction of
the piecewise constant approximation use {\bf -r}.
This will write an {\tt fimage}
in file {\em reconstruction}.