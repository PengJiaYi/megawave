This module implements the classical snakes model using a
level set formulation~\cite{caselles.catte.ea:geometric}. 
It moves a snake image (image of contours) according to the content of a natural image \verb+ref+, in
order to fit the contours of the object enclosed in each initial contour.

The initial snake Fimage is \verb+in+ ($ u = u_0 $): the inside of each
snake is represented by values smaller than the threshold \verb+thre+ 
(\verb+-t+ option).
Such an image can be generated from a polys structure using the module 
\verb+fillpoly+.% (see fig 1).
The polys structure may itself be interactively created using the \verb+readpoly+ module. % (see fig 1).

The module \verb+lsnakes+ evolves $ u_0 $ with an evolution PDE,
and generates a new snake image \verb+out+.
This output can be used as a new input of \verb+lsnakes+, together with
\verb+ref+, to continue the motion.
When the last snake image is obtained, a final image containing the natural
image with the contours found can be generated using the modules
\verb+emptypoly+ and \verb+fmask+.

\vspace{1cm}

{\em Geometric Active Contours Equation :} 
  
  $$
    \left\{
    \begin{array}{ll}
\displaystyle
    \frac {\partial u}{\partial t} = g(x) \, |\nabla u|\,
\left(\mathrm{div}\frac{\nabla u}{|\nabla u|} + \nu\right) \qquad  
    (t,x) \in {[0,\infty[ \times R^2} \\
\\
    u(0,x) = u_0(x) \qquad 
    x\in{R^2}
    \end{array}     
    \right.
  $$

$$ \mathrm{with}\quad
g(x) = \frac{1}{1+|\nabla G_\sigma \ast u_0|^2}.$$
     
\begin{itemize}
\item $ \nu = $ positive real constant which represents a force.
\item $ g_0 = $ image where we are looking for the contour of the object. 
\item $ G_\sigma(x) = C\sigma^\frac{1}{2}\exp(-\frac{|x|^2}{4\sigma}) $ a Gaussian. 
\end{itemize}
   
%\begin{figure}%[htbp]
%\epsfxsize=10cm

%\input{dessin.eps}

%\caption{Dessin}
%\label{fig:dessin}
%\end{figure}  
