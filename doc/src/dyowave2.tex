{\em dyowave2} computes the orthogonal wavelet coefficients\index{wavelet!transform!dyadic}
of the floating point image stored in the file {\em Image}, 
using filter banks associated to an orthogonal basis of wavelets. 
The main difference with the {\em owave2} module is that the decimation 
is not performed for levels higher than {\em StopDecimLevel} (-d option). 
This means that if $j$ is greater than {\em StopDecimLevel}, then one has 
in the 1D formalism : 
\begin{eqnarray*}
A_{j}[n] = \sum_{k} h_{k-n} A_{j-1}[k] \\
D_{j}[n] = \sum_{k} g_{k-n} A_{j-1}[k] 
\end{eqnarray*}
instead of
\begin{eqnarray*}
A_{j}[n] = \sum_{k} h_{k-2n} A_{j-1}[k] \\
D_{j}[n] = \sum_{k} g_{k-2n} A_{j-1}[k] 
\end{eqnarray*}

If the -o option is selected, then one has (replacing {\em StopDecimLevel} 
by $j_d$ to simplify the notations) for $j > j_d$ and $0\leq m < 2^{j-j_d}$ : 
\begin{eqnarray*}
A_{j}[2^{j-j_d}n+m] = \sum_{k} h_{k-n} A_{j-1}[2^{j-j_d}k+m] \\
D_{j}[2^{j-j_d}n+m] = \sum_{k} g_{k-n} A_{j-1}[2^{j-j_d}k+m] 
\end{eqnarray*}
This makes possible to interpret the coeffients $\{A_{j}[2^{j-j_d}n+m]\}_n$ 
and $\{D_{j}[2^{j-j_d}n+m]\}_n$ as the coefficients of the wavelet transform 
of the original signal (or image) translated by a factor of $m2^{j_d}$.

The -r, -e, -p and -n options have the same meaning as for {\em owave1} and 
{\em owave2} modules. 
