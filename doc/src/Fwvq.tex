This macro compresses a 8 bits graylevel image using a vector quantization 
algorithm applied to the orthogonal/biorthogonal wavelet coefficients. 
It is based on the {\em fwvq} module, but is simpler to use. 

One only has to specify the prefix {\em codebook} of all codebook sets 
and the input image to be compressed. The macro then looks for files 
named {\em codebook}.cb, {\em codebook}\_x.cb, {\em codebook}\_y.cb, 
{\em codebook}\_q.cb, {\em codebook}\_xq.cb, {\em codebook}\_yq.cb, 
{\em codebook}\_qr.cb and {\em codebook}\_xqr.cb, and, if it finds them, 
uses them as input for {\em fwvq}. These files correspond respectively to 
CodeBook1, CodeBook2, CodeBook3, ResCodeBook1, ResCodeBook2, ResCodeBook3, 
ResResCodeBook1 and ResResCodeBook2. 
However, if one uses the {\em Fwlbg\_adap} macro, one does not have to 
bother about these files. They are all automatically generated by this macro. 
It suffices to put the same prefix name {\em codebook} as argument 
of {\em Fwlbg\_adap} and {\em Fwvq} (see documentation of {\em Fwlbg\_adap}). 
Note that {\em Fwvq} need at least the file {\em codebook}.cb (CodeBook1) 
in order to work. 

If the -R option is selected, then both a compressed file and a quantized 
image (which can be reconstructed from the compressed file) are generated. 
The compression is made in order to reach the target bit rate {\em Rate}. 
If the input image file has the form *.rim or *.img, then 
the compressed file is named *\_{\em Rate}c.comp and the quantized image file 
is named *\_{\em Rate}q.rim.

Details about wavelet transform are the same as for {\em Fwlbg\_adap} macro. 
The reader is refered to its documentation for information about the -b1, -b2, 
-o and -e options. 

Notice that the dimensions of the image should have a minimum of factor 2 
in their decomposition in prime numbers. If not then a part of the 
image as large as possible is extracted and having the required 
number of factor 2. 

The -r, -s, -u and -R options work as for {\em fwvq} module. If the -R 
option is not selected, then the rate distortion curve is computed 
(as if the -d option was selected in {\em fwvq}). 

The -m option works slightly differently. If not selected, then it 
performs an approximative memory allocation (equivalent to -m 1 
in {\em fwvq}). If selected, then it performs an exact memory allocation 
(equivalent to -m 2 in {\em fwvq}).

No scaling of wavelet coefficients is made. In other words, 
the -w option is not activated when {\em fwvq} is called. 

\begin{thebibliography}{999999}

\bibitem[GG]{kn:gg} A. Gersho, R.M. Gray, {\em ``Vector quantization 
and signal compression'', } Kluwer Ac. Pub., Boston (1992). 

\end{thebibliography}

