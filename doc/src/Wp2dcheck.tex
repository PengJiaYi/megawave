This macro checks and demonstrates the use of the 2D-wavelet packet representation.
The input image is first decomposed and reconstructed using a Daubechies' 
basis with a cardinal sine quad-tree. The error between the original image
and the reconstructed one is printed and you should check that it is very low : 
typically the MSE (Mean Square Error) is below the computer's floating point precision.
The representation i.e. the quad-tree containing computed 2D-wavelet packet coefficients,
is also viewed in a window.

Then, the macro demonstrates how one can denoise images by wavelet packets soft 
thresholding with cycle spinning : gaussian noise is added to the input image,
and the resulting noisy image is denoised using \verb+wp2doperate+.
The printed errors allow to verify that the objective MSE criterion is
decreasing by this denoising operation. Also for visual comparison, 
the noisy and denoised images are displayed in two windows. 
