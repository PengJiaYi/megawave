{\em ibiowave2} reconstructs an image from a sequence of sub-images forming 
a wavelet decomposition, using filter banks associated to biorthogonal bases 
of wavelets. 
The notations that are used here have been already defined for {\em owave1}, 
{\em biowave1}, and {\em biowave2}, so the reader is refered 
to the documentation of these modules to see their signification. 

{\em WavTrans} is the prefix name of a sequence of files containing 
the coefficients of a wavelet decomposition  $D^{1}_{1}, D^{1}_{2}, \ldots, 
D^{1}_{J}, D^{2}_{1}, \ldots D^{2}_{J}, D^{3}_{1}, \ldots, D^{3}_{J}, A_{J}$. 
{\em ibiowave2} computes \( A_{0} \), i.e. the inverse wavelet transform 
of {\em WavTrans}. 

The algorithm has the same 2D structure as the one implemented in 
{\em iowave2}. The difference lies in the fact that at each step, 
a biortogonal 1D inverse transform is applied to the lines and 
columns of average and detail sub-images instead of an orthogonal one. 

The edge processing modes are the same as those of {\em ibiowave1}.

The complexity of the algorithm is roughly the same as for {\em biowave2}.

The name of the files containing the sub-signals 
$D^{1}_{1}, D^{1}_{2}, \ldots, D^{1}_{J}, D^{2}_{1}, \ldots D^{2}_{J}, 
D^{3}_{1}, \ldots, D^{3}_{J}, A_{J}$ must have the same prefix 
and their syntax obeys the rules described in {\em owave2}. 
The sample values of the reconstructed signal are stored in the file 
{\em RecompImage}. 

The coefficients $(h_{k})$ and $(\tilde{h}_{k})$ of the filter's impulse 
responses are read in the file {\em ImpulseResponse1} 
and {\em ImpulseResponse2}.

\begin{itemize}
\item
The -r, -h and -e options are exactly the same as in the orthogonal inverse 
transform. See {\em iowave2} module documentation for explanations. 
\item
The -n option specifies the normalisation mode of the filter impulse 
responses' coefficients. It should be selected according to the mode 
used for the decomposition in order to get exact reconstruction. 
\end{itemize}

These options should be tuned in the same way as they were 
for the decomposition (with the {\em owave2} module). 

\begin{thebibliography}{CDF}

\bibitem[CDF]{kn:cdv} A. Cohen, I. Daubechies, J.C. Feauveau. 
{\em Biorthogonal bases of compactly supported wavelets, } 
Comm. pure and applied math. 45, pp 485-560, 1992.

\end{thebibliography}
