This module returns the type of junction of the given point $(x_0,y_0)$
of the input cimage $U$. 
It returns $0$ if this point is not a junction, $1$ if it is a T-junction
and $2$ if it is a X-junction.
For a description of the Junction Detection Algorithm, please
see~\cite{KP}.

The option \verb+-a+ allows to choose the area threshold associated
to the detection, while the option \verb+-q+ is related to the quantization 
threshold. Those thresholds are used to avoid false detections.

The options \verb+-l+ and \verb+-m+ allow to get the $\lambda_1$ and
$\mu_1$ values computed by the algorithm, when  $(x_0,y_0)$ is a junction.
The borders of the two significant level sets 
$L_{\lambda_1} = \left\{(x,y) /U(x,y) \leq \lambda_1 \right\}$ and 
$M_{\mu_1} = \left\{(x,y) /U(x,y) \geq \mu_1 \right\}$ are going 
through the junction.

The options \verb+-x+, \verb+-y+, \verb+-X+ and \verb+-Y+ allow to get
the points $(x_{\lambda_1}, y_{\lambda_1})$ 
and $(x_{\mu_1}, y_{\mu_1})$ of the
four neighbour points of the junction, so that 
$(x_{\lambda_1}, y_{\lambda_1})$ belongs to the border of 
$L_{\lambda_1}$ and  $(x_{\mu_1}, y_{\mu_1})$ to $M_{\mu_1}$.

\begin{thebibliography}{Dau1}
\bibitem{KP} V.Caselles, B.Coll, J.M. Morel.
{\em A Kanizsa Programme}, Preprint CEREMADE.
\end{thebibliography}