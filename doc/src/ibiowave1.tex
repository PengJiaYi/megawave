{\em ibiowave1} reconstructs a signal from a sequence of sub-signals forming a wavelet decomposition\index{wavelet!transform!biorthogonal}, using filter banks associated to biorthogonal wavelet bases (see~\cite{cohen.daubechies.ea:biorthogonal}). 
The notations and definitions have been defined in {\em owave1} and {\em biowave1}. The coefficients of the wavelet decomposition \( A_{J}, D_{J}, D_{J-1}, \ldots, D_{1} \) are stored in a sequence of files whose names obey the syntaxic rules described in {\em biowave1} module's documentation, and whose prefix is {\em WavTrans}. The reconstructed signal's coefficients $A_{0}[k]$ are stored in the file {\em RecompSignal}.

As for the orthogonal case the inverse wavelet transform is a recursive algorithm.  \( A_{j-1} \) is computed from \( A_{j} \) and \( D_{j} \) using the relation :
\[
A_{j-1}[k] = \sum_{l} h_{k-2l} A_{j}[l] + \sum_{l} g_{k-2l} D_{j}[l]
\]

The edge processing methods are corresponding to those described for {\em biowave1}.

The complexity of the algorithm is roughly the same as that of {\em biowave1}.

The sample values of the reconstructed signal are stored in the file {\em RecompSignal}. 

The coefficients $(h_{k})$ and $(\tilde{h}_{k})$ of the filter's impulse responses are stored in the file {\em ImpulseResponse1} and {\em ImpulseResponse2}.

\begin{itemize}
\item
The -r option specifies the number of levels $J$ in the decomposition.
\item
The -e option specifies the edge processing mode (see {\em biowave1}).
\item
The -n option specifies the normalisation mode of the filter impulse responses' coefficients. It should be selected according to the mode used for the decomposition in order to get exact reconstruction. 
\end{itemize}
