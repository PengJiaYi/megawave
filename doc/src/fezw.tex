This module compresses a floating point image contained in file Image using 
Shapiro's EZW algorithm applied to its orthogonal/biorthogonal wavelet 
transform. 

A wavelet transform is first applied to the image. If the -b option is 
selected, then a biorthogonal transform is applied using the filter pair 
contained in {\em ImpulseResponse} and {\em ImpulseResponse2}. 
The edge processing is made by reflecting the image across the edges. 
If the -b option is not selected, then an orthogonal transform 
is performed using the filter contained in {\em ImpulseResponse} and, 
if the -e option is selected, the special filters for edge processing 
contained in {\em EdgeIR} (only for Daubechies wavelets). 
If neither the -e and -b options are selected, then an orthogonal transform 
is performed with periodization of the image. 

The -n option enables to control the filters normalisation. It has the same 
effect as in {\em owave1} (if an orthogonal transform is performed) or 
{\em biowave1} (if a biorthogonal transform is performed) modules.

The -r option specifies the number of level of wavelet transform (NLevel 
must be a positive integer). 

After the wavelet transform has been computed, it is compressed 
with the EZW algorithm. It generates both a compressed representation, 
which is stored in {\em Compress} (if the -o option is selected) and 
a quantized wavelet transform, which can be reconstructed with 
the data contained in {\em Compress}. A inverse wavelet transform is then 
applied to the quantized wavelet transform in order to obtain the quantized 
image which is stored in {\em Qimage} (if selected). This is the image 
which can be reconstructed by applying the decompression module {\em fiezw} 
to {\em Compress}. 

The -w, -d, -R, -P and -s options have the same effect 
as in {\em ezw} module.  

