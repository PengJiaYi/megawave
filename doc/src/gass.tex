This module implements numerically the Affine Scale Space of curves.
An initial curve $s\mapsto {\mathbf C}(s,0)$ evolves in function 
of time $t$ according to the 
equation
$$\frac{\partial {\mathbf C}}{\partial t}(s,t) 
= \kappa^{1/3}(s,t) \; {\mathbf N}(s,t),$$
where $\kappa$ represents the local curvature of ${\mathbf C}$ and
${\mathbf N}$ the normal vector to ${\mathbf C}$. This equation is
the geometric equivalent of the scalar Affine Morphological Scale
Space (see module \verb+amss+).
The Affine Scale Space is the only regular semigroup that preserves 
inclusion and that commutes with special affine transforms of the plane.

\medskip

The numerical scheme used in this module is based on the iteration
of a geometric semi-local operator called {\em affine erosion}.
This operator, described in~\cite{moisan:affine}, is ``fully consistent'' in the
sense that it satisfies all the axioms that define the Affine Scale
Space (except the semigroup property of course, otherwise it would be
the Affine Scale Space itself). Indeed, it preserves inclusion
and commutes with special affine transforms.

\medskip

This module computes the evolution of each curve of the Dlists {\em in}
and puts it in the Dlists {\em out}.
The parameter {\em last} defines the final scale (time) of evolution.
This scale is normalized in such a way that a circle with radius $t$
should vanish exactly at scale $t$. Since this normalization is not 
linear on $t$, it imposes to specify a {\em first} scale when the 
evolution has to be iterated. The {\em eps} parameter tells the program
how precise it should be in the representation of curves. Combined
with the {\em r} parameter, it defines the spatial quantization step
$\delta = 10^{-eps} \cdot r$. 
All the curves used in the algorithm will
be sampled in such a way that the distance between 2 successive points
lies in $[\delta,2\sqrt 2\delta]$. If not automatically set, the {\em r}
parameter is defined by the maximum distance to origin given by an
input point.

The default values of {\em eps} is 3. For more accurate computations,
you can use greater values (3.5, 4, 5, ...) but remember that
the cost in memory and in computation time depends linearly on $10^{eps}$.

\medskip

The scale quantization can be fixed in two ways: either by specifying
a maximal scale step or by specifying a minimum number of iterations.
These are only bounds since the algorithm may (and will) have to change
this value for some iterations due to the fact that inflexion points
vanish. Hence, the actual scale step will be smaller and the actual
number of iterations will be grater. The default is a minimum of 5 iterations.

\medskip

The \verb+-v+ option displays some information at each iteration :
the number of convex components found at the begining of the iteration,
the effective scale of erosion applied, and the total evolution scale
achieved so far.

