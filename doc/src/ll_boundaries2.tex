
This modules selects contrasted level lines from a Fimage.
It brings new features to the former \verb+ll_boundaries+ module.
(these two modules will probably be merged in the next release).

\medskip

Because of gradient quantization effect, it may be very useful to slightly 
smooth the image before applying the method. A gaussian with standard deviation
0.5 is the default kernel. 

\medskip
In the original algorithm, the gradient distribution is computed on the whole 
image (see \cite{desolneux.moisan.ea:edge}), sometimes yielding what is called the ``blue sky effect'':
there are too many detections in some regions because the image contains a very
flat part. This can be counter-balanced by using the following method (detailed
in \cite{cao.muse.sur:shape}): since meaningful boundaries are closed curves, they sever the plan
into two connected components. It is then possible to re-estimate the gradient
distribution in each connected component and apply the same method as
above. Use the \verb+-L+ flag for this local research. 


\medskip
Notice that contrary to the \verb+ll_edges+ detection, this 
module is constrained to keep full (ie closed) level lines and 
cannot break them into parts. As a consequence, the detected boundaries still
enjoy the original tree structure and are the level lines of a function. 
Meaningful boundaries thus allow to define a ``connected operator'' (following
the mathematical morphologists terminology). The reconstructed image can be
obtained with the \verb+-o+ option, while the corresponding tree can be
saved with the \verb+-k+ option. 


\medskip

The result is a collection of curves stored in a Flists structure.

