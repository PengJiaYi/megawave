This module compresses a graylevel image (fimage) using a vector quantization 
algorithm. {\em fvq} supports both classified and multistaged/residual 
vector quantization (see \cite{kn:gg}). 

The classification is based on the energy of the coefficients vectors. 
The codebooks for the first class (vectors with the highest energy) 
are contained in the CodeBook1 file.  The codebooks for the second and third 
classes (the energy of vectors dereases with the index class) 
are contained respectiveley in the CodeBook2 and CodeBook3 files. 
The values of energy thresholds which separate the different classes 
are read in the codebook files. 

ResCodeBook1, ResCodeBook2 and ResCodeBook3 contain codebooks 
for the quantization of the residual vectors coming from the 
quantization with CodeBook1, CodeBook2 and CodeBook3 respectively 
(second stage quantization). 
ResResCodeBook1 and ResResCodeBook2 contain codebooks 
for the quantization of the residual vectors coming from the 
quantization with ResCodeBook1 and ResCodeBook2 respectively 
(third stage quantization). 

Compress is the output compressed file. Notice that all lists 
of quantization symbols are arithmetically encoded using the 
{\em arithm\_encode2} module (see the documentation of this module 
for further details). 

QImage is the quantized image, which can be reconstructed 
from Compress. 

The -h option indicates that only a reduced header (not including 
the dimensions of image) should be inserted at the beginning 
of the Compress file. 

When one performs classified vector quantization, one has to encode 
in addition the index of the class of each quantized vector in order 
to know which codebook was used to quantize it. This list of indices 
can be considered as a bitmap. When the codebooks used to quantize each class 
have size one, then the vectors are quantized to 0. In this case, it is 
normally useless to encode the bitmap of indices. The -M option 
specifies to encode this bitmap anyway (this option is used by {\em fwvq} 
module). Notice that this bitmap is also arithmetically encoded. 

The -m option enables to compute a discrete rate-distortion curve. 
It only makes sense when the different codebook files contain more 
than one codebook.

The -n, -X, -Y and -Z options specify which codebook should be used 
in CodeBook1, CodeBook2, CodeBook3 and CodeBook4 respectively 
to perform the quantization. This supposes that these files contain 
more than one codebook and that the -m option is not selected.   

Likewise, the -A, -B, -C, -D, -E and -F options specify which codebook 
should be used in ResCodeBook1, ResCodeBook2, ResCodeBook3, ResCodeBook4, 
ResResCodeBook1 and ResResCodeBook2 respectively 
to perform the quantization. 


\begin{thebibliography}{999999}

\bibitem[GG]{kn:gg} A. Gersho, R.M. Gray, {\em ``Vector quantization 
and signal compression'', } Kluwer Ac. Pub., Boston (1992). 

\end{thebibliography}
