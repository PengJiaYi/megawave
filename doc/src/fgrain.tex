This module applies the so-called grain filter to an image \cite{Masnou}.
This is a filter close to the extrema killer of Vincent \cite{Vincent}, with
one major difference: it is self-dual, meaning that it commutes with the
negative operator,
$$
T(-u) = -T(u).
$$
Its other properties are mainly that it is morphological, affine invariant,
idempotent,
it does not destroy T-junctions and does not move level lines. It is a
semi-local operator, but not local (no associated PDE).
As it is a morphological filter, it can be written as an inf-sup, and the
sup-inf version is the same (it is simultaneously an opening and a closing).

This filter removes the shapes of an image (see module \verb+flst+) that
have a too small area. So this requires the datum of a threshold of area, this
area could be interpreted as the scale to define a scale-space
\cite{ss99,fllt_theory}.
It is a good example of a very simple use of the modules \verb+flst+ and
\verb+flst_recontruct+.

\begin{thebibliography}{9}
\bibitem{Vincent}
Vincent, L.:
Grayscale Area Openings and Closings, Their Efficient
  Implementation and Applications,
Proc. of $1^{st}$ Workshop on Math. Morphology and its Appl.
  to Signal Proc., J.~Serra and Ph. Salembrier, Eds. (1993) 22--27
                                                                               \bibitem{Masnou}
Masnou, S., Morel, J.M.:
Image Restoration Involving Connectedness,
Proc. of the $6^{th}$ Int. Workshop on Digital I.P. and Comp.
  Graphics, SPIE {\bf 3346}, Vienna, Austria (1998)
                                                                               \bibitem{fllt} Monasse, P., Guichard, F.:~
Fast Computation of a Contrast-Invariant Image Representation,
to appear in IEEE Trans. on Image Processing,
Preprint CMLA 9815, available from {\texttt
http://www.cmla.ens-cachan.fr/index.html} (1998)

\bibitem{ss99} Monasse, P., Guichard, F.:~
Scale-Space from a Level Lines Tree,
Proc. of 2nd Int. Conf. on Scale-Space Theories in Computer
Vision, Corfu, Greece, LNCS 1682, pp. 175--186.

\bibitem{fllt_theory}
Monasse, P.:
An Inclusion Tree Describing the Topological Structure of an Image,
in preparation.
\end{thebibliography}
