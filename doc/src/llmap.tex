This module produces a basic representation of the level lines 
\index{level line}
borders of an 
image. To get a more sophisticated representation, see the modules 
\verb+ml_decompose+ and \verb+ml_draw+.
Because this is a bitmap representation, borders at the same location are
represented only once : the borders are also borders of the iso lines.

The parameter \verb+level_set+ is a step on the gray level. If set to $1$, 
all the level lines are caught. If set to a larger value then only the
level lines coresponding to a gray level value which is a multiple of  
\verb+level_set+ are catched.

For a better representation of the level lines, we invit the user to 
do a zoom of factor $2$ on the input image, before calling this module.

The option \verb+-t+ allows to code the type of border : not a border 
corresponds to value $255$, left border to $200$, upper border to $100$ 
and corner (up and left border) to $0$. Without this option, only the
values $255$ (not a border) and $0$ (any type of border) are used.
The map obtained with the option \verb+-t+ allows, together with the
gray levels sequence of the image (one pixel for each connex component,
see module \verb+encodelevels+), to reconstruct the original image
(see module \verb+glmaprecons+).
