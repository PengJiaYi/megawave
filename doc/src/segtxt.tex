{\it segtxt} is an easy and fast 
way to get a segmentation from a textured input fimage.\\
This module uses directly the modules {\it mschannel} and {\it msegct} in order to have a direct segmentation 
of the input fimage. So the user has to enter both the parameters for the {\it mschannel}
 module and the {\it msegct} one. The only difference between 
 this module and {\it mschannel} is the number of desired regions that the user wants to reach (needed for the {\it msegct} 
 module). An output Cimage is then required.\\
 Same as {\it mschannel}, the user must enter an output fmovie. This one is used for {\it msegct}.
 Moreover, it can be usefull for the user to see the output fmovie so as to change the parameters
  if the segmentation is not correctly computed. Just by viewing the output fmovie, the user can fixed the parameters.\\
  In fact, the result of the segmentation depends of the kind of the input fimage. 
 Be not surprised if, at the first time, you don't get what you are expected. 
 Generally, the first parameter {\it N} has to be set between 2 and 6. (more if your computer has plenty of memory), it depends if the  input fimage contains a lot of litlle details. 
 In this case, {\it N} has to be more than 4. Otherwise if your image is not very complex, default parameters could be good.\\
 The others parameters are also important, be carfull in using the parameter {\it S - standard deviation} used for the smoothing filter {\it fsmooth}, 
 no more than 3, if not, the segmentation won't be good and the computing time huge. Set 
 {\it S} to 1 or 2. Same for {\it W} (1 or 2). At last {\it p=1} is better if the image is not
 very contrasted.\\
 
 {\Large \bf Reference}\\  
 
 {\it A Multiscale Algorithm For Image Segmentation By Variationnal Method, G. Koepfler, C. Lopez, J.M. Morel, Society for Industrial And Applied Mathematics.} 
 
