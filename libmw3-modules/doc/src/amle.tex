This module implements the numerical scheme associated to the
AMLE model (absolutely minimizing Lipschitz 
interpolant, see the main reference~\cite{caselles.morel.ea:axiomatic} 
and~\cite{aronsson:extension}\cite{cao:absolutely}\cite{froment:functional} for
related works).
\index{Absolute Lipschitz Minimizing Extension}
\index{AMLE|see{Absolute Lipschitz Minimizing Extension}}
\index{interpolation!AMLE}

The following evolution problem
\[
\frac{\delta u}{\delta t} = D^2 u \left( \frac{Du}{\mid Du \mid }, \frac{Du}{\mid Du \mid } \right) \; \mbox{ with } u(0,x)=u_0(x)
\]
is solved using an implicit Euler scheme.

The input image corresponds to the initial function $u_0$, and the output
image to the solution $u(t,x)$ at the time $t = n \times h_t$ where
$n$ is the number of iterations and $h_t$ the time increment.

In practice, the input image must have the value $0$ at pixel locations where
no initial data have to be set (e.g. unknown values) and non $0$ values
to set initial data. 
The output image has the same values that the input image for the
pixel locations set as initial data. For the other locations (set to $0$ in
the input image), the diffusion process of AMLE interpolates a new value.

You may use the module \verb+amle_init+ to create the input image
from an original image, this last one being typically a quantized image:
the AMLE model will reconstuct (or ``dequantize'') the image by
interpolating the missing level lines.

It is possible to make the interpolation faster by setting values closed
to the reconstructed values at pixel locations which do not correspond
to the initial data (i.e. at locations set to $0$ in the input image).
To do so, include the image of the estimated values using the \verb+-i+
option.
In practice, the estimated image is the quantized image put in the input
of the \verb+amle_init+ module.


