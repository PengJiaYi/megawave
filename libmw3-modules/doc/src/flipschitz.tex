This modules iterates on an image $in$ the operator
$$Tu(x) = \sup_{\delta \in F} \left( \frac{}{}
u(x+\delta)-C\|\delta\| \right),$$ 
(or the opposite process
$$Tu(x) = \inf_{\delta \in F} \left( \frac{}{}
u(x+\delta)+C\|\delta\| \right)$$
if the \verb+-i+ option is set), with $C=lip$.
The set $F$ is a ``structured element''. In general, it is a disc
(whose radius $r$ can be specified with the \verb+-r+ option), but
a general set (shape $s$) can be used.

\medskip

If $F$ is the whole plane, then the operator $T$ is idempotent 
($T^2=T$) and computes the $C$-lipschitz 
\index{lipschitz}
sup-enveloppe of $in$
(or inf-enveloppe with the \verb+-i+ option), that is the smallest
image $v$ such that $u\leq v$ and $v$ is $C$-lipschitz. In the
continuous domain, since one has $T_r T_r = T_{2r}$ (here
``$T_r$'' means that a disc with radius $r$ is used for $F$),
then it is equivalent to use a finite $r$ and iterate the process.
In the discrete domain it is not exactly true, but since the complexity
is $O(n\cdot r^2)$ it is often more efficient (though less precise)
to iterate the process with a small $r$.

\medskip

Note that when $C=0$, this modules computes the dilation (or erosion
with the \verb+-i+ option) of $u$.


