This module computes the bit rate needed to encode
the origin points of the curves contained in the input set.
The bit rate is given in bit per point, the number of points being
the total number of points contained in all curves.

Two structures are considered to represent the coordinates of the
origin points : a binary bitmap (a pixel is set to $1$ if it is
an origin point, or to $n$ if it is a multiple origin point of multiplicity 
order $n$) and the sequence of the distances between two successive
origin points, the points being scanned in the natural order (from left
to right and up to down).

Several coding schemes are considered over all the following algorithms,
to code the former structures :
entropy coding, arithmetic coding, predictive arithmetic coding.

The bit rate returned by the function is the best bit rate over the
two structures and over all coding schemes.
The function outputs also the total number of points \verb+N+
and the number of bits \verb+B+ needed to code the structure.

Use the module \verb+cvsencode+ to get the total bit rate to encode
a set of curves.
See the module \verb+arencode2+ for more information on the arithmetic
coding algorithms.
