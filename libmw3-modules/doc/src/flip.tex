With this module, two images $in1$ and $in2$ can be interactively compared.
The image $in1$ is displayed on a window, and each time the ``space'' key
is pressed the window flips between $in1$ and $in2$. Thus, even if the
two images differ only slightly, an apparent motion will be perceived when 
the ``space'' key is pressed, so that the two images can be accurately 
compared. Several features come in addition to this basic alternate
display:
\index{comparison!of two images}
\index{visualization!of two images}

\medskip

- Images can be zoomed with the right mouse button, and unzoomed with the
``u'' key. The initial zoom (and thus the window size) is specified 
with the \verb+-z+ option. The zoom method (interpolation order) is
specified with the \verb+-o+ option (default is 0, nearest neighbor 
interpolation), and can be interactively changed with the ``o'' key.
The cursor keys can be used to move the displayed window.

\medskip

- Contrast can be changed with the ``c'' key, causing renormalization
of the grey level scale for the current displayed (sub-)image. This
grey level scale is then kept when further zooming, unzooming, flipping,
etc. until the ``c'' key is pressed again (causing a new renormalization).

\medskip

As the left mouse button is pressed, the name of the current displayed 
image is shown. Each time it is pressed again, other informations 
are shown (interpolation method and currently displayed window).  

\medskip

If the middle mouse button, then initial settings (zoom, contrast) are 
restored. As usual, pressing the ``h'' key causes a list of all available
interactive commands to be displayed.
