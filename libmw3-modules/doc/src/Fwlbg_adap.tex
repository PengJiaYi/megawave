This macro generates a complete set of codebook sequences for the quantization 
of image wavelet transforms\index{quantization!vector}. 
It is based on the {\em fwlbg\_adap} module, but is simpler to use. 

One only has to specify the prefix {\em codebook} of all generated 
codebook sets and at least one (maximum four) input image for the 
training sets. The macro generates codebook sequences 
CodeBook1, CodeBook2, CodeBook3, ResCodeBook1, ResCodeBook2, ResCodeBook3, 
ResResCodeBook1 and ResResCodeBook2, which can be used directly 
as input to the compression module {\em fwvq}. These are put in the files 
{\em codebook}.cb, {\em codebook}\_x.cb, {\em codebook}\_y.cb, 
{\em codebook}\_q.cb, {\em codebook}\_xq.cb, {\em codebook}\_yq.cb, 
{\em codebook}\_qr.cb and {\em codebook}\_xqr.cb respectively. 
However, if one uses the {\em Fwvq} and {\em Fwivq} macros, 
one does not have to bother about these files. It suffices to put 
the same prefix name {\em codebook} as argument of 
{\em Fwvq} and {\em Fwivq} (see documentation). 

The default wavelet transform is a biorthogonal one using 
7/9 Cohen--Daubechies filter pair (see \cite{cohen.daubechies.ea:biorthogonal}). 
The filters are read in the h/sd07.ri and htilde/sd09.ri 
which must be available either in the MegaWave2 data directory or 
data/quantization subdirectory. 
However one can use other filter pairs (using -b1 and -b2 options) 
or switch to orthogonal transform, using -o option 
(possibly with the -e option if special edge processing is requested, 
see {\em owave2} module documentation for details). 
The normalization mode is the default one. 

Codebook sequences are generated for details from level 1 to 4, and 
for average at level 3 and 4. The size of blocks is $4\times 4$ at 
levels 1 and 2, $2\times 2$ at levels 3 and 4.

The overlapping option is activated, as well as the default decimation 
stop (at level 2) option. 

The maximum size of codebooks are 
\begin{itemize} 
\item 4096 for average codebooks. 
\item 1024 for detail codebooks in CodeBook1, CodeBook2, ResCodeBook1, 
ResCodeBook2, ResResCodeBook1 and ResResCodeBook2.
\item 4096 for detail codebooks in CodeBook3 and ResCodeBook3.
\end{itemize}

The threshold value options are set to 3.0 and 1.0. 

This option setting is well adapted to the construction of a codebook set 
from 2,3 or 4 $512\times 512$ images. It enables to get rather high bit rates 
when using {\em fwvq} module or {\em Fwvq} macro with the generated codebook 
set. However, the running time might be very long (several days!). 
In order to perform a quick test, one has to modify the maximum sizes 
of codebooks directly in the macro file. Typically, one should put 
sizes not greater than 16. However, one won't be able to get 
very high bit rate compression when using these codebooks.

