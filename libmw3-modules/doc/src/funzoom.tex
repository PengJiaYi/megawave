This module unzooms an image by a factor $z$. The input image is first
projected orthogonally onto the space of B-splines \index{B-spline}
of order $n$ (see
the \verb+fcrop+ module). The order $n$ can be specified 
by the \verb+-o+ option. Then, an appropriate subsampling is performed.
Hence, the $L^2$ norm of the error caused by image reduction is minimized.
This method, described in~\cite{unser.aldroubi.ea:enlargement}, allows to 
avoid undesirable artifacts like aliasing. 
\index{reduction!of an image}
\index{zoom out!of an image}

\medskip

$\bullet$ 
If $n=0$, the method is a simple averaging under the assumption that pixel
are adjacent squares. The input image $u$ is supposed to take constant values
in each $1\times 1$ square (nearest neighbor interpolation), and the output
value at a given pixel is given by the average of $u$ on the corresponding
$z\times z$ square. In particular, this yields the classical simple 
``block averaging'' for integer reduction factors.

\medskip

$\bullet$ 
For $n= 1$, the algorithm consists in two steps~:

1. a zoom by a certain cubic (non-regular) spline, 
$\frac 1z \beta^1_z * \beta^1$, where $\beta^1$ is the 
B-spline of order 1 and $\beta^1_z(x)=\beta^1(x/z)$.

2. an inverse spline transform of order 3, performed by the
\verb+finvspline+ module.

\medskip

$\bullet$ 
For $n> 1$, the reduction is realized on a similar basis but with higher 
order splines.
