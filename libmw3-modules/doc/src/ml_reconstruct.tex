Using the {\tt morpho\_lines} of {\em m\_image} this 
module reconstructs an {\tt fimage} {\em image\_out}.

\smallskip

The background of {\em image\_out} is set to the minvalue
of {\em m\_image} and the list of {\tt morpho\_lines}
is read from the first to the last element. 
For each {\tt morpho\_line} the interior is set to
the minvalue associated to the current line.\\
This allows to reconstruct exactly an image decomposed
by {\bf ml\_decompose} with option {\bf -o 0}.

\medskip

If the {\bf -i} flag is on, the background of {\em image\_out} 
is set to the maxvalue of {\em m\_image} and the list of 
{\tt morpho\_lines} is read from the first to the last element.
For each {\tt morpho\_line} the interior is set to the
maxvalue associated to the current line.\\
This allows to reconstruct exactly an image decomposed
by {\bf ml\_decompose} with option {\bf -o 1}.

\medskip

To reconstruct an {\tt m\_image} containing isolines
it doesn't matter if the {\bf -i} flag is on or not.

\medskip

If you have transformed the {\tt morpho\_lines}, you
have to verify  that the points are 4 adjacent and 
that the conventions, defined in {\bf ml\_extract}, are verified
by the coordinates. Else the reconstruction will fail.

\smallskip

The program always draws the {\tt morpho\_sets} following the order
in which the {\tt morpho\_lines} appear in {\em m\_image}, 
the (background) color being defined by the {\bf -i} flag.\\
Thus you are supposed to know the order in which the {\tt morpho\_lines}
 appear in the input of {\bf ml\_reconstruct}.
