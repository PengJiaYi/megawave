This module gets back a list of symbols which have been arithmetically encoded 
with the {\em arithm\_encode2} module according to the algorithm 
described in \cite{witten.neal.ea:arithmetic}. 

The -r option enables to specify the number of rows in the Output fimage. 
If it is not selected, then it is computed in order that the 
blank space after the list of symbols is as small as possible. 

The -n option enables to specify the size of the output symbol alphabet. 
This option should be selected if the -H option was not activated 
at the encoding phase. Otherwise this information is included in the header. 

The -c option has the same meaning as for {\tt arithm\_encode2}. 
It should be tuned in the same way as in the encoding phase. 

The -p option tells wether predictive encoding has been used or not 
during the encoding phase. Predic equal to 0 (resp. 1) means that encoding 
was not (resp. was) predictive. If this option is not activated, 
then it means that the information is contained in the header and thus 
that the -H option has been activated during the encoding phase. 

If the -h option is activated, this means that the histogram contained 
in the fsignal Histo has been used as the fixed source distribution histogram 
during the encoding phase. Thus it should be used again for decoding.  
This option should be tuned in the same way as in the encoding phase. 
