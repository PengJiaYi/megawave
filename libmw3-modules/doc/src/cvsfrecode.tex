This module computes the bit rate needed to encode
the points of the curves contained in the input set, but
the origin points.
The bit rate is given in bit per point, the number of points being
the total number of points contained in all curves.
The curves are supposed to be made by connected points 
(for the 4-connectivity).

This module implements the Freeman's 
algorithm~\cite{freeman:computer}\index{coding!curve} : 
the data structure is made by the change of direction from one point to the next.
After that, the data is coded using one of the following schemes
(only the best bit rate is returned) :
entropy coding, arithmetic coding, predictive arithmetic coding.

The function outputs also the total number of points \verb+N+
and the number of bits \verb+B+ needed to code the structure.

Use the module \verb+cvsencode+ to get the total bit rate to encode
a set of curves.
See the module \verb+arencode2+ for more information on the arithmetic
coding algorithms.

Such module is useful to encode level lines geometry, such as 
in~\cite{froment:functional}.
