This module allows to visualize the level lines 
\index{level line!visualization} and the level sets 
\index{level set!visualization} of
the input image. This image can be a Fimage, but this module only handles 
level lines corresponding to integer values.
It is designed to be used interactively (with a control
window), but the parameters can be preset with the module options.

\medskip

{\bf \verb+-d+ option (display mode) :}

0: no level structure displayed

1: level lines for all levels $\lambda$ such that $\lambda = l \;\;{\mathrm mod}(s)$

2: level lines for one level only ($\lambda=l$)

3: lower level set, that is $\{x;\; u(x)<l\}$

3: upper level set, that is $\{x;\; u(x)\geq l\}$

4: bi-level set, that is $\{x;\; l \leq u(x)<l+s\}$

\medskip

{\bf \verb+-b+ option (background mode) :}

0: original image

1: attenuated image (grey levels rescaled between 64 and 255)

2. no background

\medskip

{\bf \verb+-i+ option (interpolation mode) :} see \verb+fcrop+ module

\index{interpolation!of an image}

0: nearest-neighbor interpolation (level lines surround squares)

1: bilinear interpolation (level lines are piecewise hyperboles)

2: interpolation with bicubic Keys' function 

3: cubic spline interpolation

4: spline interpolation (order 5)

5: spline interpolation (order 7)

\medskip

The \verb+-z+ option sets the size of the display window by multiplying
by the specified factor the size of the original image. This factor
cannot be set interactively. The \verb+-l+ and \verb+-s+ options 
set the current level and the level step. They can also be set interactively
using the bars and buttons in the control window. In the display window,
the following events are handled:

\medskip

{\bf left mouse button:} set current level to the value present at mouse location

{\bf middle mouse button:} restore the original display region

{\bf right mouse button:} zoom by a factor 2 (center at mouse position)

{\bf keys 'u' and 'U':} unzoom by a factor 2

{\bf keys 'q' and 'Q':} quit 

\medskip

The -o option permits to save the last display in a Ccimage.
It can be combined with the \verb+-n+ option (no display) to
use \verb+llview+ as a non-interactive module.


